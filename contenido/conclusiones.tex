\chapter{Conclusiones y recomendaciones}
\label{ch:conclusiones}
\section{Conclusiones}
Se detallan las conclusiones correspondientes con cada objetivo específico planteado:

\begin{itemize}

    \item \textit{Estudiar los aspectos concernientes al desarrollo de sistemas de conducción autónoma y sistemas de aprendizaje.}
    
    \textbf{Conclusión:} Se estudiaron los aspectos relacionados al desarrollo de sistemas de conducción autónoma, en principio, 
    analizando su origen y potencialidad en la Sección(\ref{sec:antecedentes}). Para posteriormente estudiar los conceptos básicos 
    de aprendizaje 
    automático, aprendizaje profundo, redes convolucionales y el proceso de entrenamiento de una red neuronal.

    \item \textit{Analizar los requerimientos de un sistema de conducción autónoma capaz identificar y mantener su carril 
    mientras se conduce.}

    \textbf{Conclusión:} Se ha introducido el esquema de la arquitectura de un sistema de conducción autónomo y se han planteado 
    los requisitos y funcionalidades que debe tener el sistema en su conjunto en la Sección(\ref{sec:arquitectura}) y de las 
    características de los 
    subsistemas de los que se compone: el subsistema de control y actuación en la Sección(\ref{sec:esqcontrol}), el subsistema de adquisición de 
    datos y entrenamiento en la Sección(\ref{sec:esqdaq}), el subsistema de inferencia y control autónomo en la Sección(\ref{sec:esqinferencia}). 

    \item \textit{Diseñar la arquitectura de un sistema de conducción autónoma en base a los requerimientos previamente 
    establecidos.}

    \textbf{Conclusión:} Se ha planteado la arquitectura general del sistema en base a las características analizadas en 
    el Capítulo(\ref{ch:introduccion}) y (\ref{ch:m_teorico}) en la Sección(\ref{sec:arquitectura}). Considerando todos los alcances
    y limitaciones planteados por el proyecto en el inicio. 
    Por su parte, también se han definido las herramientas de hardware en la Sección(\ref{sec:software}) y las herramientas 
    de software en 
    la Sección(\ref{sec:software}) que hicieron posible la implementación de todos los subsistemas asociados.

    \item \textit{Diseñar el subsistema de adquisición de datos y entrenamiento para tareas de conducción autónoma.}
    
    \textbf{Conclusión:} Se ha diseñado el Subsistema de Adquisición de Datos y Entrenamiento en base a la arquitectura planteada 
    en la Sección(\ref{sec:esqdaq}) en la Sección(\ref{sec:daq}) considerando los detalles de implementación, algoritmos y características necesarias para 
    las tareas de adquisición de datos, aumentación de datos y entrenamiento de modelos de redes neuronales explorados en el Capítulo(\ref{ch:ingenieria}).
    Los detalles del proceso de diseño y entrenamiento de las arquitecturas de redes neuronales planteadas se exploran con detalle en la 
    Sección(\ref{sec:design}) para el diseño de la red neuronal y en la Sección(\ref{sec:training}) para el entrenamiento. 

    \item \textit{Diseñar el subsistema de control y actuación para la conducción autónoma de un vehículo con características 
    similares a las de un vehículo doméstico real.}

    \textbf{Conclusión:} Se ha diseñado el Subsistema de Control y Actuación de acuerdo con la arquitectura planteada en la 
    Sección(\ref{sec:arquitectura}) en la Sección(\ref{sec:control}). Se ha considerado las características y requerimientos 
    planteados para dicho subsistema en 
    las tareas de control de tiempo real implementado en un microcontrolador, interfaz con los actuadores y sensores 
    sobre la plataforma de comunicación de ROS. Se han expuesto los detalles de implementación tanto a nivel de hardware 
    como de software procurando que el diseño de este subsistema sea modular.

    \item \textit{Diseñar el subsistema de inferencia y control autónomo basado en el uso de redes neuronales convolucionales.}
    
    \textbf{Conclusión:} Por su parte, se ha diseñado a detalle el Inferencia y Control autónomo en base a todos los fundamentos 
    teóricos de redes neuronales exploradas en la Sección(\ref{sec:sec:esqinferencia}). Los detalles de la implementación se pueden 
    encontrar en la Sección(\ref{sec:inferencia})
    donde se abunda en la implementación de los módulos que componen este subsistema para las tareas de predicción de dirección
    con la red neuronal convolucional, detección de obstáculos con un sensor de proximidad y el algoritmo del piloto automático 
    implementados como nodos de ROS para garantizar la modularidad del sistema.

    \item \textit{Analizar los resultados del entrenamiento e implementación del subsistema de inferencia y control autónomo.}
    
    \textbf{Conclusión:} El análisis de los resultados del entrenamiento de la red neuronal convolucional se ha explorado en 
    la Sección(\ref{sec:analisistrain}) pudiendo hacer la comparación de resultados en errores de entrenamiento, validación 
    y prueba de dos arquitecturas:
    una con una red neuronal tradicional o densamente conectada, y otra con una red neuronal convolucional. En base a los 
    resultados y puntajes en el entrenamiento obtenidos se puede concluir que una red neuronal convolucional es capaz de cumplir la 
    tarea de generar comandos de control para la tarea de conducción autónoma. 

    Por otro lado, la implementación de la red y los resultados en pruebas de campo se han explorado en la 
    Sección(\ref{sec:analisistest}) en la cual 
    se pudo observar la efectividad de la red neuronal convolucional para generar representaciones internas relevantes y útiles. 
    En base a este análisis se puede concluir que, efectivamente, una la red convolucional entrenada es capaz de generar 
    representaciones internas relevantes en muestras nunca antes vistas, en otras palabras, que la capacidad de generalización 
    es aceptable.

    \item \textit{Realizar pruebas de rendimiento y análisis comparativos en el sistema implementado.}
    
    \textbf{Conclusión:} Con el fin de realizar un análisis comparativo entre distintas implementaciones de redes neuronales 
    para la tarea especificada por este proyecto, se ha diseñado y entrenado un par de arquitecturas de red neuronal sobre las 
    cuales se realiza un análisis comparativo de rendimiento en base a indicadores y puntajes estándar en el campo de la estadística
    en la Sección(\ref{sec:scores}). 

    Por su parte, también se ha analizado las predicciones generadas por la red convolucional, para distintos casos, explorando 
    la naturaleza de las representaciones generadas por la misma en la Sección(\ref{sec:representaciones}). Por tanto, se puede 
    concluir que la implementación 
    de la red neuronal convolucional planteada se ha realizado exitosamente.

\end{itemize}

\section{Recomendaciones}

Luego de haber analizado y generado conclusiones relativas a los resultados obtenidos en base a los objetivos planteados en 
la etapa inicial del presente proyecto, se plantea una serie de recomendaciones:

\begin{itemize}
    \item Debido a la capacidad de generalización que puede lograrse con una red neurona convolucional se recomienda 
    generar distintos conjuntos de entrenamiento en condiciones climáticas diversas. Esto incrementará la complejidad 
    de la red convolucional y logrará que las predicciones de la red sean más robustas en cambios de iluminación 
    causados por distintos aspectos climáticos.
    \item Por su parte, gracias a la naturaleza modular del presente proyecto, se recomienda diseñar, entrenar y 
    validar arquitecturas de redes neuronales distintas o con variaciones a una red neuronal convolucional 
    tradicional. Se recomienda, por ejemplo, implementar una red convolucional recurrente que sea capaz de tomar 
    en cuenta el estado anterior de la misma. Implementar otros algoritmos de predicción y visión artificial también 
    ayudarían al desarrollo de sistemas más robustos.
    \item Se recomienda también extender la implementación de la red a una tarea de clasificación categórica para 
    comparar el rendimiento entre el entrenamiento y rendimiento obtenidos en el problema planteado como una regresión 
    y una clasificación.
    \item Debido ala modularidad del sistema planteando en el presente proyecto, se recomienda implementar el mismo 
    sistema fin a fin para otros modelos de vehículos, como pueden ser un robot de tracción diferencial u otro modelo.
    \item Es también importante poder extender el sistema presentado en este proyecto con tareas de control y detección 
    avanzadas, incluyendo en el flujo de trabajo módulos de planificación de trayectorias, detección de objetos y 
    programación de misiones.
    \item Dada la gran potencialidad en el área de sistemas de conducción autónoma se recomienda que se pueda crear 
    una línea de investigación dedicada al diseño e implementación de sistemas robóticos inteligentes en la cual 
    se agrupen esfuerzos para el desarrollo de cada uno de los subsistemas que componen un vehículo autónomo. 
    \item Por último, la forma en la que ha sido implementado el presente proyecto permite modificarlo y extenderlo 
    de distintas maneras y a distintos niveles. Se recomienda considerar este proyecto como una plataforma de 
    desarrollo sobre la cual se puedan implementar diversos algoritmos y aplicaciones útiles para el desarrollo de 
    nuestra sociedad.

\end{itemize}