\chapter*{Resumen} % si no queremos que añada la palabra "Capitulo"
\addcontentsline{toc}{chapter}{Resumen} % si queremos que aparezca en el índice
\markboth{RESUMEN}{RESUMEN} % encabezado

Los vehículos autónomos han pasado de ser un tema de ciencia ficción a convertirse en una 
realidad cada vez más cercana. Si bien existe un recorrido muy largo para llegar a 
implementar sistemas completamente autónomos en las calles, los recientes avances en la tecnología 
junto y creciente interés económico de grandes empresas, universidades y centros de investigación 
en el mundo han hecho posible la inclusión exitosa de diversos niveles de autonomía en vehículos, con fines de 
uso doméstico e industrial. El presente proyecto se centra en el desarrollo de un sistema 
de conducción autónoma basado en visión artificial, para la generación de comandos de control para la 
conducción autónoma de un vehículo doméstico. Se ha logrado desarrollar un sistema de aprendizaje 
“fin a fin” basado en una red neuronal convolucional, que consta de un modelo de predicción que genera 
comandos de control a partir de un estímulo visual proveniente de una cámara monocular. El sistema de 
aprendizaje está implementado sobre una plataforma de cómputo de bajo costo y bajo consumo de energía 
basado en un microcontrolador ARM Cortex M y una SBC Raspberry Pi encargados del control de bajo nivel 
en tiempo real; la adquisición de datos de entrenamiento, el entrenamiento de la red neuronal y un 
sistema de control e inferencia autónomo implementados en los lenguajes de programación Python y C++ 
usando ROS y Tensorflow. Finalmente, se ha validado el entrenamiento de la red neuronal convolucional 
en conjunto con todo el sistema de conducción autónoma en base a pruebas estadísticas de rendimiento y un 
análisis de representaciones internas de la red a estìmulos visuales de diversa naturaleza.
