\chapter*{Resumen} % si no queremos que añada la palabra "Capitulo"
\addcontentsline{toc}{chapter}{Resumen} % si queremos que aparezca en el índice
\markboth{RESUMEN}{RESUMEN} % encabezado

Los vehículos autónomos han pasado de ser un tema de ciencia ficción a convertirse una realidad cada vez más 
cercana. Si bien existe un recorrido muy largo para llegar a implementar sistemas completamente autónomos en las calles, 
los recientes avances en la tecnología junto con el interés económico de grandes empresas y corporaciones en el mundo 
ha hecho posible incluir diversos niveles de autonomía a vehículos con fines de uso doméstico e industrial con éxito.

Una de las áreas que más se ha nutrido de los recientes avances es el área de la visión artificial o visión por computador; 
resolviendo con facilidad tareas de una complejidad muy alta, como la detección y reconocimiento de objetos. 
Este crecimiento, en gran parte, se ha debido al desarrollo y optimización de las redes neuronales, las cuales se han 
constituido en una herramienta con muchas potencialidades y aplicaciones por la forma en la que se procesa la información 
y su capacidad para generalizar tareas complejas en base a una gran cantidad de datos de entrenamiento. En específico, 
las redes neuronales convolucionales han podido revivir al campo de la visión artificial gracias a la forma eficiente 
en la se que procesan imágenes o matricies multidimensionales y la capacidad de crear representaciones internas a partir 
de filtros de convolución en las capas ocultas de la red neuronal\cite{krizhevsky2012imagenet}.

La visión artificial juega un papel muy importante en el desarrollo de vehículos autónomos por cuanto permite 
procesar imágenes digitales provenientes de cámaras instaladas en los mismos vehículos y extraer información 
valiosa para la navegación y la conducción, como ser la detección de carril, peatones, signos de tránsito, otros vehículos, 
etc. Esta utilidad hace posible diversas oportunidades de investigación y desarrollo de algoritmos y sistemas de visión 
artificial orientados a conducción autónoma.

El presente proyecto se centra en el desarrollo de un sistema de conducción autónoma basado en visión artificial para 
la generación de comandos de control para la conducción autónoma de un vehículo doméstico con un modelo de locomoción 
de Ackerman. Se intenta desarrollar un sistema de aprendizaje “fin a fin” que consta de un modelo de predicción 
que genera comandos de control a partir de un estímulo visual proveniente de una cámara monocular.