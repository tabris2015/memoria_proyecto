\documentclass[oneside,12pt,letterpaper]{book}
\setcounter{tocdepth}{3}
\setcounter{secnumdepth}{3}

%%%%%%%%%%%%%%%% for dev
% \includeonly{Chapter_1,Chapter_4}
%%%%%%%%%%%%%%%%
%%%%% CAMBIAR LA PRIMERA LINEA POR LA SIGUIENTE PARA LA MEMORIA DE PROYECTO %%%%%
%\documentclass[12pt,letterpaper,oneside]{book}

% Paquetes basicos ...
\usepackage[spanish, mexico]{babel}
\usepackage[utf8]{inputenc} % OJO!!!  => MANTENER ESTA LINEA PARA FACIL CONVERSION A WORD EN EL FUTURO ...
\usepackage{float,xcolor}
\usepackage{graphicx} 
\usepackage{array}
\usepackage{tabularx}
\usepackage{amssymb, amsmath}

% Paquetes extras ... 
\usepackage{subfigure}
% \usepackage{color}
\usepackage{anysize} 
\usepackage{breakcites}
\usepackage{enumitem}
\usepackage{blindtext}
\usepackage{hyperref}
% figuras
\usepackage{wrapfig}
\usepackage{lscape}
\usepackage{rotating}
\usepackage{epstopdf}

\usepackage{booktabs}
\usepackage{multirow}
% \usepackage[table,xcdraw]{xcolor}
\usepackage{listings} % para codigo fuente

% para los algoritmos
\usepackage{algorithm}
\usepackage{algorithmic}

\floatname{algorithm}{Algoritmo}
\renewcommand{\listalgorithmname}{Lista de algoritmos}
\renewcommand{\algorithmicrequire}{\textbf{Entrada:}}
\renewcommand{\algorithmicensure}{\textbf{Salida:}}
\renewcommand{\algorithmicend}{\textbf{fin}}
\renewcommand{\algorithmicif}{\textbf{Si}}
\renewcommand{\algorithmicthen}{\textbf{entonces:}}
\renewcommand{\algorithmicelse}{\textbf{Si no}}
\renewcommand{\algorithmicelsif}{\algorithmicelse,\ \algorithmicif}
\renewcommand{\algorithmicendif}{\algorithmicend\ \algorithmicif}
\renewcommand{\algorithmicfor}{\textbf{Para}}
\renewcommand{\algorithmicforall}{\textbf{Para todo}}
\renewcommand{\algorithmicdo}{\textbf{hacer:}}
\renewcommand{\algorithmicendfor}{\algorithmicend\ \algorithmicfor}
\renewcommand{\algorithmicwhile}{\textbf{Mientras}}
\renewcommand{\algorithmicendwhile}{\algorithmicend\ \algorithmicwhile}
\renewcommand{\algorithmicloop}{\textbf{repetir:}}
\renewcommand{\algorithmicendloop}{\algorithmicend\ \algorithmicloop}
\renewcommand{\algorithmicrepeat}{\textbf{repetir:}}
\renewcommand{\algorithmicuntil}{\textbf{hasta que:}}
\renewcommand{\algorithmicprint}{\textbf{Imprimir}} 
\renewcommand{\algorithmicreturn}{\textbf{Retornar:}} 
\renewcommand{\algorithmictrue}{\textbf{verdadero }} 
\renewcommand{\algorithmicfalse}{\textbf{falso }} 
 % mi archivo de traducción

\usepackage{lipsum}
\usepackage{courier}

\usepackage{tikz}
\def\checkmark{\tikz\fill[scale=0.4](0,.35) -- (.25,0) -- (1,.7) -- (.25,.15) -- cycle;}

\lstset{basicstyle=\footnotesize\ttfamily, breaklines=true}
% %% para que aparezca Capitulo
% \usepackage{tocloft}

% \renewcommand{\cftchappresnum}{Capítulo }
% \renewcommand{\cftchapaftersnum}{:}
% \renewcommand{\cftchapnumwidth}{25mm}

% hola bola este es un bug

\begin{document}
\marginsize{2.5cm}{2cm}{2cm}{2cm} 

% Para que no aparezca la numeracion en el pie de pagina de todo el documento ...
%\pagestyle{empty}


%%%%%% ******  INICIO CARATULA ***** %%%%%%%%%
% Especificaciones de la caratula PPG

\begin{titlepage}
    \begin{center}
    \vspace*{-0.5in}
    \begin{large}
    \textbf{UNIVERSIDAD MAYOR DE SAN ANDRÉS}\\
    \vspace*{0.15in}
    \textbf{FACULTAD DE INGENIERÍA}\\
    \vspace*{0.15in}
    \textbf{CARRERA DE INGENIERÍA ELECTRÓNICA}\\
    \vspace*{0.1in}
    \end{large}
    % Logo UMSA
    \begin{figure}[htb]
    \begin{center}
    \includegraphics[width=8cm]{img/umsa.jpg}
    % \includegraphics{img/umsa.jpg}
    \end{center}
    \end{figure}
    \begin{Large}
    \textbf{PROYECTO DE GRADO} 
    \end{Large}
    \vspace*{0.4in}
    
    \begin{normalsize}
    \textbf{``Aprendizaje fin a fin para la conducción autónoma de vehículos domésticos usando visión artificial y redes neuronales convolucionales''} \\
    \end{normalsize}
    
    \vspace*{0.2in}
    
    \begin{large}
    \textbf{POSTULANTE:} JOSE EDUARDO LARUTA ESPEJO\\
    \end{large}
    
    \begin{large}
    \hspace{0.08in} \textbf{TUTOR:} JAVIER SANABRIA GARCIA\\
    \end{large}
    
    \begin{large}
    \hspace{0.44in} \textbf{D.A.M.:} GONZALO SAMUEL CABA MORALES\\
    \end{large}
    
    \vspace*{0.2in}
    
    \begin{normalsize}
    LA PAZ, AGOSTO 2018\\
    \end{normalsize}
    \end{center}
    \end{titlepage}
    
    
    \thispagestyle{empty}
\pagenumbering{Roman} % para comenzar la numeracion de paginas en numeros romanos
\chapter*{}
% \pagenumbering{Roman} % para comenzar la numeracion de paginas en numeros romanos
\begin{flushright}
\textit{Dedicado a mis padres Edwin y Lourdes, \\
pilares fundamentales de mi formación y principal inspiración \\
en mi búsqueda de superación profesional.}
\end{flushright}


\chapter*{Agradecimientos} % si no queremos que añada la palabra "Capitulo"
\addcontentsline{toc}{chapter}{Agradecimientos} % si queremos que aparezca en el índice
\markboth{AGRADECIMIENTOS}{AGRADECIMIENTOS} % encabezado 
 
Agradezco infinitamente a mis padres Edwin y Lourdes,
por su incansable e incondicional apoyo y paciencia en mi desarrollo personal y moral.

\vspace{1cm}

A mi asesor Javier Sanabria, por su valiosa y desinteresada guía, 
enseñanzas y consejos en el ámbito académico, ético y profesional 
en el desarrollo de este proyecto y a lo largo de toda la carrera.

\vspace{1cm}

A mis profesores, por haberme transmitido amablemente su experiencia y conocimiento 
en las distintas asignaturas en toda la carrera, garantizándome una formación ingenieril integral.

\vspace{1cm}

A mis compañeros y amigos con los que pude compartir experiencias y conocimientos, que han enriquecido 
mi desarrollo profesional dentro de la carrera y mi desarrollo personal fuera de la misma.

\chapter*{Resumen} % si no queremos que añada la palabra "Capitulo"
\addcontentsline{toc}{chapter}{Resumen} % si queremos que aparezca en el índice
\markboth{RESUMEN}{RESUMEN} % encabezado

Una bonita historia

% Generacion del indice
\include{formato/indices}

% Contenido del PPG
\chapter{Introducción} \label{ch:introduccion}
\pagenumbering{arabic} % para empezar la numeración con números
La gran intro

\chapter{Marco Teórico} \label{ch:m_teorico}

% estructura
\section{Sistemas de Conducción Autónoma}
Un sistema de conducción autónoma es una combinación de varios componentes o subsistemas donde las tareas 
de percepción, toma de decisiones y operación de un vehículo son desarrolladas por un sistema electrónico en lugar
de un conductor humano. Usualmente, un sistema de conducción autónoma incluye varios subsistemas de automatización 
que operan de manera conjunta y coordinada para poder tomar el control total o parcial del vehículo. 

En algunas ocasiones, la automía del control se implementa de manera condicional, es decir, que el sistema toma 
el control del vehículo para ciertas situaciones pero no todo el tiempo como por ejemplo sistemas de estabilización 
de frenos o prevención de impactos. Este tipo de sistemas se ha ido desarrollando e implementando en vehículos comerciales 
de manera paulatina pero todavía no existe un vehículo completamente autónomo circulando por las calles o carreteras. Notese
que los términos autonomía y automatización se usan de manera intercambiable en este contexto.

    \subsection{Niveles de Autonomía}
    Debido al creciente interés e inversión en el desarrolo de sistemas de conducción autónoma se ha establecido 
    una manera de categorizar los niveles de automatización de la conducción por parte de  la Sociedad de Ingenieros en Automoción
    (SAE, por sus siglas en inglés) en la que se definen seis niveles de automatización en vehículos terrestres, acuáticos y aéreos.

        \subsubsection{Nivel 0: Sin automatización}
        El conductor está en completo control de todas las funciones del vehículo en todo momento, no existe intervención 
        de ningún sistema automatizado en el control. Sistemas de alerta de colisión o pérdida de carril entran en esta categoría.
        \subsubsection{Nivel 1: Conducción asistida}
        El conductor tiene el control del vehículo, pero el sistema puede modificar la aceleración o dirección del mismo. Los 
        sistemas de control de velocidad de crucero caen en esta categoría.
        \subsubsection{Nivel 2: Automatización parcial}
        El conductor debe poder ser capaz de tomar el control del vehículo si ciertas se necesitan ciertas correcciones, pero  
        ya no está en control de la aceleración y dirección del vehículo directamente. Es importante resaltar que desde los
        niveles 0 al 2 el conductor no puede estar distraido en ningún momento de la conducción. Los sistemas de parqueo 
        automático representan un buen ejemplo de sistemas de Nivel 2.
        \subsubsection{Nivel 3: Automatización condicional}
        El sistema automatizado tiene el control del vehículo, tanto de la aceleración, dirección así como también del monitoreo 
        del entorno bajo condiciones específicas. El conductor debe estar preparado para intervenir cuando el sistema así lo 
        requiera, por tanto, se permiten distracciones ocasionales. Uno de los sistemas recientemente implementados que cae en esta 
        categoría es el sistema \textit{autopilot} de los vehículos de Tesla Motors. % agregar referencia
        \subsubsection{Nivel 4: Automatización elevada}
        El sistema está en completo control del vehículo y la presencia humana ya no es necesaria, sin embargo, la operación autónoma 
        del vehículo está limitada a condiciones específicas. Si las actuales condiciones del entorno sobrepasan las fronteras 
        de rendimiento definidas, el vehículo puede desplegar un protocolo o secuencia de emergencia. Actualmente el desarrollo 
        de vehículos autónomos o \textit{self driving cars} se enfoca en este nivel. 
        \subsubsection{Nivel 5: Automatización completa}
        El sistema está en completo control del vehículo y la presencia humana no es necesaria en absoluto. El sistema es capaz 
        de proveer las mismas características que en el Nivel 4, pero en esta ocasión puede operar al vehículo en todas las condiciones.
        En este nivel, el conductor pasa a ser un pasajero en el vehículo. Actualmente, no existen sistemas que operen en este nivel.

    La relación entre la responsabilidad del sistema y el conductor en los distintos niveles se puede apreciar en la 
    Tabla \ref{tbl:niveles}:
    
        \begin{table}[!h]
            \centering
            \resizebox{\textwidth}{!}{%
            \begin{tabular}{@{}|c|l|c|c|c|c|@{}}
            \toprule
            \textbf{Nivel SAE} & \textbf{Denominación}      & \textbf{\begin{tabular}[c]{@{}l@{}}Ejecución de aceleración \\ y dirección\end{tabular}} & \textbf{\begin{tabular}[c]{@{}l@{}}Monitoreo del \\ entorno\end{tabular}} & \textbf{\begin{tabular}[c]{@{}l@{}}Responsable en \\ condiciones difíciles\end{tabular}} & \textbf{\begin{tabular}[c]{@{}l@{}}Modos de \\ conducción\end{tabular}} \\ \midrule
            0                  & Sin Automatización         & Humano                                                                                   & \multirow{3}{*}{Humano}                                                   & \multirow{4}{*}{Humano}                                                                  & Ninguno                                                                 \\ \cmidrule(r){1-3} \cmidrule(l){6-6} 
            1                  & Conducción asistida        & Humano y sistema                                                                         &                                                                           &                                                                                          & \multirow{3}{*}{Algunos Modos}                                          \\ \cmidrule(r){1-3}
            2                  & Automatización parcial     & \multirow{4}{*}{Sistema}                                                                 &                                                                           &                                                                                          &                                                                         \\ \cmidrule(r){1-2} \cmidrule(lr){4-4}
            3                  & Automatización condicional &                                                                                          & \multirow{3}{*}{Sistema}                                                  &                                                                                          &                                                                         \\ \cmidrule(r){1-2} \cmidrule(l){5-6} 
            4                  & Automatización elevada     &                                                                                          &                                                                           & \multirow{2}{*}{Sistema}                                                                 & Varios Modos                                                            \\ \cmidrule(r){1-2} \cmidrule(l){6-6} 
            5                  & Automatización completa    &                                                                                          &                                                                           &                                                                                          & Todos los Modos                                                         \\ \bottomrule
            \end{tabular}%
            }
            \caption{Niveles de automatización según SAE. Fuente: SAE} % TODO: referencia
            \label{tbl:niveles}
            \end{table}

    \subsection{Arquitectura de un sistema de conducción autónoma}
    % \subsection{Aprendizaje Profundo}

% TODO: 
\section{Visión por computador}
-
    \subsection{Procesamiento de imágenes}
    \subsection{Filtrado}

\section{Redes Neuronales Artificiales}

    \subsection{Aprendizaje Automático}
    El aprendizaje automático es un subcampo de la inteligencia artificial que intenta extraer 
    patrones mediante un proceso de \textit{aprendizaje} a partir de datos \cite{Mitchell1990}. Este proceso 
    de aprendizaje se define de acuerdo a una \textbf{tarea específica} $T$ que intenta aprenderse en base 
    a \textbf{experiencia pasada} $E$ tomando como referencia una \textbf{medida de rendimiento} $P$ . 
    Dentro de esta definición, se puede listar varios ejemplos de tareas de aprendizaje que usualmente se resuelven  
    usando los conceptos del aprendizaje automático o también llamado \textit{machine learning}:
    \\
    \\
    \begin{itemize}
        \item \textbf{Un algoritmo de aprendizaje que pueda jugar ajedrez:}
        \begin{itemize}
            \item \textbf{Tarea $T$:} Jugar Ajedrez.
            \item \textbf{Medida de Rendimiento $P$:} Porcentaje de partidas ganadas contra el oponente.
            \item \textbf{Experiencia $E$:} Información de varias partidas de práctica.
        \end{itemize}
        
        \item \textbf{Un algoritmo de aprendizaje que pueda reconocer dígitos manuscritos:}
        \begin{itemize}
            \item \textbf{Tarea $T$:} Reconocer y clasificar dígitos manuscritos dentro de una imagen.
            \item \textbf{Medida de Rendimiento $P$:} Porcentaje de dígitos correctamente clasificados.
            \item \textbf{Experiencia $E$:} Base de datos de imágenes de dígitos con sus etiquetas correspondientes.
        \end{itemize}

        \item \textbf{Un algoritmo de aprendizaje que pueda reconocer la voz:}
        \begin{itemize}
            \item \textbf{Tarea $T$:} Extraer una secuencia de palabras de una grabación de voz.
            \item \textbf{Medida de Rendimiento $P$:} Porcentaje de palabras correctamente predichas.
            \item \textbf{Experiencia $E$:} Grabaciones de voz con una transcripción correspondiente.
        \end{itemize}
    
    \end{itemize}
    

    Esta definición de aprendizaje es lo suficientemente amplia como para englobar todas las tareas 
    que el campo del aprendizaje automático intenta resolver en la actualidad. Sin embargo, debido a 
    su naturaleza, se pueden clasificar las tareas de aprendizaje en tres grandes categorías que tienen
    características particulares: aprendizaje supervisado, aprendizaje no supervisado y aprendizaje por refuerzo.

    La diferencia entre estos tres tipos de problemas surge de la distinta naturaleza de la experiencia $E$ disponible
    para el entrenamiento. A continuación, se procede a detallar cada uno de ellos.

        \subsubsection{Aprendizaje supervisado} \label{sss:supervisado}
        En el caso de las tareas de aprendizaje supervisado, la experiencia constituye un conjunto de datos o \textit{dataset}
        que contiene ejemplos con \textit{características} y cada ejemplo está asociado con una \textit{etiqueta}. Por ejemplo, 
        un conjunto de datos de flores donde cada registro contiene datos de la flor (características) y la especie a la que pertenece (etiqueta). 
        Dentro de los algoritmos que atacan problemas de aprendizaje supervisado se pueden encontrar 2 grandes categorías.
            \paragraph{Clasificación}
            Las tareas de clasificación tienen como característica el hecho de que la etiqueta de cada ejemplo en el 
            conjunto de datos pertenece a una categoría o, en otras palabras, tiene una naturaleza discreta y finita. Por ejemplo, 
            en el caso de la clasificación de las flores mencionado anteriormente, la etiqueta solamente puede pertenecer a un
            conjunto finito de especies de flores y cada ejemplo pertenece a una de estas especies.
            \paragraph{Regresión}
            En las tareas de regresión, las etiquetas pertenecen a un conjunto de números reales o de naturaleza 
            contínua. En este caso, las etiquetas no se asocian con categorías sino más bien con otro tipo de variables. Un 
            ejemplo muy conocido es el de la tarea de la predicción del precio de una casa en base a sus características, el precio 
            de una casa no puede categorizarse porque representa un número que puede tener infinitos valores dentro de un rango definido.
        
        En las tareas del aprendizaje supervisado, se puede considerar cada ejemplo como una descripción de 
        una situación (características) en conjunto con una especificación (etiqueta), cada uno de los ejemplos 
        dentro el conjunto de datos son eventos independientes y se pueden analizar por separado. En este sentido
        la tarea del algoritmo es generalizar la respuesta para casos no presentes en el conjunto inicial de datos.

        \subsubsection{Aprendizaje no supervisado}
        En las tareas del aprendizaje no supervisado la experiencia contenida en el conjunto de datos tiene la característica de 
        no poseer ninguna etiqueta, por tanto, usualmente se intenta buscar una estructura escondida dentro el conjunto de datos 
        o, dicho de otra manera, se buscan patrones que puedan presentarse en dichos datos. Estos patrones pueden aprovecharse 
        para extraer información relevante de la naturaleza de datos de muy alta dimensionalidad, información que normalmente no 
        es trivial de encontrar o visualizar por una persona. Entre algunas de las tareas más comunes dentro del aprendizaje no 
        supervisado, se pueden listar:

            \paragraph{Clustering}
            Refiere a la tarea de separar y agrupar los datos en un número finito de conjuntos o \textit{clusters}. Los 
            \textit{clusters} normalmente denotan una estructura oculta dentro de los datos y proporcionan información acerca 
            de la similaridad entre ejemplos del conjunto de datos.

            \paragraph{Reducción de dimensionalidad}
            Uno de los problemas con las bases de datos y conjuntos de datos disponibles es que poseen una dimensionalidad 
            bastante alta haciendo prácticamente imposible para un humano poder visualizar o encontrar patrones e información 
            útil en los mismos. Este problema se suele tratar con algoritmos de reducción de dimensionalidad, en la que 
            se encuentra una representación estimada de los datos pero con menos dimensiones. Uno de los algoritmos más 
            conocidos y usados en esta categoría es el análisis de componente principal o PCA, por sus siglas en inglés, en el 
            que se encuentra una representación de los datos en una menor dimensión usando proyecciones ortogonales.

            \paragraph{Estimación de probabilidad}
            Muchos conjuntos de datos son obtenidos de distintas fuentes y a lo largo de varios intervalos de tiempo, en este 
            entendido, es muy útil conocer o aproximar la distribución de probabilidad de los datos para luego poder realizar 
            predicciones o tratarlos con algún modelo en específico.

        \subsubsection{Aprendizaje por refuerzo}
        En las tareas de aprendizaje por refuerzo se toma en cuenta la interacción de un agente con su entorno y la forma 
        en la que las acciones que toma dicho agente afectan a su entorno y se materializan en una recompensa o castigo \cite{sutton2018reinforcement}. Formalmente
        se pueden definir ciertos elementos que componen una tarea de aprendizaje por refuerzo:
        \begin{itemize}
            \item \textbf{Agente.} Es la entidad que interactúa con el entorno. El agente se comunica con el entorno mediante acciones.
            \item \textbf{Política.} Representan la forma de actuar del agente en base al conocimiento que ha adquirido.
            \item \textbf{Recompensa.} Es la función que define la efectividad del agente de cumplir el objetivo deseado, normalmente, el aprendizaje se enfoca en maximizar la recompensa que el agente puede obtener. 
        \end{itemize}
        
        % TODO: insertar grafico de aprendizaje por refuerzo

    % TODO: 
    \subsection{Aprendizaje Profundo}
    Dentro del campo de la inteligencia artificial y el aprendizaje automático se han implementado diversos tipos 
    de algoritmos con éxito en los tipos de tareas de aprendizaje mencionados anteriormente. La base teórica y los detalles 
    de implementación de estos algoritmos son muy variados, sin embargo, las redes neuronales artificiales han experimentado 
    un incremento en el interés en la investigación y en las aplicaciones muy importante. Tal es el éxito de las mismas 
    que se ha creado un subcampo exclusivo llamado aprendizaje profundo o \textit{deep learning}. El aprendizaje profundo 
    es un campo de la inteligencia artificial que se encarga de estudiar exclusivamente a las redes neuronales artificiales, 
    sus componentes, arquitectura y aplicaciones. El impresionante rendimiento de estos algoritmos reside principalmente en el 
    concepto de la representación que generan a partir de los datos que se procesan. 

    El aprendizaje profundo resuelve el problema del aprendizaje de representaciones al introducir representaciones que se 
    expresan en términos de otras representaciones más simples. Además, permite a una computadora construir conceptos complejos 
    a partir de conceptos más simples. Un ejemplo de la generación de estos conceptos o representaciones se puede apreciar en la 
    Figura(\ref{fig:representacion}).
    
    
    \begin{figure}[!h] 
        \centering
        \includegraphics[width=0.75\textwidth]{img/representacion}
        \caption{Ilustración de un modelo de aprendizaje profundo. Las representaciones se generan en las capas ocultas y corresponden con características de distintos niveles de complejidad. Fuente: \cite{Goodfellow-et-al-2016} }
        \label{fig:representacion}
    \end{figure}

    A continuación, se procede a definir los conceptos más importantes de redes neuronales artificiales con los cuales se podrá 
    plantear la solución al problema de la conducción autónoma usando visión artificial.

        \subsubsection{Redes neuronales feedforward}
        Las redes neuronales feedforward o también llamadas perceptrón multicapa, son la base fundamental de los modelos 
        de aprendizaje profundo. El objetivo de una red neuronal feedforward es el de aproximar una función $f^\ast$. Por 
        ejemplo, para una tarea de clasificación, $y = f^\ast (\mathbf{x})$ mapea una entrada $\mathbf{x}$ a una categoría $y$.
        Una red neuronal feedforward define un mapeo $\mathbf{y} = f(\mathbf{x},\mathbf{W})$ y aprende el valor de los parámetros 
        $\mathbf{W}$ que resulten en la mejor aproximación\cite{Goodfellow-et-al-2016}.

        Este tipo de modelos son denominados feedforward debido a que la información fluye a por la función siendo evaluada 
        desde $\mathbf{x}$, a través de distintos cálculos intermedios definidos por $f$, hasta llegar a la salida $\mathbf{y}$.
        No existen conexiones de retroalimentación en las que la salidas del modelo se inyecten de nuevo a sí mismo. Las redes 
        neuronales que poseen este tipo de conexiones de retroalimentación son denominadas redes neuronales recurrentes.

        Para definir una red neuronal feedforward se puede comenzar definiendo un modelo basado en una combinación 
        lineal en conjunto con una función no lineal que toma la siguiente forma:

        \begin{equation}\label{eq:modelobase}
            y(\mathbf{x}, \textbf{W}) = f\left(\sum_{j=1}^M w_j  x_j\right)
        \end{equation}

        donde $f()$ es una función de activación no lineal. Esto lleva al modelo básico de una red neuronal que puede ser 
        descrita como una serie de transformaciones. Primero, se construyen $M$ combinaciones lineales de las variables 
        de entrada $x_1, \ldots , x_D$ donde $D$ es la dimensión del vector de entrada $\mathbf{x}$:
        
        \begin{equation}
            a_j = \sum_{i=1}^D w_{ji}^{(1)} x_i + w_{j0}^{(1)}
        \end{equation}

        donde $j = 1, \ldots , M$, y el superíndice $(1)$ indican que los correspondientes parámetros se encuentran en la 
        primera capa de la red. Los parámetros $w_{ji}^{(1)}$ se suelen conocer también con el nombre de \textit{pesos} y 
        los parámetros $ w_{j0}^{(1)}$ con el nombre de \textit{sesgos} o \textit{biases}. Las cantidades $a_j$ se conocen 
        como \textit{activaciones}, y cada una de ellas es luego transformada usando una función no lineal y derivable conocida 
        como la \textit{función de activación} $h()$ para luego obtener:

        \begin{equation}
            z_j = h(a_j)
        \end{equation}

        Estas cantidades corresponden con la salida de la capa y también se suelen referir por el nombre de  \textit{unidades ocultas}.
        Las funciones no lineales $h()$ pueden escogerse dependiendo a diversos criterios de rendimiento o de comportamiento. Siguiendo
        a la Ecuación(\ref{eq:modelobase}), las unidades ocultas se pueden volver a procesar con una combinación lineal y función 
        de activación en una segunda capa:

        \begin{equation}
            a_k = \sum_{j=1}^M w_{kj}^{(2)} z_j + w_{k0}^{(2)}
        \end{equation}

        donde $k = 1, \ldots, K$ y $K$ corresponden con el número de salidas de la segunda capa. Finalmente, si consideramos a 
        esta capa como la capa de salida, podemos transformar las activaciones de la segunda capa con una función de activación. 
        Normalmente, para una tarea de regresión, la función de activación es una función identidad, es decir $y_k = a_k$. Para 
        una tarea de clasificación binaria, en cambio, la función de activación es una función sigmoide:

        \begin{equation}
            y_k = \sigma(a_k)
        \end{equation}

        donde
        
        \begin{equation}
            \sigma(a) = \frac{1}{1 + e^{-a}}    
        \end{equation}
        
        Finalmente, se pueden combinar las etapas en una función general de la red que, para una salida sigmoidal, toma la 
        siguiente forma:

        \begin{equation}\label{eq:reddoscapas}
            y_k(\mathbf{x}, \mathbf{W}) = \sigma \left( \sum_{j=1}^M w_{kj}^{(2)} h \left( \sum_{i=1}^D w_{ji}^{(1)} x_i + w_{j0}^{(1)} \right) + w_{k0}^{(2)} \right)
        \end{equation}

        De esta manera, se define una red neuronal de dos capas a partir de la combinación lineal de las entradas y las unidades 
        ocultas con los parámetros o pesos de la red transformados por funciones de activación no lineal. La arquitectura de la 
        red definida en la Ecuación(\ref{eq:reddoscapas}) se puede visualizar en la Figura(\ref{fig:reddoscapas}) donde se observa claramente las relaciones que se 
        han definido anteriormente en forma gráfica y la naturaleza del flujo en una sola dirección (feerforward) de los datos 
        desde la entrada hasta la salida. En este caso, la red neuronal analizada es una red neuronal con una capa oculta.

        \begin{figure}[!h] 
            \centering
            \includegraphics[width=0.75\textwidth]{img/reddoscapas}
            \caption{Diagrama de la red neuronal de dos capas correspondiente a la Ecuación(\ref{eq:reddoscapas}). Fuente: \cite{Bishop2006} }
            \label{fig:reddoscapas}
        \end{figure}

        \subsubsection{Funciones de activación}
        \subsubsection{Funcion de costo}
        \subsubsection{Gradientes y retropropagación}
        \subsubsection{Diseño de Arquitecturas}

    \subsection{Redes Neuronales Convolucionales}
    Las redes neuronales convolucionales son un tipo especializado de red neuronal que 
    sirven para procesar datos de tipo "grilla" \cite{Goodfellow-et-al-2016}. Algunos ejemplos 
    de datos de tipo grilla que se pueden mencionar son los siguientes:
    \begin{itemize}
        \item \textbf{Series de tiempo.} Grilla de una dimensión tomados en intervalos regulares de tiempo.
        \item \textbf{Imágenes digitales.} Grilla de pixeles de dos o más dimensiones (Escala de grises, RGB).
    \end{itemize}

    Las también llamadas redes convolucionales, han demostrado un éxito impresionante en diversas 
    aplicaciones prácticas especialmente en el campo de la visión por computador y el procesamiento de texto y lenguaje natural. 
    El término ``red neuronal convolucional'' proviene del hecho de que en este tipo 
    de redes neuronales se utiliza una operación matemática llamada \textbf{convolución}, siendo la convolución 
    una operación lineal especializada para procesar datos de tipo grilla.

    En los párrafos posteriores, se procede a describir la operación de convolución en el contexto de 
    redes neuronales, pues, no siempre la definición de la misma corresponde con el concepto de convolución
    usado en distintos campos de la ciencia y la ingeniería.

        \subsection{Operación de convolución}
        En su forma más general, la convolución es una operación entre dos funciones reales y su definición se puede introducir
        usando el concepto de un promedio ponderado. Sea una función $x(t)$ dependiente del tiempo, 
        tanto $x$ como $t$ son números reales; en este caso, la función $x$ puede entenderse como una serie de medidas
        en un instante de tiempo $t$. Considérese una segunda función de ponderación $w(\tau)$ donde $\tau$ es la antiguedad 
        de una medida. Si se aplica la función de ponderación en cada instante de tiempo, se puede obtener una nueva función 
        definida por:
        \begin{equation}
            s(t) = \int x(\tau)w(t - \tau) d\tau
        \end{equation} 
        Esta operación es llamada la \textit{operación de convolución} y es denotada tradicionalmente con un asterisco:
        \begin{equation}
            s(t) = (x\ast w)(t)
        \end{equation}
        En el ejemplo de la ponderación, $w$ debe ser una función de densidad de probabilidad válida, o la salida no podrá
        ser considerada como un promedio ponderado. Además, $w$ también debe ser $0$ para cualquier $t<0$, esta última 
        característica se denomina comunmente como el principio de ``causalidad''. En general, la convolución está 
        definida para cualquier función en la cual la integral anteriormente declarada esté definida y puede ser 
        usada para otros propósitos aparte de promedios ponderados.

        Hablando en términos de una red neuronal convolucional, el primer argumento (en el ejemplo, la función $x$) 
        es comunmente referido como la \textbf{entrada}, y el segundo argumento ($w$, en el ejemplo) es referido 
        como el \textbf{kernel}. La salida, a su vez, es normalmente referida como el \textbf{mapa de características}.
        
        Por su parte, cuando se trata de señales digitales, como los datos en una computadora, el tiempo tiene una 
        naturaleza discreta, es decir, que los datos estarán disponibles en intervalos regulares de tiempo. En este 
        caso, el índice de tiempo $t$ puede tomar solamente valores enteros y, entonces, es válido asumir 
        que tanto $x$ como $w$ estan definidos solamente para valores enteros de $t$. De este modo, 
        se puede definir la convolución discreta:
        \begin{equation}
            s(t) = (x \ast w)(t) = \sum_{\tau=-\infty}^{\infty}x(\tau)w(t-\tau)
        \end{equation}

        En el contexto de las aplicaciones de aprendizaje automático o, más específicamente, aprendizaje profundo,
        la entrada es usualmente un arreglo multidimensional de datos, y el kernel es usualmente un arreglo 
        multidimensional de parámetros que se adaptan en el proceso de aprendizaje. 

        \subsubsection{Procesamiento de imágenes con redes neuronales convolucionales}

        % TODO: poner un grafico de la convolucion en 2d de una imagen 


        La operación de convolución se usa frecuentemente sobre datos con más de una dimensión. Las imágenes digitales 
        son un perfecto ejemplo de un arreglo multidimensional de datos. Una imagen digital se representa mediante una
        matriz con filas y columnas, donde cada elemento se denomina pixel y contiene información acerca de la intensidad
        o luminancia, para una imagen en escala de grises o el nivel de color para distintos canales en una imagen a color.
        Si se toma el ejemplo de la imagen en escala de grises, se tiene una entrada o imagen bidimensional $I$ con un
        kernel bidimensional correspondiente $K$:

        \begin{equation} \label{eq:conv2d}
            S(i,j)=(I\ast K)(i,j) = \sum_{m} \sum_{n} I(m,n)K(i-m,j-n)
        \end{equation}

        Dado que la convolución es conmutativa, se puede reescribir la ecuación \ref{eq:conv2d} como:

        \begin{equation}
            S(i,j)=(K\ast I)(i,j) = \sum_{m} \sum_{n} I(i-m,j-n)K(m,n)
        \end{equation}

        Frecuentemente, la última fórmula es la más utilizada en librerías de aprendizaje profundo 
        por su sencillez en la implementación en un sistema computacional, esto, dado que existe menos 
        variación en el rango de valores válidos de $m$ y $n$.

        \subsubsection{Aprendizaje de representaciones internas}
        Una de las preguntas clave en la visión por computador es el cómo generar una buena y significativa
        representación interna de una imagen, dado que la mayor parte de la imagen corresponde con pixeles que no 
        aportan mucha información relevante a la tarea asignada. Por ejemplo, si se quisiera detectar 
        un rostro dentro de una imagen, normalmente se suele encontrar una representación que ayude a aislar solamente 
        las porciones de la imagen que pueden contener el rostro, tales como la búsqueda de contornos, bordes y 
        características típicas de un rostro. Antes de la aparición de las redes convolucionales, estas representaciones 
        se hallaban de manera manual y gracias al conocimiento de expertos en el área del procesamiento de imágenes. 
        La definición de características y mapas de características era comunmente conocida como la 
        \textit{ingeniería de características}, en la cual los expertos creaban descriptores para tareas específicas con 
        una gran inversión de tiempo en la sintonización fina de los mismos. 

        % TODO: poner el ejemplo de viola jones 

        En contraste con el anterior enfoque, las redes convolucionales generan sus propias representaciones internas
        de manera automática gracias al aprendizaje de los parámetros de cada uno de los kernels que componen las distintas 
        capas de la red neuronal. En principio, las redes convolucionales se inspiraron en el trabajo de Hubel y Wiesel 
        sobre la corteza visual primaria de un gato\cite{lecun2010convolutional}. En dicho trabajo, se logró identificar células simples que respondían
        de manera sobresaliente a distintas orientaciones con campos receptivos locales. Éstas células receptivas simples 
        se pueden corresponder con los kernels de convolución usados en las redes convolucionales por la sencillez y la 
        localidad de su campo de receptividad.

        Posteriormente, las redes convolucionales ganaron una gran popularidad debido a su rendimiento en tareas de 
        clasificación de imágenes y detección y reconocimiento de objetos en imágenes. El primer hito de su capacidad 
        para procesar imágenes de manera efectiva fue en concurso de clasificación de imágenes de ImageNet, donde 
        el equipo de Geoffrey Hinton logró sobrepasar el mejor resultado en precisión de clasificación por un gran márgen 
        usando una arquitectura de red convolucional \cite{krizhevsky2012imagenet}. En este trabajo, se pudo apreciar con 
        gran detalle las ventajas del enfoque del aprendizaje de representaciones internas en una red convolucional.

        \begin{figure}[!h] 
            \centering
            \includegraphics[width=0.75\textwidth]{img/fmap_imagenet}
            \caption{Kernels convolucionales de tamaño $11 \times 11 \times 3$ en la primera capa convolucional. Fuente: \cite{krizhevsky2012imagenet} }
            \label{fig:fmap_imagenet}
        \end{figure}
            
        Tal como se puede apreciar en la Figura(\ref{fig:fmap_imagenet}), en la primera capa convolucional, 
        los kernels de convolución corresponden con representaciones básicas en una imagen como la búsqueda de 
        bordes en distintas orientaciones, esto va acorde a lo establecido anteriormente en el modelo de 
        la corteza visual de un gato. Puede decirse entonces que las redes convolucionales emulan, en cierto modo, 
        al proceso biológico de visión en animales.

    % TODO: 
    \subsection{Sistemas de Aprendizaje Fin a Fin}

\section{Modelo cinemático del vehículo}
-
    \subsection{Ecuaciones de movimiento}




\chapter{Ingeniería del proyecto}\label{ch:ingenieria}
\section{Arquitectura del sistema}
En la presente sección se procede a detallar la arquitectura del sistema de forma general analizando cada uno de 
los subsistemas, sus funcionalidades, componentes y características. Posteriormente se detallarán los detalles técnicos
y de implementación de cada subsistema. Para comenzar, es necesario describir la visión general del sistema, la finalidad 
y alcance del mismo.

    \subsection{Visión general}
    Tal como se ha establecido en el Capítulo(\ref{ch:introduccion}), el objetivo del presente proyecto es el de diseñar 
    un sistema de aprendizaje fin a fin para la tarea de conducción autónoma en vehículos domésticos. Este sistema ha sido 
    diseñado con la finalidad de plantear una alternativa para el desarrollo de sistemas de conducción autónoma en especial 
    en el subsistema de inferencia y control autónomo. No obstante, se ha desarrollado un prototipo completamente funcional 
    de un vehículo autónomo que cumple la tarea de seguir una carretera y detenerse cuando un obstáculo se interpone de 
    manera autónoma. 

    Es importante destacar que este proyecto también brinda un conjunto de herramientas de software 
    y hardware de manera que se pueda replicar el mismo de forma fácil y con un presupuesto reducido. Si bien el 
    presente proyecto se centra en el desarrollo 
    del sistema de visión artificial para la generación de comandos de dirección usando una red neuronal convolucional, la 
    naturaleza modular de la arquitectura del mismo permite realizar cambios o mejoras en cada uno de los subsistemas. Estos 
    cambios y mejoras se pueden introducir aprovechando la naturaleza modular de los nodos de ROS y la infraestructura de 
    comunicación presente en el sistema pudiendo agregarse más de un sistema de control en el mismo, como por ejemplo, un 
    sistema de reconocimiento de peatones o señales de tránsito.

    \subsection{Esquema del sistema}

    Se procede a detallar el esquema general del sistema en base a la interacción de tres subsistemas básicos. 

    \subsection{Subsistema de control y actuación} \label{sec:esqcontrol}
    El subsistema de control y actuación tiene el objetivo de servir como base física para la implementación de los algoritmos 
    de control. En la Figura() se puede apreciar sus componentes y la forma en que interactúan entre sí. El subsistema 
    interactúa tanto con el subsistema de adquisición de datos y entrenamiento como con el subsistema de inferencia y control 
    autónomo. 
    
    \subsection{Subsistema de adquisición de datos y entrenamiento}

    \subsection{Subsistema de inferencia y control autónomo}
    

\section{Herramientas de software}
    \subsection{Robot Operating System - ROS}
    ROS o Sistema Operativo Robótico es un \textit{framework} flexible para desarrollar software para robots. Se compone 
    de una colección de herramientas, librerías y convenciones que tienen el objetivo de simplificar la tarea de crear 
    comportamientos complejos y robustos en plataformas de robótica en general \cite{ros}.

    ROS ha sido construido con el objetivo de hacer accesible el desarrollo de sistemas robóticos mediante el trabajo 
    colaborativo de paquetes y utilidades, su naturaleza modular hace posible que se puedan implementar sistemas pieza 
    por pieza de acuerdo a las necesidades específicas de cada proyecto. Dentro de las facilidades que ROS ofrece, podemos 
    encontrar diversas utilidades que permiten el desarrollo de sistemas con una complejidad elevada.

        \subsubsection{Infraestructura de comunicación}
        En su núcleo, ROS ofrece una interfaz de intercambio de mensajes que provee comunicación inter-procesos y es 
        comunmente referida como el \textit{middleware}. El \textit{middleware} de ROS ofrece las siguientes facilidades:

        \begin{itemize}
            \item Intercambio de mensajes mediante publicación/subscripción y tópicos.
            \item Registro y reproducción de mensajes.
            \item Llamadas a procedimientos del tipo request/response.
            \item Sistema de administración distribuido de parámetros.
        \end{itemize}

        \begin{figure}[!h] 
            \centering
            \includegraphics[width=0.75\textwidth]{img/rqtgraph}
            \caption[Diagrama de comunicación de nodos]{Diagrama de comunicación de nodos usando mensajes. Fuente: \cite{roswiki} }
            \label{fig:rqtgraph}
        \end{figure}

        La naturaleza distribuida de ROS y las facilidades que ofrece el \textit{middleware}, hacen que el desarrollo de sistemas 
        robóticos modulares sea una tarea trivial. Aparte de la infraestructura de comunicación, ROS ofrece otras características
        especialmente diseñadas para el desarrollo de robots.
        
        \subsubsection{Características específicas para robótica}
        Adicionalmente a los componentes del \textit{middleware}, ROS tiene a disposición librerías y herramientas específicas 
        para el desarrollo rápido de sistemas robóticos. Algunas de las características más importantes se listan a continuación:

        \begin{itemize}
            \item Definiciones de mensajes estándar para robots.
            \item Lenguaje de descripción de robots URDF\footnote{URDF: Universal Robot Description Format}.
            \item Herramientas de diagnóstico.
            \item Localización.
            \item Mapeo.
            \item Navegación.
            \item Drivers de sensores y actuadores.
        \end{itemize}

        \subsubsection{Herramientas adicionales}
        Una de las características más atractivas de ROS es el conjunto de herramientas para desarrollo. Estas herramientas 
        soportan análisis, depuración y visualización del estado del sistema que esta siendo desarrollado. Los mecanismos presentes
        de publicación y subscripción permiten analizar de manera espontánea el flujo de datos en el sistema. Las herramientas 
        de ROS aprovechan esta característica y se presentan como una colección de herramientas gráficas y de línea de comandos que 
        simplifican el desarrollo y depuración de robots.

        \begin{figure}[!h] 
            \centering
            \includegraphics[width=0.75\textwidth]{img/rviz}
            \caption[Interfaz de visualización de ROS rviz]{Interfaz de visualización de ROS rviz. Fuente: \cite{roswiki} }
            \label{fig:rviz}
        \end{figure}

        \begin{itemize}
            \item \textbf{Herramientas de Línea de Comandos.} Permiten el control y depuración de los sistemas 
            de manera remota en una interfaz de línea de comandos. Existen comandos disponibles para ejecutar procesos, 
            analizar tópicos y mensajes, grabar y reproducir sesiones de mensajes y ejecutar servicios.

            \item \textbf{Rviz.} Es una interfaz de visualización de diversas fuentes de datos y modelos de robots. 
            Con la herramienta rviz es posible visualizar diversos tipos de mensajes provenientes de sensores tales 
            como cámaras o sensores láser. También es posible agrupar los distintos tipos de visualizaciones de manera 
            jerárquica en la misma ventana.

            \item \textbf{Rqt.} Rqt es un \textit{framework} para el desarrollo de interfaces gráficas para robots. 
            Con rqt es posible crear interfaces de control o monitoreo de manera gráfica y personalizada usando 
            componentes llamados plugins.

        \end{itemize}


        \subsubsection{Criterios de selección}
        En el marco del presente proyecto y el tiempo establecido para su desarrollo se ha basado la selección del entorno 
        de trabajo en base a los siguientes criterios:
        \begin{itemize}
            \item \textbf{Interfaz de comunicación distribuida.} Es necesario que se puedan desarrollar componentes del sistema 
            de manera independiente y puedan ser ejecutados de la misma manera. ROS ofrece mediante el desarrollo de 
            paquetes y nodos la facilidad de poder ejecutar y comunicar procesos de manera sencilla y distribuida a través
             del intercambio de mensajes.
            \item \textbf{Implementación de funcionalidades comunes.} También se necesita una plataforma con funcionalidades básicas 
            implementadas y disponibles para su uso, esto con el fin de concentrar el tiempo de desarrollo en las funcionalidades del 
            sistema en su conjunto más que en la plataforma sobre la cual se va a desplegar. Se necesitan herramientas reutilizables 
            para evitar lo que comúnmente se denomina como \textit{reinventar la rueda}.
            \item \textbf{Uso libre y código abierto.} ROS es una plataforma de código abierto, lo que permite utilizarlo de manera 
            libre ya sea para proyectos académicos y comerciales. Además, su naturaleza open source permite también realizar cambios 
            o mejoras en su funcionalidad de manera sencilla. El uso libre es importante dado que en entornos académicos normalmente 
            no se cuenta con la facilidad de adquirir licencias de software privativo. El uso libre también permite el desarrollo por 
            parte de investigadores independientes y estudiantes que no pertenecen a alguna institución que pueda apoyarlos financieramente.
            \item \textbf{Facilidad de uso.} El entorno de trabajo debe tener la facilidad de ser accesible para personas con un 
            conocimiento previo en electrónica y programación. Tanto los lenguajes de programación como las herramientas de desarrollo, 
            compilación y despliegue tienen que estar disponibles y ser fáciles de utilizar.
            \item \textbf{Compatibilidad con herramientas externas.} En el marco del proyecto y la aplicación de los conceptos de 
            visión artificial y aprendizaje profundo. El entorno de trabajo debe ser compatible o poder extender sus funcionalidades 
            con otros entornos dedicados al procesamiento de imágenes y visión artificial como a entornos y librerías 
            para el desarrollo y entrenamiento de redes neuronales. 
            \item \textbf{Interfaces con sistemas de bajo nivel y tiempo real.} Es necesario que la plataforma también 
            sea compatible con el desarrollo de sistemas embebidos y de tiempo real para el control de actuadores y 
            sensores que no se pueden conectar a una PC directamente.
        \end{itemize}

        Es en este sentido que se ha escogido usar al \textit{framework} ROS como plataforma de desarrollo para los distintos módulos 
        del sistema. Cabe resaltar que ROS no es la única plataforma para desarrollar robots, y algunas alternativas se detallan en la 
        Tabla(\ref{tbl:frameworks}) donde se puede analizar las características de cada una. 

      
        % Please add the following required packages to your document preamble:
        % \usepackage{booktabs}
        \begin{table}[!h]
            \begin{tabular}{@{}|c|c|c|c|c|c|@{}}
            \toprule
            \textbf{Nombre}                                             & \textbf{\begin{tabular}[c]{@{}c@{}}Interfaz de \\ Comunicación\\ Distribuida\end{tabular}} & \textbf{\begin{tabular}[c]{@{}c@{}}Sistema de \\ compilación\end{tabular}} & \textbf{\begin{tabular}[c]{@{}c@{}}Gestión de \\ paquetes\end{tabular}} & \textbf{\begin{tabular}[c]{@{}c@{}}Drivers de \\ bajo nivel\end{tabular}} & \textbf{\begin{tabular}[c]{@{}c@{}}Lenguajes de \\ programación\end{tabular}} \\ \midrule
            ROS                                                         & SI                                                                                         & SI                                                                         & SI                                                                      & SI                                                                        & \begin{tabular}[c]{@{}c@{}}C++\\ Python\\ Java\end{tabular}                   \\ \midrule
            YARP                                                        & SI                                                                                         & NO                                                                         & NO                                                                      & SI                                                                        & C++                                                                           \\ \midrule
            ROCK                                                        & SI                                                                                         & SI                                                                         & NO                                                                      & NO                                                                        & C++                                                                           \\ \midrule
            MRTP                                                        & NO                                                                                         & NO                                                                         & NO                                                                      & SI                                                                        & C++                                                                           \\ \midrule
            Player                                                      & SI                                                                                         & SI                                                                         & NO                                                                      & NO                                                                        & C++                                                                           \\ \midrule
            \begin{tabular}[c]{@{}c@{}}Robotics \\ Library\end{tabular} & NO                                                                                         & NO                                                                         & NO                                                                      & SI                                                                        & C++                                                                           \\ \bottomrule
            \end{tabular}
            \caption{Tabla comparativa de características entre distintas plataformas y librerías para desarrollo de sistemas robóticos. Fuente: Elaboración propia} % TODO: referencia
            \label{tbl:frameworks}
            \end{table}

        ROS se usa de manera extensiva en el desarrollo del presente proyecto para las siguientes tareas:

        \begin{itemize}
            \item En el subsistema de control y actuación como una interfaz común de intercambio de mensajes para el control de los motores presentes en el prototipo, así como también en la recuperación de los datos de los sensores. Estas interfaces están implementadas como nodos de ROS.
            \item En el subsistema de adquisición de datos y entrenamiento como una herramienta de captura de información del control manual y la cámara, tomando en cuenta las estampas de tiempo y sincronización para cada mensaje de ROS.
            \item En el subsistema de inferencia y control autónomo como la plataforma sobre la cual se definen los distintos controladores como nodos de ROS y el programa del piloto automático como un árbitro entre los mensajes de los distintos controladores. 
            \item En todo el sistema como la interfaz de comunicación distribuida a través del intercambio de mensajes entre el prototipo y la estación de trabajo remota.
        \end{itemize}

    \subsection{Tensorflow}
    Tensorflow es una librería para cálculos numéricos que funciona en base a grafos de flujo de datos Figura(\ref{fig:grafotf}). Las operaciones matemáticas 
    se representan como nodos en el grafo y los vértices representan matrices de datos multidimensionales o tensores que fluyen de 
    un nodo a otro  \cite{tensorflow2015-whitepaper}. Debido a esta implementación, los grafos pueden ejecutarse de manera distribuida en varias CPU o GPU. Las operaciones 
    matemáticas están disponibles para utilizar en la librería y sus implementaciones estan altamente optimizadas, lo que permite 
    aprovechar al máximo el hardware disponible.

    \begin{figure}[!h] 
        \centering
        \includegraphics[width=0.55\textwidth]{img/grafotf}
        \caption[Ejemplo de un grafo de cómputo utilizado en Tensorflow]{Ejemplo de un grafo de cómputo utilizado en Tensorflow. Fuente: \cite{asjad_2016} }
        \label{fig:grafotf}
    \end{figure}

    Tensorflow se ha hecho popular por la facilidad con la que se puede implementar la arquitectura de una red neuronal usando grafos
    de cómputo y por la optimización de los algoritmos usados. Actualmente, Tensorflow representa el estándar en la implementación de 
    redes neuronales profundas tanto en la academia como la industria. 

    Otra de las características de Tensorflow es que presenta una API en el lenguaje de programación Python, lo que permite el desarrollo
    de redes neuronales de manera muy sencilla e intuitiva. 

    En el presente proyecto, se utiliza Tensorflow como librería base para la implementación de la red neuronal tanto en la etapa de 
    entrenamiento como en la etapa de inferencia. El entrenamiento e inferencia se implementan usando los algoritmos de Tensorflow 
    optimizados para GPU\footnote{GPU: Graphics Processing Unit o Unidad de Procesamiento de Gráficos} de la marca Nvidia.% TODO: agregar referencia a otros capitulos

    Es importante listar algunos términos que se usarán en el contexto de este proyecto, relacionados exclusivamente con la implementación 
    de la red neuronal convolucional correspondiente con el sistema fin a fin que se implementa. 
    \begin{itemize}
        \item \textbf{Tensor:} Es una generalización de un vector o una matriz en dimensiones superiores. Internamente, 
        Tensorflow representa tensores como arreglos n-dimensionales de tipos de datos base, como ser Int32 o Float64.

        \item \textbf{Variable: } Refiere a la manera de presentar el estado persistente que se puede manipular por el 
        programa o grafo de cómputo. Una variable contiene internamente un tensor con valores que se pueden modificar 
        mediante operaciones. Las variables en Tensorflow comunmente se utilizan para representar a los pesos o parámetros 
        de la red neuronal.

        \item \textbf{Grafo:} Un grafo es un objeto de Tensorflow que contiene la información acerca de la estructura 
        del grafo de cómputo que se va a utilizar. Contiene la información de las distintas operaciones y las conexiones 
        entre las mismas por las que fluyen los tensores. La estructura del grafo debe ser declarada antes de su ejecución.

        \item \textbf{Operación:} Una operación representa a un nodo en el grafo, tiene como entrada uno o varios 
        tensores y produce como salida uno o varios tensores. Las operaciones definen los cálculos que se realizan entre 
        tensores como ser una multiplicación de matrices o una operación de convolución, entre otras.
    \end{itemize}

    En el siguiente ejemplo, se puede observar la definición de un grafo de cómputo básico en Tensorflow:

    \begin{lstlisting}[title={Ejemplo de un programa escrito con Tensorflow},captionpos=b,language=Python]
        import tensorflow as tf 
            #definicion de variables
            input1 = tf.Variable(3.0) 
            input2 = tf.Variable(2.0)
            input3 = tf.Variable(5.0)

            #definicion de las operaciones y el grafo
            intermed = tf.add(input2,input3)
            mul = tf.mul(input1,intermed)

            #ejecucion de las operaciones 
            with tf.Session() as sess:
                result = sess.run([mul,intermed])
                print(result) 

    \end{lstlisting}

    \subsection{Keras}
    Keras es una librería para la definición e implementación de redes neuronales de alto nivel escrita en Python y compatible con 
    diversas plataformas de cómputo tales como Tensorflow, CNTK o Theano \cite{chollet2015keras}. Esta librería ha sido desarrollada con el objetivo de 
    facilitar la experimentación y prototipado rápido de modelos de aprendizaje profundo. Las características de la librería que 
    la convierten en una opción viable en el desarrollo de modelos de aprendizaje profundo son las siguientes:

    \begin{itemize}
        \item Permite el prototipado rápido a través de su facilidad de uso, modularidad y capacidad de ser extendida.
        \item Soporta la definición de redes neuronales recurrentes y redes neuronales convolucionales. La última categoría es la más importante para el presente proyecto.
        \item Soporta la ejecución tanto en CPU como en GPU.
    \end{itemize}

    Keras se basa en la definición de redes neuronales en base a capas. Existe una clase especial de modelo llamado \textit{Sequential} que 
    representa básicamente una red neuronal feedforward (Sección(\ref{sec:feedforward})). En un modelo \textit{Sequential} se define 
    a la red en base a las capas de las que se compone, cada capa puede tener distinta naturaleza y características. 
    
    Además de la definición de las capas, Keras también cuenta con implementaciones de algoritmos de optmización y funciones de costo 
    comunmente utilizadas en trabajos de investigación en la actualidad, lo cual facilita todavía más el desarrollo de modelos de 
    redes neuronales. En el siguiente 
    ejemplo, se puede apreciar la definición de la red neuronal de dos capas definida en la Ecuación(\ref{eq:reddoscapas}) con 32 unidades en 
    la capa de entrada y 4 unidades en la capa de salida, con una función de costo de entropía cruzada categórica y el algoritmo de 
    optimización de \textit{Stochastic Gradient Descent}:

    \begin{lstlisting}[title={Ejemplo de una red neuronal usando la librería Keras},captionpos=b,language=Python]
        from keras.models import Sequential
        from keras.layers import Dense, Activation

        modelo = Sequential()
        #primera capa
        model.add(Dense(32), input_dim=128)
        model.add(Activation('sigmoid'))
        #segunda capa
        model.add(Dense(4), input_dim=128)
        model.add(Activation('sigmoid'))
        #optimizador y funcion de costo
        model.compile(loss='categorical_crossentropy',
                            optimizer='sgd',
                            metrics=['accuracy']
                            )
    \end{lstlisting}

    \subsection{ARM Mbed}
    Mbed es una iniciativa llevada adelante por ARM que brinda un conjunto de herramientas de hardware y software para el 
    desarrollo de dispositivos IoT (Internet de las Cosas). Mbed es un ecosistema de desarrollo sobre el cual se pueden 
    desarrollar aplicaciones con microcontroladores con arquitectura ARM provenientes de distintos fabricantes \cite{mbed}. La 
    característica principal de Mbed es la sencillez de su uso y la amplia gama de librerías disponibles para distintos componentes 
    de hardware como sensores, actuadores o displays. ARM Mbed presenta las siguientes características clave para el desarrollo 
    de sistemas embebidos de manera rápida y favorables al contexto del presente proyecto:

    \begin{itemize}
        \item Variedad de placas de desarrollo de microcontroladores ARM de distintos fabricantes.
        \item Una interfaz de programación común a todos los microcontroladores y fabricantes para la interfaz con 
        periféricos embebidos.
        \item Un compilador en línea donde se pueden crear, compilar y desplegar proyectos.
        \item Variedad de librerías para dispositivos como sensores, módulos de comunicación o actuadores.
        \item El uso del lenguaje C++ con la especificación completa hace posible el desarrollo orientado 
        a objetos para aprovechar altos niveles de abstracción en la programación.
        \item La capacidad de generación de símbolos de depuración para su ejecución paso por paso con el fin de 
        identificar \textit{bugs} en tiempo de ejecución.
        \item La compatibilidad con la librería Rosserial que hace posible poder comunicar al microcontrolador 
        con el \textit{middleware} de ROS de manera directa usando un puerto serial.
    \end{itemize}

    Mbed se usará como la plataforma para el desarrollo del control de tiempo real en el subsistema de control y actuación 
    ya que ofrece todas las facilidades en cuanto a librerías y potencia computacional necesarias para esta tarea. En la 
    Tabla(\ref{tbl:frameworks}) se puede observar una tabla comparativa entre diversas plataformas de desarrollo embebido 
    consideradas para el presente proyecto.


    \begin{table}[!h]
        \begin{tabular}{|c|c|c|c|c|}
        \hline
        \textbf{Plataforma} & \textbf{Arquitectura}                                                          & \textbf{Lenguaje} & \textbf{\begin{tabular}[c]{@{}c@{}}Nivel de \\ abstracción\end{tabular}} & \textbf{Enfoque}                                                          \\ \hline
        Arduino             & AVR - 8 bits                                                                   & Wiring (C++)      & Alto                                                                     & \begin{tabular}[c]{@{}c@{}}Hobby, Arte \\ y Educación\end{tabular}        \\ \hline
        Mbed                & ARM - 32 bits                                                                  & C++               & Alto                                                                     & \begin{tabular}[c]{@{}c@{}}IoT, Sistemas \\ de tiempo real\end{tabular}   \\ \hline
        Energia             & \begin{tabular}[c]{@{}c@{}}TI MSP430 - 16 bits\\ TI ARM - 32 bits\end{tabular} & Wiring (C++)      & Alto                                                                     & \begin{tabular}[c]{@{}c@{}}Hobby, Sistemas \\ de tiempo real\end{tabular} \\ \hline
        Freedom E SDK       & RISC V - 32 bits                                                               & C                 & Bajo                                                                     & \begin{tabular}[c]{@{}c@{}}Sistemas de \\ tiempo real\end{tabular}        \\ \hline
        libOpenCM3          & ARM - 32 bits                                                                  & C                 & Bajo                                                                     & \begin{tabular}[c]{@{}c@{}}Sistemas de \\ tiempo real\end{tabular}        \\ \hline
        \end{tabular}
    \end{table}

    Una de las plataformas de desarrollo más utilizadas en la actualidad es Arduino, pese a haberse considerado esta plataforma 
    por su disponibilidad, popularidad y gran soporte por la comunidad se ha detectado algunas limitaciones en la misma 
    que hacen que no se la pueda recomendar para desarrollos académicos:

    \begin{itemize}
        \item Si bien el nivel de abstracción facilita la introducción a los microcontroladores para personas 
        sin experiencia, oculta varios aspectos referidos al hardware de los periféricos del microcontrolador que 
        escapan de control. Esta falta de control de bajo nivel puede ocasionar fallos y situaciones en las que no 
        se pueda predecir con seguridad el comportamiento de un sistema. La predictibilidad es una característica 
        fundamental en cualquier sistema de tiempo real.

        \item El lenguaje de programación usado Wiring es un subconjunto del lenguaje C++ que carece de varias 
        funcionalidades y no permite el desarrollo de clases con herencia y polimorfismo implementadas de manera adecuada.

        \item El entorno de desarrollo integrado porporcionado, el Arduino IDE, es un entorno demasiado limitado 
        para desarrollos de proyectos de mediana y gran envergadura.

        \item La falta de capacidades de depuración, una limitación de la propia arquitectura AVR imposibilita el 
        análisis del comportamiento en tiempo de ejecución del código y la identificación de posibles \textit{bugs} 
        que puedan aparecer. Esta característica es de vital importancia para el desarrollo de sistemas de seguridad crítica.

    \end{itemize}


\section{Herramientas de hardware}
Se han seleccionado diversas herramientas de hardware para la implementación del sistema de conducción autónoma. Dichas herramientas
corresponden con la base física electrónica sobre la cual se ejecutarán las tareas de los tres subsistemas. Se procede a detallar 
las herramientas utilizadas en el diseño del sistema.

    \subsection{Plataforma de tiempo real}
    La interfaz de más bajo nivel del sistema es el de la interacción con los actuadores de los motores del prototipo. Esta 
    interfaz debe tener la capacidad de poder comunicarse con la OBC y además de poder cumplir ciertos requisitos de ejecución 
    en tiempo real. Estos requisitos de tiempo real hace que tal comportamiento no se pueda implementar en la OBC pues la misma 
    usa un sistema operativo basado en el kernel GNU/Linux, el cual no cuenta por defecto con capacidades de tiempo real dura. 
    Por tanto, se ha establecido la necesidad de utilizar una plataforma embebida con una arquitectura más sencilla y con un 
    nivel de predictibilidad mucho mayor al de la OBC. En este caso se usará un microcontrolador con arquitectura ARM Cortex M3.

    La arquitectura ARM se ha popularizado bastante en los últimos años principalmente por su característica de tener un conjunto 
    reducido de instrucciones y sencillez en la microarquitectura del procesador en comparación con otras arquitecturas comúnmente 
    encontradas en servidores y computadoras personales como son X86 o PowerPC. Esta simplificación en la arquitectura y la reproducción
    del conjunto de instrucciones ha permitido que los dispositivos basados en ARM puedan reducir dramáticamente el consumo de 
    energía sin degradar demasiado el rendimiento. Es por eso que ARM, en la actualidad se constituye como la principal arquitectura en 
    dispositivos móviles y de bajo consumo con cientos de millones de dispositivos usándola alrededor del mundo.

    Sin embargo, el desarrollo e implemtación de esta arquitectura se ha dirigido bastante hacia procesadores de aplicación, presentes
    en dispositivos como teléfonos inteligentes o tablets. Es por eso que ARM ha presentado una familia de procesadores ARM que estan
    orientados exclusivamente al desarrollo de sistemas embebidos con capacidades de tiempo real y ultra bajo conmo de energía. Esta 
    familia es la familia ARM Cortex-M que cuenta con varias características que la hacen ideal para el desarrollo del sistema en tiempo 
    real requerido para la interfaz con los actuadores del presente proyecto.

    Por su parte, dado que ARM no fabrica chips sino mas bien vende licencias de la arquitectura a distintas marcas fabricantes, existe 
    una multitud de procesadores usando esta arquitectura de distintos fabricantes, entre los cuales podemos mencionar a ST Microelectronics,
    Texas Instruments, Nordic, entre otros. Para el presente proyecto, se necesita que el microcontrolador pueda cumplir con las 
    siguientes características. 

    \begin{itemize}
        \item Al menos un puerto de comunicación serial.
        \item Periféricos capaces de generar señales PWM o con los recursos necesarios para emular PWM por software.
        \item Memoria de datos y de programa suficiente para poder incluir la librería de Rosserial en el mismo.
        \item Capacidades de depuración y ejecución paso por paso para la identificación de fallas.
        \item Operación en niveles de tensión compatibles con la OBC.
    \end{itemize}

    Es por eso que se ha seleccionado la placa de desarrollo de ST Microelectronics Mucleo f303k8 (Figura()), tomando en cuenta 
    que cumple con todos los requisitos anteriormente establecidos y además cuenta con las siguientes características:


    \begin{figure}[!h] 
        \centering
        \includegraphics[width=0.55\textwidth]{img/nucleo}
        \caption[Placa de desarrollo Nucleof303k8 de ST Microelectronics]{Placa de desarrollo Nucleof303k8 de ST Microelectronics. Fuente: \cite{nucleof303} }
        \label{fig:nucleo}
    \end{figure}


    \begin{itemize}
        \item Circuito grabador-depurador en la placa, STlink V2.
        \item Capacidades de comunicación seria mediante USB para un puerto de comunicación seria virtual e intefaz de depuración.
        \item Múltiples fuentes de alimentación.
        \item Leds indicadores.
        \item Factor de forma compatible con varios entornos de desarrollo electrónico.
    \end{itemize}


    Se han explorado diversas alternativas al uso de la placa de desarrollo Nucleof303k8 para la implementación de este módulo. En 
    la Tabla(\ref{tbl:mcucomp}) se pueden apreciar las características de varias placas de desarrollo embebido candidatas disponibles en el mercado.

    % Please add the following required packages to your document preamble:
    % \usepackage{booktabs}
    % \usepackage{graphicx}
    \begin{table}[]
        \centering
        \resizebox{\textwidth}{!}{%
        \begin{tabular}{@{}|c|c|c|c|c|c|c|@{}}
        \toprule
        \textbf{Modelo} & \textbf{Arquitectura}                                                & \textbf{\begin{tabular}[c]{@{}c@{}}RAM\\ (KB)\end{tabular}} & \textbf{\begin{tabular}[c]{@{}c@{}}ROM\\ (KB)\end{tabular}} & \textbf{\begin{tabular}[c]{@{}c@{}}Frecuencia de \\ Reloj (MHz)\end{tabular}} & \textbf{\begin{tabular}[c]{@{}c@{}}Canales \\ PWM\end{tabular}} & \textbf{\begin{tabular}[c]{@{}c@{}}Entorno de \\ Trabajo\end{tabular}} \\ \midrule
        Atmega328p      & AVR (8 bits)                                                         & 2                                                           & 32                                                          & 16                                                                            & 6 (8 bits)                                                      & \begin{tabular}[c]{@{}c@{}}Arduino,\\ Atmel Studio\end{tabular}        \\ \midrule
        PIC18f2550      & PIC (8 bits)                                                         & 2                                                           & 32                                                          & 48                                                                            & 2 (8 bits)                                                      & MPlab                                                                  \\ \midrule
        STM32f103c8     & \begin{tabular}[c]{@{}c@{}}ARM \\ CortexM3\\  (32 bits)\end{tabular} & 20                                                          & 64                                                          & 72                                                                            & 12 (16 bits)                                                    & \begin{tabular}[c]{@{}c@{}}STM32 HAL,\\ Arduino\end{tabular}           \\ \midrule
        STM32f303k8     & \begin{tabular}[c]{@{}c@{}}ARM \\ CortexM4f\\ (32 bits)\end{tabular} & 16                                                          & 64                                                          & 72                                                                            & 8                                                               & \begin{tabular}[c]{@{}c@{}}STM32 HAL,\\ \textbf{Mbed}\end{tabular}              \\ \bottomrule
        \end{tabular}%
        }
        \caption{Comparación de características de microcontroladores disponibles. Fuente: Elaboración propia.}
        \label{tbl:mcucomp}
        \end{table}

    

    El microcontrolador utilizado, el STM32F303k8 pertenece a la familia de microcontroladores con arquitectura ARM Cortex M4f del 
    fabricante ST Microelectronics. Es parte de la gama de microcontroladores f3 de la famila STM32 que esta orientado al procesamiento 
    de señales pues cuenta con un procesador de 32 bits con una unidad de punto flotante que le permite realizar cálculos y operaciones 
    con números decimales de manera eficiente. Las características del microcontrolador seleccionado se listan a continuación:

    % Please add the following required packages to your document preamble:
    % \usepackage{booktabs}
    % \usepackage{graphicx}
    \begin{table}[]
        \centering
        \resizebox{0.8\textwidth}{!}{%
        \begin{tabular}{@{}|r|l|@{}}
        \toprule
        \textbf{Núcleo}                     & ARM Cortex M4 de 32 bits con unidad de punto flotante \\ \midrule
        \textbf{Frecuencia de reloj}        & 72 MHz Máximo                                         \\ \midrule
        \textbf{Voltaje de operación}       & 2.0 V a 3.6 V (nominal: 3.3 V)                        \\ \midrule
        \textbf{Memoria de datos}           & 16 KB SRAM                                            \\ \midrule
        \textbf{Memoria de programa}        & 64 KB FLASH                                           \\ \midrule
        \textbf{Timers Disponibles}         & 7                                                     \\ \midrule
        \textbf{Interfaces de Comunicación} & SPI/I2S, I2C, USART, CAN                              \\ \midrule
        \textbf{Periféricos adicionales}    & GPIO (con interrupciones), ADC, DAC, RTC              \\ \bottomrule
        \end{tabular}%
        }
        \caption{Características técnicas del microcontrolador STM32f303k8. Fuente: \cite{nucleof303}}
        \label{tbl:mcuspecs}
        \end{table}

    % \begin{itemize}
    %     \item \textbf{Núcleo: } ARM CortexM4 de 32 bits con unidad de punto flotante.
    %     \item \textbf{Frecuencia de reloj: } 72 MHz máximo.
    %     \item \textbf{Voltaje de operación: } desde 2.0 V hasta 3.6V (nominal 3.3V).
    %     \item \textbf{Memoria de datos: } 16 KB SRAM.
    %     \item \textbf{Memoria de programa: } 64 KB FLASH.
    %     \item \textbf{Timers: } 7.
    %     \item \textbf{Interfaces de comunicación: } SPI/I2S, I2C, USART, CAN.
    %     \item \textbf{Periféricos adicionales: } GPIO (Con interrupciones), ADC, DAC, RTC.
    % \end{itemize}

    Como se puede observar en la Tabla(\ref{tbl:mcuspecs}), las características del microcontrolador cumplen con los requisitos establecidos tanto en interfaces 
    de comunicación como en cantidad y variedad de periféricos para una posterior extensión de la funcionalidad. Por otra parte, 
    tal como se puede apreciar en la Tabla(\ref{tbl:mcucomp}), esta placa es compatible con el entorno de trabajo de programación 
    ARM Mbed, y con la librería Rosserial.

    \subsection{Computadora de Abordo - OBC}

    En el caso del hardware necesario para la OBC, se requiere un sistema capaz ejecutar un sistema operativo GNU/Linux completo 
    con el fin de poder correr el software necesario para el control, comunicación y adquisición de imágenes de una cámara, necesarias
    para el funcionamiento correcto del sistema en su conjunto. Es en este sentido que se ha optado por el uso de una \textit{Single Board Computer}
    o Computadora de Una Placa, que refiere a placas de desarrollo con todas las características de una computadora de escritorio, es decir:
    procesador, memoria y periféricos incorporados. Existe gran variedad de SBC en el mercado con distintas características y aplicaciones 
    objetivo. En la Tabla(\ref{tbl:sbc}) se puede apreciar una comparación de varias SBC disponibles en el mercado y sus características 
    relevantes al presente proyecto.

   % Please add the following required packages to your document preamble:
    % \usepackage{booktabs}
    % \usepackage{graphicx}
    \begin{table}[]
        \centering
        \resizebox{\textwidth}{!}{%
        \begin{tabular}{@{}|c|c|c|c|c|c|c|c|@{}}
        \toprule
        \textbf{Placa}                                                & \textbf{Procesador}                                                          & \textbf{RAM} & \textbf{LAN} & \textbf{\begin{tabular}[c]{@{}c@{}}Conexión\\ Inalámbrica\end{tabular}} & \textbf{\begin{tabular}[c]{@{}c@{}}Soporte y\\ documentación\end{tabular}} & \textbf{\begin{tabular}[c]{@{}c@{}}Precio\\ (\$us)\end{tabular}} & \textbf{\begin{tabular}[c]{@{}c@{}}Soporta \\ ROS\end{tabular}} \\ \midrule
        \begin{tabular}[c]{@{}c@{}}Asus \\ Tinkerboard\end{tabular}   & 4x A17 @ 1.8 GHz                                                             & 2 GB         & GBe          & \begin{tabular}[c]{@{}c@{}}Wifi\\ Bluetooth\end{tabular}                & Regular                                                                    & 60                                                               & NO                                                              \\ \midrule
        Odroid X4u                                                    & \begin{tabular}[c]{@{}c@{}}4x A15 @ 2.0 GHz +\\ 4x A7 @ 1.4 GHz\end{tabular} & 2 GB         & GBe          & NO                                                                      & Bueno                                                                      & 85                                                               & SI                                                              \\ \midrule
        \begin{tabular}[c]{@{}c@{}}Raspberry Pi\\ zero W\end{tabular} & 1x A8 @ 1GHz                                                                 & 512 MB       & NO           & \begin{tabular}[c]{@{}c@{}}Wifi\\ Bluetooth\end{tabular}                & Muy bueno                                                                  & 20                                                               & NO                                                              \\ \midrule
        \begin{tabular}[c]{@{}c@{}}Beaglebone \\ Black\end{tabular}   & 1x A8 @ 1GHz                                                                 & 512 MB       & Fast         & NO                                                                      & Bueno                                                                      & 60                                                               & SI                                                              \\ \midrule
        Raspberry Pi 3b+                                              & 4x A53 @ 1.4 GHz                                                             & 1 GB         & Fast         & \begin{tabular}[c]{@{}c@{}}Wifi\\ Bluetooth\end{tabular}                & Muy bueno                                                                  & 40                                                               & SI                                                              \\ \midrule
        Rock64                                                        & 4x A53 @ 1.5 GHz                                                             & 2 GB         & GBe          & NO                                                                      & Regular                                                                    & 35                                                               & NO                                                              \\ \midrule
        Beaglebone Blue                                               & 1x A8 @ 1GHz                                                                 & 512 MB       & no           & \begin{tabular}[c]{@{}c@{}}Wifi\\ Bluetooth\end{tabular}                & Regular                                                                    & 80                                                               & SI                                                              \\ \bottomrule
        \end{tabular}%
        }
        \caption{Tabla comparativa de SBC's disponibles en el mercado. Fuente: Elaboración propia.}
        \label{tbl:sbc}
        \end{table}


    La placa seleccionada como OBC debe cumplir ciertas características específicas para este proyecto:


    \begin{itemize}
        \item Factor de forma: Dimensiones y peso reducidos para poder ser incorporada en el prototipo.
        \item Consumo de energía: Bajo consumo de energia, es necesario que pueda ser alimentado por baterías disponibles en el mercado.
        \item Compatibilidad de software: La placa debe poder correr una distribución completa de GNU/Linux reciente, compatible con ROS.
        \item Compatibilidad de hardware: La placa debe poder conectarse de manera nativa con el microcontrolador elegido para la interfaz de tiempo real. Contar con puertos USB y puertos USART.
        \item Interfaz con una cámara: Debe contar también con una forma de conectar y adquirir imágenes provenientes de una cámara digital.
        \item Comunicación inalámbrica: Se necesita la capacidad de poder conectarse a una red LAN mediante Wifi, ya sea con un módulo incorporado o un accesorio externo.
        \item Precio y disponibilidad: Es importante que el precio no sea demasiado elevado para poder garantizar la replicabilidad del proyecto, así como también la disponibilidad en el mercado.
        \item Soporte y documentación: Es necesario que la placa cuente con una buena documentación y soporte de la comunidad o fabricante para que el resolver problemas relativos a la placa no tome demasiado tiempo.
    \end{itemize}

    \begin{figure}[!h] 
        \centering
        \includegraphics[width=0.75\textwidth]{img/raspi}
        \caption[Placa de desarrollo Raspberry Pi 3 model B+]{Placa de desarrollo Raspberry Pi 3 model B+. Fuente: \cite{raspi} }
        \label{fig:raspi}
    \end{figure}

    Al final, se ha seleccionado a la placa Raspberry Pi 3B+ (Figura()) para la OBC del proyecto pues cuenta con todas las características 
    necesarias para poder ejecutar las herramientas de software requeridas y también cuenta con los periféricos adecuados para 
    comunicarse con la plataforma de tiempo real. La placa en cuestión cuenta con las características listadas en la Tabla(\ref{tbl:raspispecs})

    
    % Please add the following required packages to your document preamble:
    % \usepackage{booktabs}
    % \usepackage{graphicx}
    \begin{table}[]
        \centering
        \resizebox{0.85\textwidth}{!}{%
        \begin{tabular}{@{}|r|l|@{}}
        \toprule
        \textbf{SoC}                        & Broadcom BCM2837B0                                                  \\ \midrule
        \textbf{CPU}                        & 1.4 GHz 64-bit quad core ARM Cortex-A53                             \\ \midrule
        \textbf{RAM}                        & 1 GB LPDDR2 SDRAM                                                   \\ \midrule
        \textbf{Wifi}                       & Dual-band 802.11ac wireless LAN (2.4GHz and 5GHz ) y Bluetooth 4.2. \\ \midrule
        \textbf{Video}                      & VideoCore IV 3D                                                     \\ \midrule
        \textbf{USB}                        & 2.0, 4 puertos                                                      \\ \midrule
        \textbf{Interfaces de Comunicación} & SPI/I2S, I2C, USART                                                 \\ \midrule
        \textbf{Interfaces de cámara}       & USB webcam, CSI                                                     \\ \bottomrule
        \end{tabular}%
        }
        \caption{Características de la placa Raspberry Pi 3 model B+. Fuente: \cite{raspi}}
        \label{tbl:raspispecs}
        \end{table}


    % \begin{itemize}
    %     \item Soc\footnote{System on Chip}: Broadcom BCM2837B0, Cortex-A53 (ARMv8) 64-bit.
    %     \item CPU: 1.4GHz 64-bit quad-core ARM Cortex-A53 CPU.
    %     \item RAM: 1GB LPDDR2 SDRAM.
    %     \item Wifi: Dual-band 802.11ac wireless LAN (2.4GHz and 5GHz ) and Bluetooth 4.2.
    %     \item Video: VideoCore IV 3D.
    %     \item USB: 2.0, 4 ports.
    % \end{itemize}


    \subsection{Estación de trabajo}
    La estación de trabajo se usa para realizar el entrenamiento de la red neuronal en el subsistema de adquisición de datos y 
    entrenamiento. Dada la naturaleza de la tarea del entrenamiento de una red neuronal, esta estación de trabajo presenta algunos 
    requerimientos especiales: 

    \begin{itemize}
        \item Sistema Operativo: Se necesita un sistema operativo basado en GNU/Linux compatible con ROS, Tensorflow y Keras.
        \item GPU: Con la finalidad de acelerar el tiempo de entrenamiento de la red neuronal convolucional, es altamente recomendable contar con una GPU de la marca Nvidia, compatible con Tensorflow.
        \item Wifi: Es necesaria una conexión Wifi inalámbrica para poder comunicarse con el prototipo tanto en la etapa de adquisición de datos y entrenamiento, como en la de inferencia para el monitoreo remoto.
    \end{itemize}
    
    Considerando los requisitos, las características de la estación de trabajo seleccionada se listan en la Tabla(\ref{tbl:pcspecs}). 

    % Please add the following required packages to your document preamble:
    % \usepackage{booktabs}
    % \usepackage{graphicx}
    \begin{table}[!h]
        \centering
        \resizebox{0.85\textwidth}{!}{%
        \begin{tabular}{@{}|r|l|@{}}
        \toprule
        \textbf{Modelo}            & MSI GL62 6qd                                                              \\ \midrule
        \textbf{CPU}               & Intel Core i7-6700HQ @ 2.6 GHz x 8 núcleos                                \\ \midrule
        \textbf{RAM}               & 16 GB DDR4 @ 2133 MHz                                                     \\ \midrule
        \textbf{Wifi}              & Intel Dual-band 802.11ac wireless LAN (2.4GHz and 5GHz ) y Bluetooth 4.2. \\ \midrule
        \textbf{GPU}               & Nvidia GTX950m 2GB VRAM                                                   \\ \midrule
        \textbf{Sistema Operativo} & Ubuntu 18.04 Linux 4.15                                                   \\ \bottomrule
        \end{tabular}%
        }
        \caption{Características de la estación de trabajo seleccionada. }
        \label{tbl:pcspecs}
        \end{table}

    La estación de trabajo se utilizará para el entrenamiento de la red neuronal pues cuenta con los requisitos de memoria RAM y una 
    GPU compatible para la paralelización de los algoritmos de entrenamiento de la librería Tensorflow. Estos algoritmos incluyen 
    el cálculo de gradientes de toda la red y el ajuste de los pesos optimizados para ejecutarse de forma paralela dada la naturaleza
    matricial de las operaciones involucradas.

    \subsection{Sensores}
    Un aspecto fundamental en el desarrollo de un sistema de conducción autónoma es la elección adecuada de los sensores, que son 
    los dispositivos que recolectan datos acerca del estado del entorno del vehículo. En el presente proyecto, se están utilizando 
    dos tipos de sensores: Una cámara para recuperar las imágenes de la carretera y un sensor de proximidad para detectar obstáculos 
    al frente del vehículo.
        \subsubsection{Cámara}
        En el caso de la cámara, existen diversas opciones para poder recuperar imágenes del entorno variando desde la calidad 
        de la imagen que recuperan, velocidad de captura, rango dińamico y otras características. Sin embargo, se debe tomar 
        en cuenta algunas limitaciones impuestas por el sistema en su conjunto:

        \begin{itemize}
            \item La capacidad de procesamiento es limitada en la OBC al tratarse de un sistema de bajo consumo de energía y dimensiones reducidas.
            \item La capacidad de transferencia en la red está limitada a especificaciones de los módulos de comunicación inalámbrica, reduciendo la cantidad de información que puede ser compartida entre la estación de trabajo remota y el prototipo.
            \item El consumo de energía debe ser limitado al estar el sistema alimentado por baterías.
        \end{itemize}

        Considerando las limitaciones planteadas, se puede reducir las opciones de cámaras a utilizar en este proyecto a cámaras web 
        y al módulo de cámara de Raspberry Pi (Figura(\ref{fig:raspicam})). El último ítem cuenta con algunas características muy interesantes que la hacen una 
        candidata idónea para su uso en el presente proyecto que se pueden observar en la Tabla(\ref{tbl:raspicamspecs}).

        \begin{figure}[!h] 
            \centering
            \includegraphics[width=0.65\textwidth]{img/raspicam}
            \caption[Raspberry Pi Camera Module V2]{Raspberry Pi Camera Module V2. Fuente: \cite{raspicam} }
            \label{fig:raspicam}
        \end{figure}

        % Please add the following required packages to your document preamble:
        % \usepackage{booktabs}
        % \usepackage{graphicx}
        \begin{table}[!h]
            \centering
            \resizebox{0.5\textwidth}{!}{%
            \begin{tabular}{@{}|r|l|@{}}
            \toprule
            \textbf{Sensor}      & Sony IMX219 8 megapixeles          \\ \midrule
            \textbf{Resolución}  & Fotografía: 3280x2464 Video: 1080p \\ \midrule
            \textbf{Framerate}   & 1080p@30Hz, 720p@60Hz, 640p@90Hz   \\ \midrule
            \textbf{Interfaz}    & CSI                                \\ \midrule
            \textbf{Dimensiones} & 25mm x 23mm x 9mm                  \\ \midrule
            \textbf{Peso}        & 3.4g                               \\ \bottomrule
            \end{tabular}%
            }
            \caption{Características de la Raspberry Pi Camera Module V2. Fuente: \cite{raspicam}}
            \label{tbl:raspicamspecs}
            \end{table}
        
        Dentro de las razones por la cual se ha escogido a la Raspberry Pi Camera Module V2 podemos mencionar la capacidad de 
        grabar video a 90 cuadros por segundo, una característica importante para poder incrementar el tiempo de muestreo del 
        bucle de control. En segundo lugar, la cámara utiliza una interfaz CSI que puede conectarse directamente a la placa Raspberry
        Pi tal como se puede apreciar en la Figura(\ref{fig:raspicamconn})
        
        \begin{figure}[!h] 
            \centering
            \includegraphics[width=0.65\textwidth]{img/raspicamconn}
            \caption[Conexión de la Camera Module V2]{Conexión de la Camera Module V2 con una placa Raspberry Pi. Fuente: \cite{raspicam} }
            \label{fig:raspicamconn}
        \end{figure}

        La ventaja de poderse conectar con la placa mediante la interfaz CSI es que los \textit{frames} provenientes del sensor 
        se procesan directamente en la GPU de la Raspberry Pi y no así en la GPU, esto gracias a la disponibilidad de los 
        controladores correspondientes para la Camera Module V2. Esto no es posible cuando se trata de cámaras USB o webcam 
        debido a que los controladores de las mismas procesan los \textit{frames} en la CPU haciendo su procesamiento más lento 
        y ocupando ciclos de CPU.

        \subsubsection{Sensor de proximidad}
        El segundo sensor necesario para la implementación del sistema de control autónomo es un sensor de proximidad que 
        sea capaz de detectar la presencia y distancia de obstáculos que se presenten en frente del vehículo. Este sensor 
        también debe ser compatible con los protocolos de comunicación o periféricos disponibles en la OBC y en el sistema 
        de control de tiempo real.

        \begin{figure}[!h] 
            \centering
            \includegraphics[width=0.35\textwidth]{img/laser}
            \caption[Sensor VL53L0X de ST Microelectronics]{Conexión de la Camera Module V2 . Fuente: \cite{laser} }
            \label{fig:laser}
        \end{figure}

        El sensor seleccionado para la tarea es el sensor VL53L0X (Figura(\ref{fig:laser})) de ST Microelectronics. El cual es un sensor óptico de 
        proximidad con láser que usa el principio ToF{\textit{Time of Flight}} o Tiempo de Vuelo, se basa en el tiempo 
        en el que la luz tarda en reflejarse de la superficie para calcular la distancia dada una velocidad constante de la luz 
        mediante la siguiente fórmula:
        
        \begin{equation}\label{eq:tof}
            d = \frac{c}{t_{vuelo} / 2}
        \end{equation}

        donde $d$ es la distancia recorrida y $c$ es la velocidad de la luz. Se puede apreciar una ilustración del principio 
        de funcionamiento del sensor en la Figura(\ref{fig:tof}). 
        
        \begin{figure}[!h] 
            \centering
            \includegraphics[width=0.35\textwidth]{img/tof}
            \caption[Ilustración del principio ToF]{Ilustración del principio ToF. Fuente: \cite{laser} }
            \label{fig:tof}
        \end{figure}

        Los sensores láser que usan el principio ToF tienen varias ventajas en relación a otro tipo de sensores como los sensores 
        ultrasónicos, o los sensores infrarrojos refractarios. En primer lugar, son inmunes a interferencia sonora o ultrasónica 
        y son más robustos en cuanto a la variedad de superfices sobre las cuales se puede reflejar el haz de luz. El sensor 
        VL53L0X se constituye como una opción favorable para el desarrollo del presente proyecto y por eso se ha elegido. Sus 
        características se detallan en la Tabla(\ref{tbl:laserspecs}).

        % Please add the following required packages to your document preamble:
        % \usepackage{booktabs}
        % \usepackage{graphicx}
        \begin{table}[!h]
            \centering
            \resizebox{0.65\textwidth}{!}{%
            \begin{tabular}{@{}|r|l|@{}}
            \toprule
            \textbf{Láser}                     & 940 nm Clase 1 (seguro a los ojos) \\ \midrule
            \textbf{Rango efectivo}            & 30 mm  - 1200 mm                   \\ \midrule
            \textbf{Interfaz de comunicación}  & I2C @ 400 KHz                      \\ \midrule
            \textbf{Tiempo de muestreo mínimo} & 20 ms                              \\ \midrule
            \textbf{Voltaje de operación}      & 2.6 V  - 3.5 V (nominal: 3.3 V)    \\ \bottomrule
            \end{tabular}%
            }
            \caption{Características del sensor VL53L0X. Fuente: \cite{vl53l0x}}
            \label{tbl:laserspecs}
            \end{table}
        
            
        El sensor se ha utilizado en una placa de desarrollo o \textit{Breakout Board} que incluye un regulador de tensión 
        y los pines de comunicación correspondientes para su conexión a la SBC. La información proveniente del sensor 
        de proximidad será recuperada y procesada mediante un nodo de ROS de manera que se pueda integrar al sistema 
        en su conjunto.
            
            \begin{figure}[] 
                \centering
                \includegraphics[width=0.45\textwidth]{img/laserbreak}
                \caption[Breakout Board para el sensor VL53L0X]{Breakout Board para el sensor VL53L0X. Fuente: \cite{laser} }
                \label{fig:laserboard}
            \end{figure}
    
\section{Subsistema de Control y actuación}

El subsistema de control y actuación representa la base fundamental sobre la cual se desarrolla el resto del proyecto. De acuerdo 
con la naturaleza del mismo, se pretende implementar una plataforma flexible y modular sobre la cual se pueda controlar el 
prototipo de múltiples maneras, partiendo desde un modo teleoperado básico donde un operador humano tiene el control 
de todos los grados de libertad del vehículo, pasando por un modo híbrido hasta un modo autónomo donde los comandos de 
control son generados por un programa que procesa los datos provenientes de los sensores y otras fuentes externas.

La finalidad del subsistema es brindar una interfaz amigable y flexible para el control de los actuadores del vehículo. Esta 
interfaz se logra gracias a la infraestructura de ROS mediante el envío de mensajes. Los mensajes utilizados en este subsistema 
son los mensajes Twist de los cuales se detalla sus características en la Tabla(). Se ha escogido el mensaje Twist porque presenta 
de manera muy conveniente la separación entre la velocidad lineal o la tracción del vehículo y la desviación o la dirección. 

% tabla del mensaje twist

La interfaz con el \textit{middleware} de ROS se logra a través de un paquete llamado Rosserial, que crea un nodo especial 
que es capaz de enviar mensajes mediante el puerto serial a sistemas embebidos que no cuentan con una conexión ethernet. De esta 
manera, el control de los actuadores es transparente a cualquier otro nodo en la red haciendo posible que se pueda controlar 
el vehículo de distintas maneras, incluso desde nodos ejecutados de manera remota en estaciones de trabajo en la red local.

    \subsection{Módulo de potencia en tiempo real}
    El prototipo cuenta con dos actuadores para la tracción y la dirección respectivamente. En el caso de la tracción, se 
    utiliza un motor de corriente contínua de imán permanente en conjunción con un controlador de potencia capaz de manipular 
    al mismo. El control del motor se basa en la combinación de dos pines de dirección y un pin correspondiente con la velocidad 
    que es controlada mediante una señal modulada en ancho de pulso o PWM. 

    Por su parte, el control de la dirección se realiza mediante el uso de un servomotor dimensionado adecuadamente para la aplicación 
    que controla el ángulo de las ruedas delanteras. El ángulo de las ruedas define el ICC (Sección(\ref{sec:ruedas})), que a su vez 
    define el radio de giro mínimo en base al máximo rango en el que puede variar el ángulo de las ruedas delanteras. En la 
    Figura() se puede apreciar el prototipo y la disposición de los actuadores del mismo.

    \begin{quote}
        Foto del carro sin control
    \end{quote}

        \subsubsection{Control de actuadores mediante ROS}
        Tal como se pudo especificar en la Sección() el microcontrolador elegido para la tarea tiene la capacidad de usar la 
        librería Rosserial que hace posible crear un nodo de ROS en el microcontrolador usando el puerto serial. En este sentido, 
        el control de motores se realiza de manera asíncrona con un tiempo de muestreo fijo. Se puede apreciar el algoritmo 
        implementado en el nodo de control en el Algoritmo(\ref{alg:nodomcu}).

        \begin{algorithm}
            \begin{algorithmic}[1]
            %\REQUIRE Complejo simplicial $K=\{\sigma_1, \dots, \sigma_n \}$ no vacío. \label{lin:lineaRara}
            %\ENSURE \TRUE si $K$ es contráctil y \FALSE en caso contrario.
            \STATE Conexión con el ros master
            \WHILE {No hay conexión}
            \STATE Intentar conexión
            \ENDWHILE
            
            \WHILE {Conectado con el ros master}
                \IF{Hay mensaje de control pendiente}
                    \STATE Calcular señal de control
                    \STATE Enviar comando a los actuadores
                \ENDIF
            \ENDWHILE
            \end{algorithmic}
            \caption{Algoritmo de control de actuadores}\label{alg:nodomcu}
        \end{algorithm}
        
        En este caso, el microcontrolador entra en un bucle que se ejecuta mientras exista conexión con el maestro. Dentro del 
        bucle, espera la llegada de algún mensaje de control para realizar un pequeño cálculo dependiendo a la calibración de 
        los mismos con el fin de enviar una señal de control adecuada para la actuación del vehículo.

        En la Figura() se puede apreciar el diagrama de comunicación entre los nodos de ros correspondientes con el nodo serial 
        y un control manual con un joystick.

        %%% figura del nodo rosserial
    
    \subsection{Módulo de la computadora de abordo}
    Mientras que el módulo de control de tiempo real se encarga de traducir los mensajes de control de ROS en señales de control 
    para aplicarse directamente a los actuadores por el microcontrolador a través de un controlador de potencia, la computadora 
    de abordo u OBC, se encarga se controlar todos los aspectos de \textit{alto nivel} del vehículo. Este módulo se compone de 
    una SBC corriendo un sistema operativo Linux con una distribución de ROS instalada en ella. La OBC será la responsable de 
    la comunicación con todos los nodos que interactúan con el vehículo, tanto  de los nodos que generan datos del entorno como 
    las imágenes provenientes de la cámara o sensores; como con nodos que consumen estos datos y los procesan de alguna manera, 
    como el nodo del piloto automático. 

    En la Figura() se puede apreciar el prototipo con la OBC incorporada al mismo. Cabe resaltar que por sus características, tanto 
    la OBC, como el módulo de control de tiempo real pueden alimentarse por una batería a bordo del vehículo. A su vez, la comunicación 
    inalámbrica de la OBC permite que el vehículo pueda funcionar sin la necesidad de un solo cable ya sea de alimentación de energía 
    o comunicación.


    \subsubsection{Esquema de comunicación en la OBC}
    La OBC corre el maestro de ROS, que es el que arbitra toda la comunicación en el sistema y, junto con eso, también se encarga 
    de ejecutar varios nodos correspondientes con los sensores y el control. La lista detallada de los nodos que corren en 
    la OBC se puede observar en la Tabla(). Por su parte, el esquema de comunicación entre dichos nodos, se puede apreciar en 
    la Figura().

    %% tabla de nodos 

    %% diagrama de nodos

    Dado que ROS permite el desarrollo de sistemas distribuidos, existen otros nodos que se ejecutan en la estación de trabajo y 
    se comunican con el maestro, corriendo en la OBC, a través de Wifi mediante una conexión TCP. La implementación de la comunicación 
    en la red está implementada en el \textit{middleware} de ROS y este proyecto no se concentra en los detalles de la misma.

\section{Subsistema de Adquisición de Datos y Entrenamiento}

Una vez establecida la plataforma de trabajo en el Subsistema de Control y Actuación se procede a detallar el diseño del 
Subsistema de Adquisición de Datos y Entrenamiento que se encarga principalmente de brindar una forma adecuada de sincronizar 
y almacenar los datos necesarios para el entrenamiento de la red neuronal así como también de brindar las herramientas necesarias 
para el entrenamiento en sí. Este subsistema cuenta con vários módulos funcionales que interactúan entre sí como se pudo apreciar 
en la Sección(). 


    \subsection{Módulo de adquisición de datos y operación manual}
    Como se ha podido establecer en la Sección() referida al aprendizaje automático. Para que un algoritmo de aprendizaje pueda 
    entrenarse de manera efectiva, es necesario contar con un conjunto de datos sobre el cual se realizará el mismo. Este conjunto 
    de datos o \textit{dataset} necesita estar almacenado de manera adecuada con el formato y las características necesarias para 
    un entrenamiento efectivo.

    El módulo cuenta con un nodo que se suscribe a los mensajes provenientes tanto de la cámara como del control manual con el 
    joystick (Figura()). Luego, se realiza una sincronización para que se pueda generar un par entrada - salida adecuado y luego se procede 
    a almacenar el mismo. En la Figura() se puede apreciar un diagrama donde se detalla el procedimiento de adquisición de datos
    y almacenamiento.

    % foto del joystick

    % diagrama del modulo
    El \textit{dataset} se compone de un conjunto de imágenes en formato \lstinline{PNG} junto con un archivo \lstinline{CSV} que 
    contiene las columnas detalladas en la Tabla(). En la tabla se almacenan tanto los datos de aceleración y dirección como el 
    nombre del archivo de la imagen correspondiente con la misma. Los registros se almacenan en el mismo orden en el que fueron 
    capturados, no obstante, para la tarea de regresión definida en este proyecto, la dependencia temporal entre las muestras 
    no es importante. 

        \subsubsection{Sesiones de entrenamiento}
        La manera de adquirir el conjunto de datos es mediante las denominadas \textit{Sesiones de Entrenamiento o Grabación de Comportamiento}.
        Estas sesiones corresponden con periodos de tiempo en el cual se recuperan datos de manera constante. La cantidad de registros 
        o muestras recuperadas depende tanto de la resolución temporal de la generación de imágenes y comandos de control sincronizados 
        así como también de la duración de la sesión en sí. Cada sesión de entrenamiento puede generar un \textit{dataset} distinto 
        que se puede combinar con otros \textit{datasets} o tomarse de manera independiente. 

        Es importante tomar en cuenta que mientras más sesiones se realicen y más variadas en condiciones ambientales sean las mismas, 
        más robusto será el algoritmo entrenado y mejor será su generalización. Sin embargo, esta capacidad de generalización estará 
        limitada por la arquitectura y la cantidad de parámetros disponibles para generar representaciones internas del proceso. 

        En el presente proyecto, se han realizado sesiones de entrenamiento con un clima parcialmente nublado con buena iluminación 
        ambiental, de manera que se pueda distinguir la pista del piso (Figura()).

        %% foto de ejemplo

    \subsection{Módulo de aumentación de datos}
    Las redes neuronal profundas tienen la impresionante capacidad de aprender representaciones internas útiles de manera automática 
    usando el algoritmo de la retropropagación, pero esta capacidad solamente resalta cuando se cuenta con un conjunto de datos 
    de un tamaño considerable, es decir, que se necesitan muchas muestras para que el entrenamiento de una red neuronal profunda 
    sea exitosa y pueda mostrar resultados favorables. 
    
    Por su parte, la recuperación de datos es un proceso costoso en tiempo y recursos, por lo que no se puede invertir demasiado 
    tiempo solamente recuperando muestras. De hecho, se ha establecido que el proceso de la generación de un \textit{dataset} efectivo 
    es el más complicado en un sistema de aprendizaje automático. Por tanto, es importante poder aprovechar al máximo los datos 
    obtenidos en una sesión de entrenamiento.

    Es por este motivo que se ha decidido incluir una etapa de aumentación de datos que se compone de una serie de transformaciones 
    de las imágenes originales del \textit{dataset} para poder incrementar la cantidad de muestras sobre la cual se realice el 
    entrenamiento de la red de manera sintética. En la Tabla() se puede apreciar las transformaciones usadas.
    
    % tabla de transformaciones 

    Este módulo no solamente cuenta con ciertas transformaciones disponibles para aumentar los datos existentes en el \textit{dataset}
    si no también cuenta con la funcion de poder eliminar registros con un valor de dirección nulo si es que fuera necesario. La utilidad
    de esta funcionalidad se puede apreciar en la etapa de entrenamiento en la Sección().
    
    \subsection{Módulo de generación de datos de entrenamiento}

    Una de las desventajas de las redes neuronales convolucionales para el procesamiento de imágenes es que las imágenes 
    ocupan bastante espacio en memoria y cargar todo el conjunto de datos de entrenamiento en la memoria RAM es usualmente 
    imposible o demasiado costoso. Es por eso que se necesita una forma de cargar las imágenes en la memoria de forma 
    fraccionada, de manera que los requisitos de memoria sean manejables por la estación de trabajo. 

    El módulo de generación de datos de entrenamiento se ha desarrollado con la finalidad de poder generar \textit{mini batches} 
    o conjuntos pequeños de datos para el entrenamiento de la red neuronal convolucional. El algoritmo de la generación se 
    detalla en el Algoritmo().

    Este módulo es capaz de generar \textit{mini batches} de manera parametrizada de acuerdo a la necesidad de cada sesión de 
    entrenamiento.

    \subsection{Módulo de Entrenamiento}
    Una vez se tienen las herramientas para la correcta generación y aumentación de datos del conjunto de datos de entrenamiento 
    se necesita una herramienta para ejecutar el entrenamiento de la red neuronal convolucional en sí. El módulo de entrenamiento 
    se compone de un programa que ejecuta las siguientes tareas:

    \begin{itemize}
        \item Carga del dataset y el modelo de red neuronal.
        \item Definición de hiperparámetros.
        \item Ejecución, monitoreo y control del proceso de entrenamiento en línea. 
        \item Salvaguarda de los parámetros del modelo entrenado.
        \item Generación de reportes del proceso de entrenamiento.
    \end{itemize}

    En el Algoritmo() se puede apreciar el proceso detallado de entrenamiento de la red de acuerdo a parametrización definida 
    por el usuario. 
    
    % algoritmo del entrenamiento

    Este módulo se ejecuta en la estación de trabajo y hace uso de la GPU disponible para la paralelización de los algoritmos de 
    cálculo de gradientes y optimización con el fin de acelerar el tiempo de entrenamiento. El módulo es capaz de realizar 
    el entrenamiento de distintos modelos de arquitectura de redes neuronales definidas por el usuario en un archivo de código 
    fuente, en otras palabras, se puede utilizar el mismo programa para entrenar múltiples redes neuronales sin realizar grandes 
    cambios en el código fuente. También se puede definir los directorios donde están almacenados los datos de entrenamiento y 
    el destino de los reportes del entrenamiento, esto es útil para cuando se necesita validar el entrenamiento de múltiples 
    modelos con múltiples conjuntos de datos o para separar las sesiones de entrenamiento de distintos sistemas o prototipos.

    Los productos obtenidos por este módulo son dos:
    
    \begin{itemize}
        \item \textbf{Arquitectura de la red neuronal:} Se almacena la información acerca de la arquitectura de la red en un archivo con formato \lstinline{JSON} donde se puede encontrar la información acerca de las dimensiones de las capas ocultas, cantidad de unidades por capa y dimensiones de cada unidad en la red.
        \item \textbf{Pesos entrenados de la red neuronal:} Se almacena también los valores de todos los parámetros o pesos de la red neuronal, resultado del entrenamiento. Estos valores están almacenados en un archivo con extensión \lstinline{H5}. 
    \end{itemize}

    Con estos dos archivos será posible ejecutar la tarea de inferencia en etapas posteriores. Además, es importante mencionar que 
    los mismos archivos pueden usarse para re-entrenar la red con un nuevo conjunto de datos para mejorar su rendimiento.

    Como se ha podido observar, los módulos que componen el Subsistema de Adquisición de Datos y Entrenamiento se han diseñado 
    con el fin de poderse utilizar de manera individual con otros sistemas o de manera conjunta, enfocando las funcionalidades 
    en el entrenamiento de redes neuronales convolucionales para tareas de procesamiento de imágenes y visión artificial.

\section{Subsistema de Inferencia y control autónomo}
El subsistema de inferencia y control autónomo se ha diseñado tomando en cuenta la modularidad necesaria para poder ser 
extendible en funcionalidades de 
    \subsection{Módulo de inferencia con una red neuronal convolucional}
    Este módulo tiene la tarea de ejecutar la tarea de predicción del comando de control de dirección con la red neuronal entrenada 
    por el módulo de entrenamiento del subsistema de adquisición de datos y entrenamiento.
    \subsection{Módulo de detección de obstáculos}
    \subsection{Módulo del piloto automático}

\section{Diseño de la arquitectura de la red neuronal}
-
    \subsection{Consideraciones y requerimientos}
    \subsection{Unidades y profundidad}
    \subsection{Funciones de Activación}
    \subsection{Función de Costo}
    \subsection{Optimizador}
    
\section{Proceso de Entrenamiento de la \\ Red neuronal Convolucional}
-
    \subsection{Sistemas de Imitación de comportamiento}
    \subsection{Evaluación del conjuto de datos}
    \subsection{Curvas de Aprendizaje}

\section{Implementación del prototipo}
- 
    \subsection{Descripción general del prototipo}
    \subsection{Restricciones de diseño}
    \subsection{Modos de funcionamiento}
        \subsubsection{Modo de entrenamiento y adquisición de datos}
        \subsubsection{Modo de conducción autónoma}
        

\chapter{Análisis y discusión de resultados}
\label{ch:resultados}

\section{Pruebas}
\section{Resultados generales}
\section{Proceso de entrenamiento}
    \subsection{Análisis de las curvas de entrenamiento}
    \subsection{Análisis del rendimiento en conjuntos de prueba y validación cruzada}
\section{Despliegue del prototipo}

\chapter{Conclusiones y recomendaciones}
\label{ch:conclusiones}
\section{Conclusiones}
Se detallan las conclusiones correspondientes con cada objetivo específico planteado:

\begin{itemize}

    \item \textit{Estudiar los aspectos concernientes al desarrollo de sistemas de conducción autónoma y sistemas de aprendizaje.}
    
    \textbf{Conclusión:} Se estudiaron los aspectos relacionados al desarrollo de sistemas de conducción autónoma, en principio, 
    analizando su origen y potencialidad en la Sección(\ref{sec:antecedentes}). Para posteriormente estudiar los conceptos básicos 
    de aprendizaje 
    automático, aprendizaje profundo, redes convolucionales y el proceso de entrenamiento de una red neuronal.

    \item \textit{Analizar los requerimientos de un sistema de conducción autónoma capaz identificar y mantener su carril 
    mientras se conduce.}

    \textbf{Conclusión:} Se ha introducido el esquema de la arquitectura de un sistema de conducción autónomo y se han planteado 
    los requisitos y funcionalidades que debe tener el sistema en su conjunto en la Sección(\ref{sec:arquitectura}) y de las 
    características de los 
    subsistemas de los que se compone: el subsistema de control y actuación en la Sección(\ref{sec:esqcontrol}), el subsistema de adquisición de 
    datos y entrenamiento en la Sección(\ref{sec:esqdaq}), el subsistema de inferencia y control autónomo en la Sección(\ref{sec:esqinferencia}). 

    \item \textit{Diseñar la arquitectura de un sistema de conducción autónoma en base a los requerimientos previamente 
    establecidos.}

    \textbf{Conclusión:} Se ha planteado la arquitectura general del sistema en base a las características analizadas en 
    el Capítulo(\ref{ch:introduccion}) y (\ref{ch:m_teorico}) en la Sección(\ref{sec:arquitectura}). Considerando todos los alcances
    y limitaciones planteados por el proyecto en el inicio. 
    Por su parte, también se han definido las herramientas de hardware en la Sección(\ref{sec:software}) y las herramientas 
    de software en 
    la Sección(\ref{sec:software}) que hicieron posible la implementación de todos los subsistemas asociados.

    \item \textit{Diseñar el subsistema de adquisición de datos y entrenamiento para tareas de conducción autónoma.}
    
    \textbf{Conclusión:} Se ha diseñado el Subsistema de Adquisición de Datos y Entrenamiento en base a la arquitectura planteada 
    en la Sección(\ref{sec:esqdaq}) en la Sección(\ref{sec:daq}) considerando los detalles de implementación, algoritmos y características necesarias para 
    las tareas de adquisición de datos, aumentación de datos y entrenamiento de modelos de redes neuronales explorados en el Capítulo(\ref{ch:ingenieria}).
    Los detalles del proceso de diseño y entrenamiento de las arquitecturas de redes neuronales planteadas se exploran con detalle en la 
    Sección(\ref{sec:design}) para el diseño de la red neuronal y en la Sección(\ref{sec:training}) para el entrenamiento. 

    \item \textit{Diseñar el subsistema de control y actuación para la conducción autónoma de un vehículo con características 
    similares a las de un vehículo doméstico real.}

    \textbf{Conclusión:} Se ha diseñado el Subsistema de Control y Actuación de acuerdo con la arquitectura planteada en la 
    Sección(\ref{sec:arquitectura}) en la Sección(\ref{sec:control}). Se ha considerado las características y requerimientos 
    planteados para dicho subsistema en 
    las tareas de control de tiempo real implementado en un microcontrolador, interfaz con los actuadores y sensores 
    sobre la plataforma de comunicación de ROS. Se han expuesto los detalles de implementación tanto a nivel de hardware 
    como de software procurando que el diseño de este subsistema sea modular.

    \item \textit{Diseñar el subsistema de inferencia y control autónomo basado en el uso de redes neuronales convolucionales.}
    
    \textbf{Conclusión:} Por su parte, se ha diseñado a detalle el Inferencia y Control autónomo en base a todos los fundamentos 
    teóricos de redes neuronales exploradas en la Sección(\ref{sec:sec:esqinferencia}). Los detalles de la implementación se pueden 
    encontrar en la Sección(\ref{sec:inferencia})
    donde se abunda en la implementación de los módulos que componen este subsistema para las tareas de predicción de dirección
    con la red neuronal convolucional, detección de obstáculos con un sensor de proximidad y el algoritmo del piloto automático 
    implementados como nodos de ROS para garantizar la modularidad del sistema.

    \item \textit{Analizar los resultados del entrenamiento e implementación del subsistema de inferencia y control autónomo.}
    
    \textbf{Conclusión:} El análisis de los resultados del entrenamiento de la red neuronal convolucional se ha explorado en 
    la Sección(\ref{sec:analisistrain}) pudiendo hacer la comparación de resultados en errores de entrenamiento, validación 
    y prueba de dos arquitecturas:
    una con una red neuronal tradicional o densamente conectada, y otra con una red neuronal convolucional. En base a los 
    resultados y puntajes en el entrenamiento obtenidos se puede concluir que una red neuronal convolucional es capaz de cumplir la 
    tarea de generar comandos de control para la tarea de conducción autónoma. 

    Por otro lado, la implementación de la red y los resultados en pruebas de campo se han explorado en la 
    Sección(\ref{sec:analisistest}) en la cual 
    se pudo observar la efectividad de la red neuronal convolucional para generar representaciones internas relevantes y útiles. 
    En base a este análisis se puede concluir que, efectivamente, una la red convolucional entrenada es capaz de generar 
    representaciones internas relevantes en muestras nunca antes vistas, en otras palabras, que la capacidad de generalización 
    es aceptable.

    \item \textit{Realizar pruebas de rendimiento y análisis comparativos en el sistema implementado.}
    
    \textbf{Conclusión:} Con el fin de realizar un análisis comparativo entre distintas implementaciones de redes neuronales 
    para la tarea especificada por este proyecto, se ha diseñado y entrenado un par de arquitecturas de red neuronal sobre las 
    cuales se realiza un análisis comparativo de rendimiento en base a indicadores y puntajes estándar en el campo de la estadística
    en la Sección(\ref{sec:scores}). 

    Por su parte, también se ha analizado las predicciones generadas por la red convolucional, para distintos casos, explorando 
    la naturaleza de las representaciones generadas por la misma en la Sección(\ref{sec:representaciones}). Por tanto, se puede 
    concluir que la implementación 
    de la red neuronal convolucional planteada se ha realizado exitosamente.

\end{itemize}

\section{Recomendaciones}

Luego de haber analizado y generado conclusiones relativas a los resultados obtenidos en base a los objetivos planteados en 
la etapa inicial del presente proyecto, se plantea una serie de recomendaciones:

\begin{itemize}
    \item Debido a la capacidad de generalización que puede lograrse con una red neurona convolucional se recomienda 
    generar distintos conjuntos de entrenamiento en condiciones climáticas diversas. Esto incrementará la complejidad 
    de la red convolucional y logrará que las predicciones de la red sean más robustas en cambios de iluminación 
    causados por distintos aspectos climáticos.
    \item Por su parte, gracias a la naturaleza modular del presente proyecto, se recomienda diseñar, entrenar y 
    validar arquitecturas de redes neuronales distintas o con variaciones a una red neuronal convolucional 
    tradicional. Se recomienda, por ejemplo, implementar una red convolucional recurrente que sea capaz de tomar 
    en cuenta el estado anterior de la misma. Implementar otros algoritmos de predicción y visión artificial también 
    ayudarían al desarrollo de sistemas más robustos.
    \item Se recomienda también extender la implementación de la red a una tarea de clasificación categórica para 
    comparar el rendimiento entre el entrenamiento y rendimiento obtenidos en el problema planteado como una regresión 
    y una clasificación.
    \item Debido ala modularidad del sistema planteando en el presente proyecto, se recomienda implementar el mismo 
    sistema fin a fin para otros modelos de vehículos, como pueden ser un robot de tracción diferencial u otro modelo.
    \item Es también importante poder extender el sistema presentado en este proyecto con tareas de control y detección 
    avanzadas, incluyendo en el flujo de trabajo módulos de planificación de trayectorias, detección de objetos y 
    programación de misiones.
    \item Dada la gran potencialidad en el área de sistemas de conducción autónoma se recomienda que se pueda crear 
    una línea de investigación dedicada al diseño e implementación de sistemas robóticos inteligentes en la cual 
    se agrupen esfuerzos para el desarrollo de cada uno de los subsistemas que componen un vehículo autónomo. 
    \item Por último, la forma en la que ha sido implementado el presente proyecto permite modificarlo y extenderlo 
    de distintas maneras y a distintos niveles. Se recomienda considerar este proyecto como una plataforma de 
    desarrollo sobre la cual se puedan implementar diversos algoritmos y aplicaciones útiles para el desarrollo de 
    nuestra sociedad.

\end{itemize}


% \lhead[\thepage]{REFERENCIAS}
\rhead[REFERENCIAS]{\thepage}

\cleardoublepage
\addcontentsline{toc}{chapter}{Bibliografía}
\bibliographystyle{IEEEtran}
\bibliography{referencias}
\cleardoublepage
% \chapter{Bibliografía}
\addcontentsline{toc}{chapter}{Bibliografía}
\bibliographystyle{IEEEtran}

\bibliography{referencias}


\appendix
\addcontentsline{toc}{chapter}{Apéndice} 
% \appendix
\chapter{Función sigmoide}\label{sec:sigmoide} %% TODO: pasar a anexos
            Esta función ha sido introducida en la Ecuación(\ref{eq:sigmoide}) y representa, históricamente, la función 
            más utilizada en las primeras redes neuronales artificiales porque modela de manera aproximada, una respuesta 
            característica de las neuronas del cerebro \cite{narayan1997generalized}, pero sobre todo, porque posee las dos características mencionadas 
            anteriormente: naturaleza no lineal y diferenciable. Otra característica llamativa, es que se puede interpretar 
            a la salida como una función de probabilidad, pues sus valores van desde $0$ a $1$, como se puede apreciar 
            en la Figura(\ref{fig:sigmoide}). Usualmente se incluye un parámetro adicional para controlar el \textit{radio de activación}
            que hace que la función responda con más o menos sensibilidad a su entrada.

            \begin{figure}[!h] 
                \centering
                \includegraphics[width=0.75\textwidth]{img/sigmoide}
                \caption[Gráfico de la función sigmoide]{Gráfico de la función sigmoide $\sigma(v) = 1 / (1 + e^{(-v)})$ (curva roja), y dos variaciones con un radio de activación de la forma $\sigma(sv)$ con valores $s=1/2$ (curva azul) y $s=10$ (curva púrpura) . Fuente: \cite{Goodfellow-et-al-2016} }
                \label{fig:sigmoide}
            \end{figure}
            
            Pese a las características anteriormente mencionadas, la función sigmoide tiene una gran desventaja: el llamado 
            \textit{desvanecimiendo de gradientes} \cite{hochreiter1998vanishing}. Este fenómeno ocurre cuando la activación 
            de una capa oculta tiene valores muy altos o muy negativos, se puede apreciar en la Figura(\ref{fig:sigmoide}), que 
            para estos valores, la derivada tiene un valor muy pequeño, aproximándose a cero mientras más grandes sean los valores.
            Este fenómeno ocasiona que, mientras se realiza la retropropagación de gradientes, dado que el gradiente tiene un valor 
            muy bajo, el ajuste en los pesos sea mínimo, deteniendo así el aprendizaje de la neurona en la que ocurre el fenómeno.

            Esta dificultad se presenta especialmente cuando la red neuronal se compone de varias capas ocultas y 
            ha motivado el desarrollo de nuevas funciones de activación que no presenten esta limitación y puedan 
            mantener valores de gradientes adecuados durante el entrenamiento.
            
            Cabe resaltar que, pese a las dificultades con la propagación de los gradientes de la función sigmoide, ésta se suele 
            utilizar bastante en la capa de salida, cuando se trata de clasificación binaria, ya que representa de manera adecuada
            la noción de probabilidad, lo cual es deseable en este tipo de modelos.
% \appendix
\chapter{Código Fuente del Proyecto}\label{apx:source}
\section{Control de Actuadores en Tiempo Real}
\begin{lstlisting}[title={main.cpp},language=c++]
#include <mbed.h>
#include <Servo.h>
#include <Motor.h>
#include <ros.h>

// ros msgs
#include <geometry_msgs/Twist.h> // sub to Twist message for control
#include <sensor_msgs/Range.h>   // pub to Range message for sensor

DigitalOut led(D13);

Servo steering(D9);            // steering servo
Motor motor(D12, D10, D11); // motor driver

// ROS part
// software utils
Timer t;
Ticker pub_ticker;
// ros part
ros::NodeHandle nh;

// messages
geometry_msgs::Twist control_cmd; // control command is of type twist

// callback prototypes
void ctrlCommandCb(const geometry_msgs::Twist &command);

// subscribers
// ros subscriber for control command
ros::Subscriber<geometry_msgs::Twist> ctrlSub("cmd_vel", ctrlCommandCb);

// prototypes 
// callbacks
void ctrlCommandCb(const geometry_msgs::Twist &command);

int main() {
    
    motor.period(0.0005);     // frecuencia 2 Khz
    steering = 0;
    // init sensors
    led = 1;
    // ros initialization
    nh.initNode();
    nh.subscribe(ctrlSub);
    //
    led = 0;

    while(1) {
        nh.spinOnce();
        wait_ms(1);
    }
}

// control command callback
void ctrlCommandCb(const geometry_msgs::Twist &command)
{
    led = !led;
    float linear = command.linear.x;
    float angular = command.angular.z / 2 + 0.5; // (command.angular.z * 0.35) + 0.32;
    steering = angular;
    // testigo = abs(linear * 8);
    motor.speed(linear);
}
\end{lstlisting}

\section{Script de entrenamiento}
\begin{lstlisting}[title={train\_model.py},language=Python]
from keras.preprocessing.image import load_img
from keras.preprocessing.image import img_to_array

from keras.models import model_from_json
    
import numpy as np
import pandas as pd
import bcolz
import threading
    
from time import time
import os
import sys
import glob
import shutil

from sklearn.model_selection import train_test_split
    
from keras.models import Sequential
from keras.layers import Dense, Dropout, Flatten
from keras.layers import Conv2D, MaxPooling2D
from keras import backend as K
from keras.callbacks import TensorBoard, ModelCheckpoint
    
from keras.utils import plot_model
import models
from utils import *
    
class RobocarTrainer(object):
        
    # data for training
    input_shape=(224,224,3)
    im_shape = (224, 224)
    
    # train parameters
    batch_size = 32
    n_epochs = 100
    
    def __init__(self, model_name, model_path, dataset_path, log_path='trainlogs'):
        self.model_name = model_name
        self.model_path = model_path
        self.log_path = log_path
        self.dataset_path = dataset_path
            
        # creating callbacks
        self.tfBoardCB = TensorBoard('{}/{}_{}'.format(self.log_path, model_name, time()), write_graph=True)
    
        filepath= model_path + model_name + '_best.h5'
    
        self.checkpointCB = ModelCheckpoint(filepath, monitor='val_loss', verbose=1, save_best_only=True, mode='min')
    
        
    def LoadDataset(self):
    
        print('loading dataset...')
        self.dataset = pd.read_csv(self.dataset_path + 'target.csv')
    
        self.dataset['imgpath'] = self.dataset.id.apply(file_path_from_db_id, args=("%d.bmp", self.dataset_path))
    
        self.train, self.test = train_test_split(self.dataset, test_size=0.2)
        self.valid, self.test = train_test_split(self.test, test_size=0.7)
        
        self.train_steps = int(self.train.shape[0] / self.batch_size)
        self.valid_steps = int(self.valid.shape[0] / self.batch_size)
        self.test_steps = int(self.test.shape[0] / self.batch_size)
        print('dataset loaded!')
    
    def Train(self):
    
        print('loading model...')
        self.model = models.vanilla(self.input_shape)
        self.model.summary()
        model_json = self.model.to_json()
        with open(self.model_path + self.model_name + '.json', "w") as json_file:
                json_file.write(model_json)
                
        plot_model(self.model, to_file=self.model_name + '.png', show_shapes=True)
        print('dataset size: ', self.train.shape[0],
                    'train_steps: ', self.train_steps, 
                    'valid steps: ', self.valid_steps, 
                    'test_steps: ', self.test_steps)
            
        print('hiperparameters:')
        print('batch size:{}'.format(self.batch_size))
            
    
        print('training...')
        self.model.fit_generator(
                                generator_from_df(self.train, self.batch_size, self.im_shape, 'angular'),
                                steps_per_epoch=self.train_steps, 
                                epochs=self.n_epochs,
                                validation_data=generator_from_df(self.valid, self.batch_size, self.im_shape, 'angular'),
                                validation_steps=self.valid_steps,
                                callbacks=[self.tfBoardCB, self.checkpointCB],
                                verbose=2
                                )
        print('finished training')
        self.score = self.model.evaluate_generator(
                                generator_from_df(self.test, self.batch_size, self.im_shape, 'angular'), 
                                steps=self.test_steps
                                )
        print('loss: ', self.score)
    
    def SaveModel(self):
        # serialize model to JSON
        self.model_json = self.model.to_json()
        with open(self.model_name + '.json', "w") as json_file:
                json_file.write(self.model_json)
        # serialize weights to HDF5
        self.model.save_weights(self.model_name + '.h5')
        print("Saved model to disk")
    
    
    
    
if __name__ == '__main__':
        
    import argparse
    parser = argparse.ArgumentParser(description='Entrenamiento de una red neuronal convolucional')
    parser.add_argument("model_name", help="name of the model to train")
    parser.add_argument("model_path", help="path where the model files will be saved")
    parser.add_argument("dataset_path", help="path where the the dataset is saved")
        
    args = parser.parse_args()
        
    trainer = RobocarTrainer(args.model_name, args.model_path, args.dataset_path)
    trainer.LoadDataset()
    trainer.Train()
    trainer.SaveModel()

\end{lstlisting}

\section{Sincronización de Mensajes}

\begin{lstlisting}[title={msg\_sync.py},language=Python]
    from keras.preprocessing.image import load_img
    from keras.preprocessing.image import img_to_array
    
    from keras.models import model_from_json
        
    import numpy as np
    import pandas as pd
    import bcolz
    import threading
        
    from time import time
    import os
    import sys
    import glob
    import shutil
    
    from sklearn.model_selection import train_test_split
        
    from keras.models import Sequential
    from keras.layers import Dense, Dropout, Flatten
    from keras.layers import Conv2D, MaxPooling2D
    from keras import backend as K
    from keras.callbacks import TensorBoard, ModelCheckpoint
        
    from keras.utils import plot_model
    import models
    from utils import *
        
    class RobocarTrainer(object):
            
        # data for training
        input_shape=(224,224,3)
        im_shape = (224, 224)
        
        # train parameters
        batch_size = 32
        n_epochs = 100
        
        def __init__(self, model_name, model_path, dataset_path, log_path='trainlogs'):
            self.model_name = model_name
            self.model_path = model_path
            self.log_path = log_path
            self.dataset_path = dataset_path
                
            # creating callbacks
            self.tfBoardCB = TensorBoard('{}/{}_{}'.format(self.log_path, model_name, time()), write_graph=True)
        
            filepath= model_path + model_name + '_best.h5'
        
            self.checkpointCB = ModelCheckpoint(filepath, monitor='val_loss', verbose=1, save_best_only=True, mode='min')
        
            
        def LoadDataset(self):
        
            print('loading dataset...')
            self.dataset = pd.read_csv(self.dataset_path + 'target.csv')
        
            self.dataset['imgpath'] = self.dataset.id.apply(file_path_from_db_id, args=("%d.bmp", self.dataset_path))
        
            self.train, self.test = train_test_split(self.dataset, test_size=0.2)
            self.valid, self.test = train_test_split(self.test, test_size=0.7)
            
            self.train_steps = int(self.train.shape[0] / self.batch_size)
            self.valid_steps = int(self.valid.shape[0] / self.batch_size)
            self.test_steps = int(self.test.shape[0] / self.batch_size)
            print('dataset loaded!')
        
        def Train(self):
        
            print('loading model...')
            self.model = models.vanilla(self.input_shape)
            self.model.summary()
            model_json = self.model.to_json()
            with open(self.model_path + self.model_name + '.json', "w") as json_file:
                    json_file.write(model_json)
                    
            plot_model(self.model, to_file=self.model_name + '.png', show_shapes=True)
            print('dataset size: ', self.train.shape[0],
                        'train_steps: ', self.train_steps, 
                        'valid steps: ', self.valid_steps, 
                        'test_steps: ', self.test_steps)
                
            print('hiperparameters:')
            print('batch size:{}'.format(self.batch_size))
                
        
            print('training...')
            self.model.fit_generator(
                                    generator_from_df(self.train, self.batch_size, self.im_shape, 'angular'),
                                    steps_per_epoch=self.train_steps, 
                                    epochs=self.n_epochs,
                                    validation_data=generator_from_df(self.valid, self.batch_size, self.im_shape, 'angular'),
                                    validation_steps=self.valid_steps,
                                    callbacks=[self.tfBoardCB, self.checkpointCB],
                                    verbose=2
                                    )
            print('finished training')
            self.score = self.model.evaluate_generator(
                                    generator_from_df(self.test, self.batch_size, self.im_shape, 'angular'), 
                                    steps=self.test_steps
                                    )
            print('loss: ', self.score)
        
        def SaveModel(self):
            # serialize model to JSON
            self.model_json = self.model.to_json()
            with open(self.model_name + '.json', "w") as json_file:
                    json_file.write(self.model_json)
            # serialize weights to HDF5
            self.model.save_weights(self.model_name + '.h5')
            print("Saved model to disk")
        
        
        
        
    if __name__ == '__main__':
            
        import argparse
        parser = argparse.ArgumentParser(description='Entrenamiento de una red neuronal convolucional')
        parser.add_argument("model_name", help="name of the model to train")
        parser.add_argument("model_path", help="path where the model files will be saved")
        parser.add_argument("dataset_path", help="path where the the dataset is saved")
            
        args = parser.parse_args()
            
        trainer = RobocarTrainer(args.model_name, args.model_path, args.dataset_path)
        trainer.LoadDataset()
        trainer.Train()
        trainer.SaveModel()
    
    \end{lstlisting}

\section{Aumentación de datos}

\begin{lstlisting}[title={augmentation.py},language=Python]
    from keras.preprocessing.image import load_img, ImageDataGenerator
    from keras.preprocessing.image import img_to_array
    
    from pandas.plotting import bootstrap_plot
    import numpy as np
    import pandas as pd
    import bcolz
    import threading
    
    import cv2
    
    from utils import *
    
    import os
    import sys
    import glob
    import shutil
    import matplotlib.pyplot as plt
    get_ipython().magic(u'matplotlib inline')
    
    dataset1 = pd.read_csv('../../datasets/dataset/target.csv')
    dataset1['imgpath'] = dataset1.id.apply(file_path_from_db_id)
    
    datagen = ImageDataGenerator(
            rotation_range=20,
            height_shift_range=0.2,
            shear_range=0.15,
            zoom_range=0.15,
            fill_mode='nearest')
    
    test_gen = batch_generator(dataset1, 1, (224,224), 'angular', process=False, shuffle=False)
    img, tg = test_gen.next()
    img = img[0]
    img = (img)*255
    plt.subplot(131)
    plt.imshow(img)
    
    for i, row in dataset1.sample(1).iterrows():
        img3 = img_to_array(load_img(row['imgpath']))
    
    plt.subplot(132)
    plt.imshow(img3*255)
    
    img2, tg2 = horizontal_flip(img, tg)
    plt.subplot(133)
    plt.imshow(img2)
    
    datagen = ImageDataGenerator(
            rotation_range=20,
            height_shift_range=0.2,
            shear_range=0.15,
            zoom_range=0.15,
            fill_mode='nearest')
    
    test_gen = batch_generator(dataset1, 1, (224,224), 'angular', process=False, shuffle=False)
    
    offset = dataset1.shape[0] + 1
    id_offset = dataset1['id'].max() + 1
    i = 0
    newData = []
    for idx, row in dataset1.iterrows():
        # generador auxiliar
        
        ### extraemos la imagen en una variable auxiliar
        img = img_to_array(load_img(row['imgpath']))
        target = row['angular']
        ### aplicamos la transformacion de acuerdo a ciertas condiciones
        ## si es cero, o muy cercano a 0 no aplicamos la transformacion
        if np.abs(target) > 0.09:
            ## transformacion horizontal
            hImg, hTarget = horizontal_flip(img, target) 
            # guardar imagen y entry en el dataframe
            filename = "dataset/" + str(id_offset + i) + ".bmp";
            hImg = cv2.cvtColor(hImg,cv2.COLOR_BGR2RGB)
            cv2.imwrite(filename, hImg)
            newData.append([(id_offset + i), 
                            row['linear'], 
                            hTarget,
                            filename])
            # incrementa indice para siguiente entrada
            i += 1
            
            ## transformacion aleatoria
            choice = np.random.choice([0,1])
            if choice == 1:
                rImg = datagen.flow(img.reshape((1,) + img.shape),y=None, batch_size=1).next()
                rImg = rImg[0]
                # guardar imagen y entry en el dataframe
                filename = "dataset/" + str(id_offset + i) + ".bmp";
                rImg = cv2.cvtColor(rImg,cv2.COLOR_BGR2RGB)
                cv2.imwrite(filename, rImg)
                newData.append([(id_offset + i), 
                                row['linear'], 
                                target,
                                filename])
                # incrementa indice para siguiente entrada
                i += 1
 
        ### guardamos la imagen en el directorio y el nuevo target en el d
    columns = ['id', 'linear', 'angular', 'imgpath']
    newDataset = pd.DataFrame(newData, columns=columns)
    
    
    newDataset = pd.DataFrame(newData, columns=columns)
    
    dataset = dataset1.append(newDataset, ignore_index=True)
    dataset.to_csv('augmented.csv')
    \end{lstlisting}

\section{Generación de datos}
\begin{lstlisting}[title={utils.py},language=Python]
    from keras.preprocessing.image import load_img
    from keras.preprocessing.image import img_to_array
    
    from keras.models import model_from_json
    
    import numpy as np
    import pandas as pd
    import bcolz
    import threading
    
    import cv2
    
    import os
    import sys
    import glob
    import shutil
    
    from sklearn.model_selection import train_test_split
    
    import models
    
    def generator_from_df(df, batch_size, target_size, target_column='target', features=None, process=True):
        print('generating minibatch!')
        nbatches, n_skipped_per_epoch = divmod(df.shape[0], batch_size)
        #print nbatches
        count = 1
        epoch = 0
        # New epoch.
        while 1:
            df = df.sample(frac=1) # shuffle in every epoch
            epoch += 1
            i, j = 0, batch_size
            # Mini-batches within epoch.
            mini_batches_completed = 0
            for _ in range(nbatches):
                sub = df.iloc[i:j]
                try:
                    if process == True:
                        X = np.array([(2 * (img_to_array(load_img(f, target_size=target_size)) / 255.0 - 0.5)) for f in sub.imgpath])
                    else:
                        X = np.array([((img_to_array(load_img(f, target_size=target_size)))) for f in sub.imgpath])
                        
                    Y = sub[target_column].values
                    # Simple model, one input, one output.
                    mini_batches_completed += 1
                    print ".",
                    
                    yield X, Y
    
                except IOError as err:
                    count -= 1
    
                i = j
                j += batch_size
                count += 1
    
\end{lstlisting}

\section{Nodo de Inferencia}

\begin{lstlisting}[title={neural\_node.py},language=Python]
#!/usr/bin/env python
from __future__ import print_function
import roslib
import rospkg
roslib.load_manifest('robocar')
import sys
import csv
import numpy as np
import rospy
import cv2
import message_filters
from std_msgs.msg import String, Float32
from sensor_msgs.msg import Image, Range, CompressedImage, Joy
from geometry_msgs.msg import Twist, TwistStamped
from cv_bridge import CvBridge, CvBridgeError

from keras.preprocessing.image import load_img
from keras.preprocessing.image import img_to_array

from keras.models import model_from_json

import pandas as pd
import threading

import tensorflow as tf
import os
import sys
import glob
import shutil

from keras.models import Sequential
from keras.layers import Dense, Dropout, Flatten
from keras.layers import Conv2D, MaxPooling2D
from keras import backend as K
from keras.callbacks import TensorBoard, ModelCheckpoint

from keras.utils import plot_model
import models
from models import custom_loss
import imutils

class AutoPilot:
    idx = 0
    
    dim = (224, 224)

    linear = 0
    angular_joy = 0

    def __init__(self, folder):
        rospack = rospkg.RosPack()
        pack_path = rospack.get_path('robocar')
        model_path = pack_path + '/scripts/simple2'
        
        # get ros params
        self.img_topic = rospy.get_param('img_topic', default='/camera/image/compressed')
        self.output_topic = rospy.get_param('output_topic', default='/neural_output')
        self.model_name = rospy.get_param('model', default=model_path)

        ## cargar la red neuronal en la memoria 
        # load json and create model
        json_file = open(self.model_name + '.json', 'r')
        loaded_model_json = json_file.read()
        json_file.close()
        self.model = model_from_json(loaded_model_json)

        # load weights into new model
        self.model.load_weights(self.model_name + "_best.h5")
        print("Loaded model from disk") 
        self.model.compile(loss='mse', optimizer='adam', metrics=['accuracy'])
        self.model.summary()
        self.graph = tf.get_default_graph()

        print('creando subs y pubs...')
        # image subscriber for the predictor
        self.image_sub = rospy.Subscriber(self.img_topic, CompressedImage, self.imCallback, queue_size=1)
        
        # float32 publisher for output 
        self.output_pub = rospy.Publisher(self.output_topic, Float32, queue_size=1)
    
    #this callback executes when the two subscribers sync
    def imCallback(self, img):
        """ este calback lee la imagen de la camara, la preprocesa y obtiene 
        una prediccion para el comando de control del robot"""

        # lee la imagen y la preprocesa
        np_image = cv2.imdecode(np.fromstring(img.data, np.uint8),cv2.IMREAD_COLOR)
        np_image = cv2.resize(np_image, self.dim, interpolation = cv2.INTER_AREA)
        #print (np_image.shape)
        np_image = (2 * (np_image / 255.0 - 0.5))
        x = np_image.reshape((1,) + np_image.shape)  # this is a Numpy array with shape (1, h,w, c)
        #print (x.shape)
        # obtiene la prediccion de la red neuronal
        angular = 0.0
        with self.graph.as_default():
            angular = self.model.predict(x, batch_size=1, verbose=0)

        angular = np.asscalar(angular.flatten())
        #print ("prediccion: ", angular)
        # crea el mensaje para el control del carro y publica 
        # msg = Twist()
        
        # msg.angular.z = angular
        # self.twist_pub.publish(msg)

        output_msg = Float32()
        output_msg.data = angular
        self.output_pub.publish(output_msg)

        
def main(args):
    rospy.init_node('neural_node', anonymous=True)
    stamper = AutoPilot(None)

    try:
        rospy.spin()
    except KeyboardInterrupt:
        print("shutting down")
    cv2.destroyAllWindows()

if __name__ == '__main__':
    main(sys.argv)

\end{lstlisting}

\section{Nodo de detección de obstáculos}
\begin{lstlisting}[title={obstacle\_node.py},language=Python]
    #!/usr/bin/env python
    from __future__ import print_function
    import roslib
    import rospkg
    import rospy
    roslib.load_manifest('robocar')
    
    from sensor_msgs.msg import Range
    from std_msgs.msg import Float32
    import sys
    import csv
    
    class ObstacleDetector:
        idx = 0
        # model_name = '/home/pepe/catkin_ws/src/robocar/scripts/simple2'
        dim = (224, 224)
    
        linear = 0
        angular_joy = 0
        max_acc = 0.3
    
        kp = 4.0
    
        def __init__(self, folder):
            # get ros params
            rospy.loginfo("Obstacle node init")
            self.range_topic = rospy.get_param('range_topic', default='/laser')
            self.output_topic = rospy.get_param('output_topic', default='/obstacle_output')
            self.stop_distance = rospy.get_param('stop_distance', default=0.15)
            # input subscriber for the predictor
            self.range_sub = rospy.Subscriber(self.range_topic, Range, self.RangeCallback, queue_size=1)
            
            # float32 publisher for output 
            self.output_pub = rospy.Publisher(self.output_topic, Float32, queue_size=1)
        
        #this callback executes when the two subscribers sync
        def RangeCallback(self, msg):
            """ este calback recibe el rango del sensor y calcula la salida 
            correspondiente
            """
            error = msg.range - self.stop_distance
            error = error * 2 if error < 0 else error 
            
            acceleration = self.kp * error
            acceleration = self.max_acc if acceleration > self.max_acc else acceleration
            acceleration = -self.max_acc * 2 if acceleration < -self.max_acc else acceleration
            acceleration = 0 if (acceleration < (self.stop_distance + 0.03)) and (acceleration > (self.stop_distance - 0.01)) else acceleration
    
            out_msg = Float32()
            out_msg.data = acceleration
            self.output_pub.publish(out_msg)
        
    def main(args):
        rospy.init_node('neural_node', anonymous=True)
        stamper = ObstacleDetector(None)
        rospy.spin()
       
    if __name__ == '__main__':
        main(sys.argv)
    
\end{lstlisting}

\section{Piloto Automático}

\begin{lstlisting}[title={pilot\_node.cpp},language=c++]
    #include <ros/ros.h>
    #include <boost/thread.hpp>
    #include <std_msgs/Float32.h>
    
    #include <std_msgs/String.h>
    #include <geometry_msgs/Twist.h>
    #include <geometry_msgs/TwistStamped.h>
    #include <sensor_msgs/Joy.h>
    
    #include <cmath>
    
    class Pilot
    {
    
      private:
        // data
        bool manual_;
        // topics from param server
        std::string neural_topic_;
        std::string obstacle_topic_;
        std::string joy_twist_topic_;
    
        std::string output_topic_;
    
        int freq_hz_;
        // messages
        geometry_msgs::Twist twist_msg_;          // simple twist message from joystick
        geometry_msgs::Twist manual_twist_msg_;          // simple twist message from joystick
        // geometry_msgs::TwistStamped stamped; // stamped twist for sync
    
        // subs and PUbs
        ros::Subscriber neural_sub_;
        ros::Subscriber obstacle_sub_;
        ros::Subscriber joy_twist_sub_;
    
        ros::Publisher twist_cmd_pub_;
    
        // subs callback
        // for neural steering
        void NeuralCallback(const std_msgs::Float32::ConstPtr &steering);
    
        // for obsTacle nOde
        void ObstacleCallback(const std_msgs::Float32::ConstPtr &acceleration);
    
        // for joy tELeop
        void JoyTwistCallback(const geometry_msgs::Twist::ConstPtr &joy_twist);
    
        // for find object 2d
        void ObjectCallback(const std_msgs::Float32::ConstPtr &steering);
    
      public:
        Pilot();
    
        ros::Timer timer;
    
        ros::NodeHandle nh_;
        void TimerCallback(const ros::TimerEvent &event);
        float GetInterval();
    };
    // constructor
    Pilot::Pilot() : manual_(false)
    {
        ROS_INFO("Iniciando nodos");
    
        nh_.param<int>("freq_hz", freq_hz_, 20);
        if (freq_hz_ <= 0)
        {
            ROS_WARN("Invalid frequency value, default: 20Hz");
            freq_hz_ = 20;
        }

        // messages are received in QUEues
        nh_.param<std::string>("neural_topic", neural_topic_, "/neural_output");
        nh_.param<std::string>("obstacle_topic", obstacle_topic_, "/obstacle_output");
        nh_.param<std::string>("joy_twist_topic", joy_twist_topic_, "/joy_cmd_vel");
    
        nh_.param<std::string>("output_topic", output_topic_, "/cmd_vel");
        
        // subscribe to nodes
        ROS_INFO("Creando subs y pubs");
        neural_sub_ = nh_.subscribe<std_msgs::Float32>(neural_topic_, 1, &Pilot::NeuralCallback, this);
        obstacle_sub_ = nh_.subscribe<std_msgs::Float32>(obstacle_topic_, 1, &Pilot::ObstacleCallback, this);
        joy_twist_sub_ = nh_.subscribe<geometry_msgs::Twist>(joy_twist_topic_, 1, &Pilot::JoyTwistCallback, this);

        // output topic por piLOt node
        twist_cmd_pub_ = nh_.advertise<geometry_msgs::Twist>(output_topic_, 1);
    
    }
    
    float Pilot::GetInterval()
    {
        return 1.0 / freq_hz_;
    }
    
    ///////////////////////////////////////
    // CALLBACKS //////////////////////////
    ///////////////////////////////////////
    
    void Pilot::NeuralCallback(const std_msgs::Float32::ConstPtr &steering)
    {
        // extract the command
        // put in the message
        // double angular = steering;
        twist_msg_.angular.z = steering->data;
    }
    
    void Pilot::ObstacleCallback(const std_msgs::Float32::ConstPtr &acceleration)
    {
        // double LinEAr = acceleration;
        twist_msg_.linear.x = acceleration->data;
    }
    
    void Pilot::JoyTwistCallback(const geometry_msgs::Twist::ConstPtr &joy_twist)
    {
        manual_ = true;
        manual_twist_msg_.linear.x = joy_twist->linear.x;
        manual_twist_msg_.angular.z = joy_twist->angular.z;
    }
    
    void Pilot::TimerCallback(const ros::TimerEvent &event)
    {
        if(abs(manual_twist_msg_.linear.x) > 0.01 || abs(manual_twist_msg_.angular.z) > 0.01)
        {
            twist_cmd_pub_.publish(manual_twist_msg_);
            manual_ = false;
        }
        else
        {
            twist_cmd_pub_.publish(twist_msg_);
            manual_ = false;
        }
    }
    
    
    int main(int argc, char **argv)
    {
        ros::init(argc, argv, "pilot_node");
        Pilot teleop_joy;
    
        teleop_joy.timer = teleop_joy.nh_.createTimer(ros::Duration(teleop_joy.GetInterval()),
                                                        &Pilot::TimerCallback,
                                                        &teleop_joy);
        ros::spin();
    }
\end{lstlisting}

\section{Control teleoperado con Joystick}

\begin{lstlisting}[title={joy\_teleop.cpp},language=c++]
    #include <ros/ros.h>
    #include <boost/thread.hpp>
    #include <geometry_msgs/Twist.h>
    #include <geometry_msgs/TwistStamped.h>
    #include <sensor_msgs/Joy.h>
    
    class TeleopRobocar
    {
      public:
        TeleopRobocar();
    
        ros::Timer timer;
    
        ros::NodeHandle nh_;
        void timerCallback(const ros::TimerEvent &event);
        float GetInterval();
    
      private:
        void joyCallback(const sensor_msgs::Joy::ConstPtr &joy);
    
        int linear_; // axis id
        int brake_;
        int angular_;
    
        double l_scale_;
        double a_scale_;
        double l_offset_;
    
        std::string output_topic;
    
        int freq_hz_;
    
        ros::Publisher vel_pub_;
        ros::Publisher velStamped_pub_;
        ros::Subscriber joy_sub_;
    
        // for callbacks
        geometry_msgs::Twist twist;          // simple twist message from joystick
        geometry_msgs::TwistStamped stamped; // stamped twist for sync
    };
    
    TeleopRobocar::TeleopRobocar()
    {
        nh_.param<int>("freq_hz", freq_hz_, 20);
        if (freq_hz_ <= 0)
        {
            ROS_WARN("Invalid frequency value, default: 20Hz");
            freq_hz_ = 20;
        }
    
        nh_.param<int>("axis_linear", linear_, 5);
        nh_.param<int>("axis_brake", brake_, 2);
        nh_.param<double>("scale_linear", l_scale_, -0.5);
        nh_.param<double>("offset_linear", l_offset_, 0.5);
        nh_.param<int>("axis_angular", angular_, 0);
        nh_.param<double>("scale_angular", a_scale_, -1.32);    // -0.34
        nh_.param<std::string>("output_topic", output_topic, "/joy_cmd_vel");
    
        vel_pub_ = nh_.advertise<geometry_msgs::Twist>(output_topic, 1);
        velStamped_pub_ = nh_.advertise<geometry_msgs::TwistStamped>("/stamped_cmd_vel", 1);
    
        joy_sub_ = nh_.subscribe<sensor_msgs::Joy>("joy", 1, &TeleopRobocar::joyCallback, this);
    }
    
    float TeleopRobocar::GetInterval()
    {
        return 1.0 / freq_hz_;
    }
    
    void TeleopRobocar::joyCallback(const sensor_msgs::Joy::ConstPtr &joy)
    {
    
        twist.angular.z = joy->axes[angular_] * a_scale_;
    
        double acceleration = l_scale_ * joy->axes[linear_] + l_offset_;
        ROS_DEBUG("%f", acceleration);
    
        double reverse = l_scale_ * joy->axes[brake_] + l_offset_;
        ROS_DEBUG("%f", reverse);
    
        twist.linear.x = acceleration - reverse;
    
        ROS_DEBUG("%f", twist.linear.x);
    
        // vel_pub_.publish(twist);
    }
    
    void TeleopRobocar::timerCallback(const ros::TimerEvent &event)
    {
        stamped.twist = twist;
        stamped.header.stamp = ros::Time::now();
    
        vel_pub_.publish(twist);
        velStamped_pub_.publish(stamped);
    }
    
    int main(int argc, char **argv)
    {
        ros::init(argc, argv, "joy_teleop");
        TeleopRobocar teleop_joy;
    
        teleop_joy.timer = teleop_joy.nh_.createTimer(ros::Duration(teleop_joy.GetInterval()),
                                                        &TeleopRobocar::timerCallback,
                                                        &teleop_joy);
        ros::spin();
    }
\end{lstlisting}
\section{Nodo del sensor de proximidad}

\begin{lstlisting}[title={laser\_node.cpp},language=c++]
    #include "VL53L0X.h"
    #include <ros/ros.h>
    #include <sensor_msgs/Range.h> // para el sensor
    // libreria para el sensor laser
    class LaserNode
    {
      public:
        LaserNode();
    
        ros::Timer timer;
    
        ros::NodeHandle nh_;
        void timerCallback(const ros::TimerEvent &event);
        float GetInterval();
      private:
        // sensor
        VL53L0X sensor_;
    
        int freq_hz_;
        double l_scale_;
        double a_scale_;
        double l_offset_;
    
        ros::Publisher laser_pub_;
        // msgs
        sensor_msgs::Range laser_msg_;
    };
    
    LaserNode::LaserNode()
    {
        nh_.param<int>("freq_hz", freq_hz_, 20);
        if (freq_hz_ <= 0)
        {
            ROS_WARN("Invalid frequency value, default: 20Hz");
            freq_hz_ = 20;
        }
    
        // init publisher
        laser_pub_ = nh_.advertise<sensor_msgs::Range>("laser", 1);
        
        // laser
        sensor_.initialize();
        sensor_.setTimeout(200);
        
        // init some fixed data 
        laser_msg_.radiation_type = 1;
        laser_msg_.header.frame_id = "laser";
        laser_msg_.field_of_view = 0.1;
        laser_msg_.min_range = 0.0;
        laser_msg_.max_range = 1.20;
    }
    
    float LaserNode::GetInterval()
    {
        return 1.0 / freq_hz_;
    }
    
    void LaserNode::timerCallback(const ros::TimerEvent &event)
    {
        uint16_t distance = sensor_.readRangeSingleMillimeters();
    
        if (!sensor_.timeoutOccurred())
        {
            // create message and publish
            laser_msg_.range = distance / 1000.0;
            laser_msg_.header.stamp = ros::Time::now();
            laser_pub_.publish(laser_msg_);
        }
        else
        {
            ROS_WARN("Laser sensor timeout");
        }
    }
    
    int main(int argc, char **argv)
    {
        ros::init(argc, argv, "laser_node");
        LaserNode laser_node;
        
        laser_node.timer = laser_node.nh_.createTimer(ros::Duration(laser_node.GetInterval()), 
                                                    &LaserNode::timerCallback, 
                                                    &laser_node);
        ros::spin();
    }
\end{lstlisting}

% \appendix
\chapter{Hojas de datos}\label{apx:datasheet}

\end{document}
