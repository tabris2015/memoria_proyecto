\documentclass[12pt,letterpaper]{book}
\setcounter{secnumdepth}{5}

%%%%%%%%%%%%%%%% for dev
% \includeonly{Chapter_1,Chapter_4}
%%%%%%%%%%%%%%%%
%%%%% CAMBIAR LA PRIMERA LINEA POR LA SIGUIENTE PARA LA MEMORIA DE PROYECTO %%%%%
%\documentclass[12pt,letterpaper,oneside]{book}

% Paquetes basicos ...
\usepackage[spanish]{babel}
\usepackage[utf8]{inputenc} % OJO!!!  => MANTENER ESTA LINEA PARA FACIL CONVERSION A WORD EN EL FUTURO ...
\usepackage{float,xcolor}
\usepackage{graphicx} 
\usepackage{array}
\usepackage{tabularx}
\usepackage{amssymb, amsmath}

% Paquetes extras ... 
\usepackage{subfigure}
% \usepackage{color}
\usepackage{anysize} 
\usepackage{breakcites}
\usepackage{enumitem}
\usepackage{blindtext}
\usepackage{hyperref}

% %% para que aparezca Capitulo
% \usepackage{tocloft}

% \renewcommand{\cftchappresnum}{Capítulo }
% \renewcommand{\cftchapaftersnum}{:}
% \renewcommand{\cftchapnumwidth}{25mm}


\begin{document}
\marginsize{2.5cm}{2cm}{2cm}{2cm} 

% Para que no aparezca la numeracion en el pie de pagina de todo el documento ...
%\pagestyle{empty}


%%%%%% ******  INICIO CARATULA ***** %%%%%%%%%
% Especificaciones de la caratula PPG

\begin{titlepage}
    \begin{center}
    \vspace*{-0.5in}
    \begin{large}
    \textbf{UNIVERSIDAD MAYOR DE SAN ANDRÉS}\\
    \vspace*{0.15in}
    \textbf{FACULTAD DE INGENIERÍA}\\
    \vspace*{0.15in}
    \textbf{CARRERA DE INGENIERÍA ELECTRÓNICA}\\
    \vspace*{0.1in}
    \end{large}
    % Logo UMSA
    \begin{figure}[htb]
    \begin{center}
    \includegraphics[width=8cm]{img/umsa.jpg}
    % \includegraphics{img/umsa.jpg}
    \end{center}
    \end{figure}
    \begin{Large}
    \textbf{PROYECTO DE GRADO} 
    \end{Large}
    \vspace*{0.4in}
    
    \begin{normalsize}
    \textbf{``Aprendizaje fin a fin para la conducción autónoma de vehículos domésticos usando visión artificial y redes neuronales convolucionales''} \\
    \end{normalsize}
    
    \vspace*{0.2in}
    
    \begin{large}
    \textbf{POSTULANTE:} JOSE EDUARDO LARUTA ESPEJO\\
    \end{large}
    
    \begin{large}
    \hspace{0.08in} \textbf{TUTOR:} JAVIER SANABRIA GARCIA\\
    \end{large}
    
    \begin{large}
    \hspace{0.44in} \textbf{D.A.M.:} GONZALO SAMUEL CABA MORALES\\
    \end{large}
    
    \vspace*{0.2in}
    
    \begin{normalsize}
    LA PAZ, AGOSTO 2018\\
    \end{normalsize}
    \end{center}
    \end{titlepage}
    
    
    \thispagestyle{empty}

% \chapter*{}
% \pagenumbering{Roman} % para comenzar la numeracion de paginas en numeros romanos
\begin{flushright}
\textit{Dedicado a mis padres Edwin y Lourdes, \\
pilares fundamentales de mi formación y principal inspiración \\
en mi búsqueda de superación profesional.}
\end{flushright}


% \chapter*{Agradecimientos} % si no queremos que añada la palabra "Capitulo"
\addcontentsline{toc}{chapter}{Agradecimientos} % si queremos que aparezca en el índice
\markboth{AGRADECIMIENTOS}{AGRADECIMIENTOS} % encabezado 
 
Agradezco infinitamente a mis padres Edwin y Lourdes,
por su incansable e incondicional apoyo y paciencia en mi desarrollo personal y moral.

\vspace{1cm}

A mi asesor Javier Sanabria, por su valiosa y desinteresada guía, 
enseñanzas y consejos en el ámbito académico, ético y profesional 
en el desarrollo de este proyecto y a lo largo de toda la carrera.

\vspace{1cm}

A mis profesores, por haberme transmitido amablemente su experiencia y conocimiento 
en las distintas asignaturas en toda la carrera, garantizándome una formación ingenieril integral.

\vspace{1cm}

A mis compañeros y amigos con los que pude compartir experiencias y conocimientos, que han enriquecido 
mi desarrollo profesional dentro de la carrera y mi desarrollo personal fuera de la misma.

% \chapter*{Resumen} % si no queremos que añada la palabra "Capitulo"
\addcontentsline{toc}{chapter}{Resumen} % si queremos que aparezca en el índice
\markboth{RESUMEN}{RESUMEN} % encabezado

Una bonita historia

% Generacion del indice
\include{formato/indices}

% Contenido del PPG
\chapter{Introducción} \label{ch:introduccion}
\pagenumbering{arabic} % para empezar la numeración con números
La gran intro

\chapter{Marco Teórico} \label{ch:m_teorico}

% estructura
\section{Sistemas de Conducción Autónoma}
Un sistema de conducción autónoma es una combinación de varios componentes o subsistemas donde las tareas 
de percepción, toma de decisiones y operación de un vehículo son desarrolladas por un sistema electrónico en lugar
de un conductor humano. Usualmente, un sistema de conducción autónoma incluye varios subsistemas de automatización 
que operan de manera conjunta y coordinada para poder tomar el control total o parcial del vehículo. 

En algunas ocasiones, la automía del control se implementa de manera condicional, es decir, que el sistema toma 
el control del vehículo para ciertas situaciones pero no todo el tiempo como por ejemplo sistemas de estabilización 
de frenos o prevención de impactos. Este tipo de sistemas se ha ido desarrollando e implementando en vehículos comerciales 
de manera paulatina pero todavía no existe un vehículo completamente autónomo circulando por las calles o carreteras. Notese
que los términos autonomía y automatización se usan de manera intercambiable en este contexto.

    \subsection{Niveles de Autonomía}
    Debido al creciente interés e inversión en el desarrolo de sistemas de conducción autónoma se ha establecido 
    una manera de categorizar los niveles de automatización de la conducción por parte de  la Sociedad de Ingenieros en Automoción
    (SAE, por sus siglas en inglés) en la que se definen seis niveles de automatización en vehículos terrestres, acuáticos y aéreos.

        \subsubsection{Nivel 0: Sin automatización}
        El conductor está en completo control de todas las funciones del vehículo en todo momento, no existe intervención 
        de ningún sistema automatizado en el control. Sistemas de alerta de colisión o pérdida de carril entran en esta categoría.
        \subsubsection{Nivel 1: Conducción asistida}
        El conductor tiene el control del vehículo, pero el sistema puede modificar la aceleración o dirección del mismo. Los 
        sistemas de control de velocidad de crucero caen en esta categoría.
        \subsubsection{Nivel 2: Automatización parcial}
        El conductor debe poder ser capaz de tomar el control del vehículo si ciertas se necesitan ciertas correcciones, pero  
        ya no está en control de la aceleración y dirección del vehículo directamente. Es importante resaltar que desde los
        niveles 0 al 2 el conductor no puede estar distraido en ningún momento de la conducción. Los sistemas de parqueo 
        automático representan un buen ejemplo de sistemas de Nivel 2.
        \subsubsection{Nivel 3: Automatización condicional}
        El sistema automatizado tiene el control del vehículo, tanto de la aceleración, dirección así como también del monitoreo 
        del entorno bajo condiciones específicas. El conductor debe estar preparado para intervenir cuando el sistema así lo 
        requiera, por tanto, se permiten distracciones ocasionales. Uno de los sistemas recientemente implementados que cae en esta 
        categoría es el sistema \textit{autopilot} de los vehículos de Tesla Motors. % agregar referencia
        \subsubsection{Nivel 4: Automatización elevada}
        El sistema está en completo control del vehículo y la presencia humana ya no es necesaria, sin embargo, la operación autónoma 
        del vehículo está limitada a condiciones específicas. Si las actuales condiciones del entorno sobrepasan las fronteras 
        de rendimiento definidas, el vehículo puede desplegar un protocolo o secuencia de emergencia. Actualmente el desarrollo 
        de vehículos autónomos o \textit{self driving cars} se enfoca en este nivel. 
        \subsubsection{Nivel 5: Automatización completa}
        El sistema está en completo control del vehículo y la presencia humana no es necesaria en absoluto. El sistema es capaz 
        de proveer las mismas características que en el Nivel 4, pero en esta ocasión puede operar al vehículo en todas las condiciones.
        En este nivel, el conductor pasa a ser un pasajero en el vehículo. Actualmente, no existen sistemas que operen en este nivel.

    La relación entre la responsabilidad del sistema y el conductor en los distintos niveles se puede apreciar en la 
    Tabla \ref{tbl:niveles}:
    
        \begin{table}[!h]
            \centering
            \resizebox{\textwidth}{!}{%
            \begin{tabular}{@{}|c|l|c|c|c|c|@{}}
            \toprule
            \textbf{Nivel SAE} & \textbf{Denominación}      & \textbf{\begin{tabular}[c]{@{}l@{}}Ejecución de aceleración \\ y dirección\end{tabular}} & \textbf{\begin{tabular}[c]{@{}l@{}}Monitoreo del \\ entorno\end{tabular}} & \textbf{\begin{tabular}[c]{@{}l@{}}Responsable en \\ condiciones difíciles\end{tabular}} & \textbf{\begin{tabular}[c]{@{}l@{}}Modos de \\ conducción\end{tabular}} \\ \midrule
            0                  & Sin Automatización         & Humano                                                                                   & \multirow{3}{*}{Humano}                                                   & \multirow{4}{*}{Humano}                                                                  & Ninguno                                                                 \\ \cmidrule(r){1-3} \cmidrule(l){6-6} 
            1                  & Conducción asistida        & Humano y sistema                                                                         &                                                                           &                                                                                          & \multirow{3}{*}{Algunos Modos}                                          \\ \cmidrule(r){1-3}
            2                  & Automatización parcial     & \multirow{4}{*}{Sistema}                                                                 &                                                                           &                                                                                          &                                                                         \\ \cmidrule(r){1-2} \cmidrule(lr){4-4}
            3                  & Automatización condicional &                                                                                          & \multirow{3}{*}{Sistema}                                                  &                                                                                          &                                                                         \\ \cmidrule(r){1-2} \cmidrule(l){5-6} 
            4                  & Automatización elevada     &                                                                                          &                                                                           & \multirow{2}{*}{Sistema}                                                                 & Varios Modos                                                            \\ \cmidrule(r){1-2} \cmidrule(l){6-6} 
            5                  & Automatización completa    &                                                                                          &                                                                           &                                                                                          & Todos los Modos                                                         \\ \bottomrule
            \end{tabular}%
            }
            \caption{Niveles de automatización según SAE. Fuente: SAE} % TODO: referencia
            \label{tbl:niveles}
            \end{table}

    \subsection{Arquitectura de un sistema de conducción autónoma}
    % \subsection{Aprendizaje Profundo}

% TODO: 
\section{Visión por computador}
-
    \subsection{Procesamiento de imágenes}
    \subsection{Filtrado}

\section{Redes Neuronales Artificiales}

    \subsection{Aprendizaje Automático}
    El aprendizaje automático es un subcampo de la inteligencia artificial que intenta extraer 
    patrones mediante un proceso de \textit{aprendizaje} a partir de datos \cite{Mitchell1990}. Este proceso 
    de aprendizaje se define de acuerdo a una \textbf{tarea específica} $T$ que intenta aprenderse en base 
    a \textbf{experiencia pasada} $E$ tomando como referencia una \textbf{medida de rendimiento} $P$ . 
    Dentro de esta definición, se puede listar varios ejemplos de tareas de aprendizaje que usualmente se resuelven  
    usando los conceptos del aprendizaje automático o también llamado \textit{machine learning}:
    \\
    \\
    \begin{itemize}
        \item \textbf{Un algoritmo de aprendizaje que pueda jugar ajedrez:}
        \begin{itemize}
            \item \textbf{Tarea $T$:} Jugar Ajedrez.
            \item \textbf{Medida de Rendimiento $P$:} Porcentaje de partidas ganadas contra el oponente.
            \item \textbf{Experiencia $E$:} Información de varias partidas de práctica.
        \end{itemize}
        
        \item \textbf{Un algoritmo de aprendizaje que pueda reconocer dígitos manuscritos:}
        \begin{itemize}
            \item \textbf{Tarea $T$:} Reconocer y clasificar dígitos manuscritos dentro de una imagen.
            \item \textbf{Medida de Rendimiento $P$:} Porcentaje de dígitos correctamente clasificados.
            \item \textbf{Experiencia $E$:} Base de datos de imágenes de dígitos con sus etiquetas correspondientes.
        \end{itemize}

        \item \textbf{Un algoritmo de aprendizaje que pueda reconocer la voz:}
        \begin{itemize}
            \item \textbf{Tarea $T$:} Extraer una secuencia de palabras de una grabación de voz.
            \item \textbf{Medida de Rendimiento $P$:} Porcentaje de palabras correctamente predichas.
            \item \textbf{Experiencia $E$:} Grabaciones de voz con una transcripción correspondiente.
        \end{itemize}
    
    \end{itemize}
    

    Esta definición de aprendizaje es lo suficientemente amplia como para englobar todas las tareas 
    que el campo del aprendizaje automático intenta resolver en la actualidad. Sin embargo, debido a 
    su naturaleza, se pueden clasificar las tareas de aprendizaje en tres grandes categorías que tienen
    características particulares: aprendizaje supervisado, aprendizaje no supervisado y aprendizaje por refuerzo.

    La diferencia entre estos tres tipos de problemas surge de la distinta naturaleza de la experiencia $E$ disponible
    para el entrenamiento. A continuación, se procede a detallar cada uno de ellos.

        \subsubsection{Aprendizaje supervisado} \label{sss:supervisado}
        En el caso de las tareas de aprendizaje supervisado, la experiencia constituye un conjunto de datos o \textit{dataset}
        que contiene ejemplos con \textit{características} y cada ejemplo está asociado con una \textit{etiqueta}. Por ejemplo, 
        un conjunto de datos de flores donde cada registro contiene datos de la flor (características) y la especie a la que pertenece (etiqueta). 
        Dentro de los algoritmos que atacan problemas de aprendizaje supervisado se pueden encontrar 2 grandes categorías.
            \paragraph{Clasificación}
            Las tareas de clasificación tienen como característica el hecho de que la etiqueta de cada ejemplo en el 
            conjunto de datos pertenece a una categoría o, en otras palabras, tiene una naturaleza discreta y finita. Por ejemplo, 
            en el caso de la clasificación de las flores mencionado anteriormente, la etiqueta solamente puede pertenecer a un
            conjunto finito de especies de flores y cada ejemplo pertenece a una de estas especies.
            \paragraph{Regresión}
            En las tareas de regresión, las etiquetas pertenecen a un conjunto de números reales o de naturaleza 
            contínua. En este caso, las etiquetas no se asocian con categorías sino más bien con otro tipo de variables. Un 
            ejemplo muy conocido es el de la tarea de la predicción del precio de una casa en base a sus características, el precio 
            de una casa no puede categorizarse porque representa un número que puede tener infinitos valores dentro de un rango definido.
        
        En las tareas del aprendizaje supervisado, se puede considerar cada ejemplo como una descripción de 
        una situación (características) en conjunto con una especificación (etiqueta), cada uno de los ejemplos 
        dentro el conjunto de datos son eventos independientes y se pueden analizar por separado. En este sentido
        la tarea del algoritmo es generalizar la respuesta para casos no presentes en el conjunto inicial de datos.

        \subsubsection{Aprendizaje no supervisado}
        En las tareas del aprendizaje no supervisado la experiencia contenida en el conjunto de datos tiene la característica de 
        no poseer ninguna etiqueta, por tanto, usualmente se intenta buscar una estructura escondida dentro el conjunto de datos 
        o, dicho de otra manera, se buscan patrones que puedan presentarse en dichos datos. Estos patrones pueden aprovecharse 
        para extraer información relevante de la naturaleza de datos de muy alta dimensionalidad, información que normalmente no 
        es trivial de encontrar o visualizar por una persona. Entre algunas de las tareas más comunes dentro del aprendizaje no 
        supervisado, se pueden listar:

            \paragraph{Clustering}
            Refiere a la tarea de separar y agrupar los datos en un número finito de conjuntos o \textit{clusters}. Los 
            \textit{clusters} normalmente denotan una estructura oculta dentro de los datos y proporcionan información acerca 
            de la similaridad entre ejemplos del conjunto de datos.

            \paragraph{Reducción de dimensionalidad}
            Uno de los problemas con las bases de datos y conjuntos de datos disponibles es que poseen una dimensionalidad 
            bastante alta haciendo prácticamente imposible para un humano poder visualizar o encontrar patrones e información 
            útil en los mismos. Este problema se suele tratar con algoritmos de reducción de dimensionalidad, en la que 
            se encuentra una representación estimada de los datos pero con menos dimensiones. Uno de los algoritmos más 
            conocidos y usados en esta categoría es el análisis de componente principal o PCA, por sus siglas en inglés, en el 
            que se encuentra una representación de los datos en una menor dimensión usando proyecciones ortogonales.

            \paragraph{Estimación de probabilidad}
            Muchos conjuntos de datos son obtenidos de distintas fuentes y a lo largo de varios intervalos de tiempo, en este 
            entendido, es muy útil conocer o aproximar la distribución de probabilidad de los datos para luego poder realizar 
            predicciones o tratarlos con algún modelo en específico.

        \subsubsection{Aprendizaje por refuerzo}
        En las tareas de aprendizaje por refuerzo se toma en cuenta la interacción de un agente con su entorno y la forma 
        en la que las acciones que toma dicho agente afectan a su entorno y se materializan en una recompensa o castigo \cite{sutton2018reinforcement}. Formalmente
        se pueden definir ciertos elementos que componen una tarea de aprendizaje por refuerzo:
        \begin{itemize}
            \item \textbf{Agente.} Es la entidad que interactúa con el entorno. El agente se comunica con el entorno mediante acciones.
            \item \textbf{Política.} Representan la forma de actuar del agente en base al conocimiento que ha adquirido.
            \item \textbf{Recompensa.} Es la función que define la efectividad del agente de cumplir el objetivo deseado, normalmente, el aprendizaje se enfoca en maximizar la recompensa que el agente puede obtener. 
        \end{itemize}
        
        % TODO: insertar grafico de aprendizaje por refuerzo

    % TODO: 
    \subsection{Aprendizaje Profundo}
    Dentro del campo de la inteligencia artificial y el aprendizaje automático se han implementado diversos tipos 
    de algoritmos con éxito en los tipos de tareas de aprendizaje mencionados anteriormente. La base teórica y los detalles 
    de implementación de estos algoritmos son muy variados, sin embargo, las redes neuronales artificiales han experimentado 
    un incremento en el interés en la investigación y en las aplicaciones muy importante. Tal es el éxito de las mismas 
    que se ha creado un subcampo exclusivo llamado aprendizaje profundo o \textit{deep learning}. El aprendizaje profundo 
    es un campo de la inteligencia artificial que se encarga de estudiar exclusivamente a las redes neuronales artificiales, 
    sus componentes, arquitectura y aplicaciones. El impresionante rendimiento de estos algoritmos reside principalmente en el 
    concepto de la representación que generan a partir de los datos que se procesan. 

    El aprendizaje profundo resuelve el problema del aprendizaje de representaciones al introducir representaciones que se 
    expresan en términos de otras representaciones más simples. Además, permite a una computadora construir conceptos complejos 
    a partir de conceptos más simples. Un ejemplo de la generación de estos conceptos o representaciones se puede apreciar en la 
    Figura(\ref{fig:representacion}).
    
    
    \begin{figure}[!h] 
        \centering
        \includegraphics[width=0.75\textwidth]{img/representacion}
        \caption{Ilustración de un modelo de aprendizaje profundo. Las representaciones se generan en las capas ocultas y corresponden con características de distintos niveles de complejidad. Fuente: \cite{Goodfellow-et-al-2016} }
        \label{fig:representacion}
    \end{figure}

    A continuación, se procede a definir los conceptos más importantes de redes neuronales artificiales con los cuales se podrá 
    plantear la solución al problema de la conducción autónoma usando visión artificial.

        \subsubsection{Redes neuronales feedforward}
        Las redes neuronales feedforward o también llamadas perceptrón multicapa, son la base fundamental de los modelos 
        de aprendizaje profundo. El objetivo de una red neuronal feedforward es el de aproximar una función $f^\ast$. Por 
        ejemplo, para una tarea de clasificación, $y = f^\ast (\mathbf{x})$ mapea una entrada $\mathbf{x}$ a una categoría $y$.
        Una red neuronal feedforward define un mapeo $\mathbf{y} = f(\mathbf{x},\mathbf{W})$ y aprende el valor de los parámetros 
        $\mathbf{W}$ que resulten en la mejor aproximación\cite{Goodfellow-et-al-2016}.

        Este tipo de modelos son denominados feedforward debido a que la información fluye a por la función siendo evaluada 
        desde $\mathbf{x}$, a través de distintos cálculos intermedios definidos por $f$, hasta llegar a la salida $\mathbf{y}$.
        No existen conexiones de retroalimentación en las que la salidas del modelo se inyecten de nuevo a sí mismo. Las redes 
        neuronales que poseen este tipo de conexiones de retroalimentación son denominadas redes neuronales recurrentes.

        Para definir una red neuronal feedforward se puede comenzar definiendo un modelo basado en una combinación 
        lineal en conjunto con una función no lineal que toma la siguiente forma:

        \begin{equation}\label{eq:modelobase}
            y(\mathbf{x}, \textbf{W}) = f\left(\sum_{j=1}^M w_j  x_j\right)
        \end{equation}

        donde $f()$ es una función de activación no lineal. Esto lleva al modelo básico de una red neuronal que puede ser 
        descrita como una serie de transformaciones. Primero, se construyen $M$ combinaciones lineales de las variables 
        de entrada $x_1, \ldots , x_D$ donde $D$ es la dimensión del vector de entrada $\mathbf{x}$:
        
        \begin{equation}
            a_j = \sum_{i=1}^D w_{ji}^{(1)} x_i + w_{j0}^{(1)}
        \end{equation}

        donde $j = 1, \ldots , M$, y el superíndice $(1)$ indican que los correspondientes parámetros se encuentran en la 
        primera capa de la red. Los parámetros $w_{ji}^{(1)}$ se suelen conocer también con el nombre de \textit{pesos} y 
        los parámetros $ w_{j0}^{(1)}$ con el nombre de \textit{sesgos} o \textit{biases}. Las cantidades $a_j$ se conocen 
        como \textit{activaciones}, y cada una de ellas es luego transformada usando una función no lineal y derivable conocida 
        como la \textit{función de activación} $h()$ para luego obtener:

        \begin{equation}
            z_j = h(a_j)
        \end{equation}

        Estas cantidades corresponden con la salida de la capa y también se suelen referir por el nombre de  \textit{unidades ocultas}.
        Las funciones no lineales $h()$ pueden escogerse dependiendo a diversos criterios de rendimiento o de comportamiento. Siguiendo
        a la Ecuación(\ref{eq:modelobase}), las unidades ocultas se pueden volver a procesar con una combinación lineal y función 
        de activación en una segunda capa:

        \begin{equation}
            a_k = \sum_{j=1}^M w_{kj}^{(2)} z_j + w_{k0}^{(2)}
        \end{equation}

        donde $k = 1, \ldots, K$ y $K$ corresponden con el número de salidas de la segunda capa. Finalmente, si consideramos a 
        esta capa como la capa de salida, podemos transformar las activaciones de la segunda capa con una función de activación. 
        Normalmente, para una tarea de regresión, la función de activación es una función identidad, es decir $y_k = a_k$. Para 
        una tarea de clasificación binaria, en cambio, la función de activación es una función sigmoide:

        \begin{equation}
            y_k = \sigma(a_k)
        \end{equation}

        donde
        
        \begin{equation}
            \sigma(a) = \frac{1}{1 + e^{-a}}    
        \end{equation}
        
        Finalmente, se pueden combinar las etapas en una función general de la red que, para una salida sigmoidal, toma la 
        siguiente forma:

        \begin{equation}\label{eq:reddoscapas}
            y_k(\mathbf{x}, \mathbf{W}) = \sigma \left( \sum_{j=1}^M w_{kj}^{(2)} h \left( \sum_{i=1}^D w_{ji}^{(1)} x_i + w_{j0}^{(1)} \right) + w_{k0}^{(2)} \right)
        \end{equation}

        De esta manera, se define una red neuronal de dos capas a partir de la combinación lineal de las entradas y las unidades 
        ocultas con los parámetros o pesos de la red transformados por funciones de activación no lineal. La arquitectura de la 
        red definida en la Ecuación(\ref{eq:reddoscapas}) se puede visualizar en la Figura(\ref{fig:reddoscapas}) donde se observa claramente las relaciones que se 
        han definido anteriormente en forma gráfica y la naturaleza del flujo en una sola dirección (feerforward) de los datos 
        desde la entrada hasta la salida. En este caso, la red neuronal analizada es una red neuronal con una capa oculta.

        \begin{figure}[!h] 
            \centering
            \includegraphics[width=0.75\textwidth]{img/reddoscapas}
            \caption{Diagrama de la red neuronal de dos capas correspondiente a la Ecuación(\ref{eq:reddoscapas}). Fuente: \cite{Bishop2006} }
            \label{fig:reddoscapas}
        \end{figure}

        \subsubsection{Funciones de activación}
        \subsubsection{Funcion de costo}
        \subsubsection{Gradientes y retropropagación}
        \subsubsection{Diseño de Arquitecturas}

    \subsection{Redes Neuronales Convolucionales}
    Las redes neuronales convolucionales son un tipo especializado de red neuronal que 
    sirven para procesar datos de tipo "grilla" \cite{Goodfellow-et-al-2016}. Algunos ejemplos 
    de datos de tipo grilla que se pueden mencionar son los siguientes:
    \begin{itemize}
        \item \textbf{Series de tiempo.} Grilla de una dimensión tomados en intervalos regulares de tiempo.
        \item \textbf{Imágenes digitales.} Grilla de pixeles de dos o más dimensiones (Escala de grises, RGB).
    \end{itemize}

    Las también llamadas redes convolucionales, han demostrado un éxito impresionante en diversas 
    aplicaciones prácticas especialmente en el campo de la visión por computador y el procesamiento de texto y lenguaje natural. 
    El término ``red neuronal convolucional'' proviene del hecho de que en este tipo 
    de redes neuronales se utiliza una operación matemática llamada \textbf{convolución}, siendo la convolución 
    una operación lineal especializada para procesar datos de tipo grilla.

    En los párrafos posteriores, se procede a describir la operación de convolución en el contexto de 
    redes neuronales, pues, no siempre la definición de la misma corresponde con el concepto de convolución
    usado en distintos campos de la ciencia y la ingeniería.

        \subsection{Operación de convolución}
        En su forma más general, la convolución es una operación entre dos funciones reales y su definición se puede introducir
        usando el concepto de un promedio ponderado. Sea una función $x(t)$ dependiente del tiempo, 
        tanto $x$ como $t$ son números reales; en este caso, la función $x$ puede entenderse como una serie de medidas
        en un instante de tiempo $t$. Considérese una segunda función de ponderación $w(\tau)$ donde $\tau$ es la antiguedad 
        de una medida. Si se aplica la función de ponderación en cada instante de tiempo, se puede obtener una nueva función 
        definida por:
        \begin{equation}
            s(t) = \int x(\tau)w(t - \tau) d\tau
        \end{equation} 
        Esta operación es llamada la \textit{operación de convolución} y es denotada tradicionalmente con un asterisco:
        \begin{equation}
            s(t) = (x\ast w)(t)
        \end{equation}
        En el ejemplo de la ponderación, $w$ debe ser una función de densidad de probabilidad válida, o la salida no podrá
        ser considerada como un promedio ponderado. Además, $w$ también debe ser $0$ para cualquier $t<0$, esta última 
        característica se denomina comunmente como el principio de ``causalidad''. En general, la convolución está 
        definida para cualquier función en la cual la integral anteriormente declarada esté definida y puede ser 
        usada para otros propósitos aparte de promedios ponderados.

        Hablando en términos de una red neuronal convolucional, el primer argumento (en el ejemplo, la función $x$) 
        es comunmente referido como la \textbf{entrada}, y el segundo argumento ($w$, en el ejemplo) es referido 
        como el \textbf{kernel}. La salida, a su vez, es normalmente referida como el \textbf{mapa de características}.
        
        Por su parte, cuando se trata de señales digitales, como los datos en una computadora, el tiempo tiene una 
        naturaleza discreta, es decir, que los datos estarán disponibles en intervalos regulares de tiempo. En este 
        caso, el índice de tiempo $t$ puede tomar solamente valores enteros y, entonces, es válido asumir 
        que tanto $x$ como $w$ estan definidos solamente para valores enteros de $t$. De este modo, 
        se puede definir la convolución discreta:
        \begin{equation}
            s(t) = (x \ast w)(t) = \sum_{\tau=-\infty}^{\infty}x(\tau)w(t-\tau)
        \end{equation}

        En el contexto de las aplicaciones de aprendizaje automático o, más específicamente, aprendizaje profundo,
        la entrada es usualmente un arreglo multidimensional de datos, y el kernel es usualmente un arreglo 
        multidimensional de parámetros que se adaptan en el proceso de aprendizaje. 

        \subsubsection{Procesamiento de imágenes con redes neuronales convolucionales}

        % TODO: poner un grafico de la convolucion en 2d de una imagen 


        La operación de convolución se usa frecuentemente sobre datos con más de una dimensión. Las imágenes digitales 
        son un perfecto ejemplo de un arreglo multidimensional de datos. Una imagen digital se representa mediante una
        matriz con filas y columnas, donde cada elemento se denomina pixel y contiene información acerca de la intensidad
        o luminancia, para una imagen en escala de grises o el nivel de color para distintos canales en una imagen a color.
        Si se toma el ejemplo de la imagen en escala de grises, se tiene una entrada o imagen bidimensional $I$ con un
        kernel bidimensional correspondiente $K$:

        \begin{equation} \label{eq:conv2d}
            S(i,j)=(I\ast K)(i,j) = \sum_{m} \sum_{n} I(m,n)K(i-m,j-n)
        \end{equation}

        Dado que la convolución es conmutativa, se puede reescribir la ecuación \ref{eq:conv2d} como:

        \begin{equation}
            S(i,j)=(K\ast I)(i,j) = \sum_{m} \sum_{n} I(i-m,j-n)K(m,n)
        \end{equation}

        Frecuentemente, la última fórmula es la más utilizada en librerías de aprendizaje profundo 
        por su sencillez en la implementación en un sistema computacional, esto, dado que existe menos 
        variación en el rango de valores válidos de $m$ y $n$.

        \subsubsection{Aprendizaje de representaciones internas}
        Una de las preguntas clave en la visión por computador es el cómo generar una buena y significativa
        representación interna de una imagen, dado que la mayor parte de la imagen corresponde con pixeles que no 
        aportan mucha información relevante a la tarea asignada. Por ejemplo, si se quisiera detectar 
        un rostro dentro de una imagen, normalmente se suele encontrar una representación que ayude a aislar solamente 
        las porciones de la imagen que pueden contener el rostro, tales como la búsqueda de contornos, bordes y 
        características típicas de un rostro. Antes de la aparición de las redes convolucionales, estas representaciones 
        se hallaban de manera manual y gracias al conocimiento de expertos en el área del procesamiento de imágenes. 
        La definición de características y mapas de características era comunmente conocida como la 
        \textit{ingeniería de características}, en la cual los expertos creaban descriptores para tareas específicas con 
        una gran inversión de tiempo en la sintonización fina de los mismos. 

        % TODO: poner el ejemplo de viola jones 

        En contraste con el anterior enfoque, las redes convolucionales generan sus propias representaciones internas
        de manera automática gracias al aprendizaje de los parámetros de cada uno de los kernels que componen las distintas 
        capas de la red neuronal. En principio, las redes convolucionales se inspiraron en el trabajo de Hubel y Wiesel 
        sobre la corteza visual primaria de un gato\cite{lecun2010convolutional}. En dicho trabajo, se logró identificar células simples que respondían
        de manera sobresaliente a distintas orientaciones con campos receptivos locales. Éstas células receptivas simples 
        se pueden corresponder con los kernels de convolución usados en las redes convolucionales por la sencillez y la 
        localidad de su campo de receptividad.

        Posteriormente, las redes convolucionales ganaron una gran popularidad debido a su rendimiento en tareas de 
        clasificación de imágenes y detección y reconocimiento de objetos en imágenes. El primer hito de su capacidad 
        para procesar imágenes de manera efectiva fue en concurso de clasificación de imágenes de ImageNet, donde 
        el equipo de Geoffrey Hinton logró sobrepasar el mejor resultado en precisión de clasificación por un gran márgen 
        usando una arquitectura de red convolucional \cite{krizhevsky2012imagenet}. En este trabajo, se pudo apreciar con 
        gran detalle las ventajas del enfoque del aprendizaje de representaciones internas en una red convolucional.

        \begin{figure}[!h] 
            \centering
            \includegraphics[width=0.75\textwidth]{img/fmap_imagenet}
            \caption{Kernels convolucionales de tamaño $11 \times 11 \times 3$ en la primera capa convolucional. Fuente: \cite{krizhevsky2012imagenet} }
            \label{fig:fmap_imagenet}
        \end{figure}
            
        Tal como se puede apreciar en la Figura(\ref{fig:fmap_imagenet}), en la primera capa convolucional, 
        los kernels de convolución corresponden con representaciones básicas en una imagen como la búsqueda de 
        bordes en distintas orientaciones, esto va acorde a lo establecido anteriormente en el modelo de 
        la corteza visual de un gato. Puede decirse entonces que las redes convolucionales emulan, en cierto modo, 
        al proceso biológico de visión en animales.

    % TODO: 
    \subsection{Sistemas de Aprendizaje Fin a Fin}

\section{Modelo cinemático del vehículo}
-
    \subsection{Ecuaciones de movimiento}




% \chapter{Marco Práctico} \label{ch:m_practico}
\section{Arquitectura del sistema}
    \subsection{}
        \subsubsection{}
\section{Subsistema de Adquisición de Datos y Entrenamiento}
    \subsection{Descripción general del subsistema}
    \subsection{Módulo de adquisición de datos y operación manual}
    \subsection{Módulo de aumentación de datos y almacenamiento}
    \subsection{Módulo de Entrenamiento}
\section{Subsistema de Control y actuación}
    \subsection{Descripción general del subsistema}
    \subsection{Características del prototipo físico}
    \subsection{Módulo de potencia y sensado de tiempo real}
    \subsection{Módulo de la computadora de abordo}
    \subsection{Interfaces de comunicación}
    
\section{Subsistema de Inferencia y control autónomo}

% \chapter{Análisis y discusión de resultados}
\label{ch:resultados}

\section{Pruebas}
\section{Resultados generales}
\section{Proceso de entrenamiento}
    \subsection{Análisis de las curvas de entrenamiento}
    \subsection{Análisis del rendimiento en conjuntos de prueba y validación cruzada}
\section{Despliegue del prototipo}

% \chapter{Conclusiones y recomendaciones}
\label{ch:conclusiones}
\section{Conclusiones}
Se detallan las conclusiones correspondientes con cada objetivo específico planteado:

\begin{itemize}

    \item \textit{Estudiar los aspectos concernientes al desarrollo de sistemas de conducción autónoma y sistemas de aprendizaje.}
    
    \textbf{Conclusión:} Se estudiaron los aspectos relacionados al desarrollo de sistemas de conducción autónoma, en principio, 
    analizando su origen y potencialidad en la Sección(\ref{sec:antecedentes}). Para posteriormente estudiar los conceptos básicos 
    de aprendizaje 
    automático, aprendizaje profundo, redes convolucionales y el proceso de entrenamiento de una red neuronal.

    \item \textit{Analizar los requerimientos de un sistema de conducción autónoma capaz identificar y mantener su carril 
    mientras se conduce.}

    \textbf{Conclusión:} Se ha introducido el esquema de la arquitectura de un sistema de conducción autónomo y se han planteado 
    los requisitos y funcionalidades que debe tener el sistema en su conjunto en la Sección(\ref{sec:arquitectura}) y de las 
    características de los 
    subsistemas de los que se compone: el subsistema de control y actuación en la Sección(\ref{sec:esqcontrol}), el subsistema de adquisición de 
    datos y entrenamiento en la Sección(\ref{sec:esqdaq}), el subsistema de inferencia y control autónomo en la Sección(\ref{sec:esqinferencia}). 

    \item \textit{Diseñar la arquitectura de un sistema de conducción autónoma en base a los requerimientos previamente 
    establecidos.}

    \textbf{Conclusión:} Se ha planteado la arquitectura general del sistema en base a las características analizadas en 
    el Capítulo(\ref{ch:introduccion}) y (\ref{ch:m_teorico}) en la Sección(\ref{sec:arquitectura}). Considerando todos los alcances
    y limitaciones planteados por el proyecto en el inicio. 
    Por su parte, también se han definido las herramientas de hardware en la Sección(\ref{sec:software}) y las herramientas 
    de software en 
    la Sección(\ref{sec:software}) que hicieron posible la implementación de todos los subsistemas asociados.

    \item \textit{Diseñar el subsistema de adquisición de datos y entrenamiento para tareas de conducción autónoma.}
    
    \textbf{Conclusión:} Se ha diseñado el Subsistema de Adquisición de Datos y Entrenamiento en base a la arquitectura planteada 
    en la Sección(\ref{sec:esqdaq}) en la Sección(\ref{sec:daq}) considerando los detalles de implementación, algoritmos y características necesarias para 
    las tareas de adquisición de datos, aumentación de datos y entrenamiento de modelos de redes neuronales explorados en el Capítulo(\ref{ch:ingenieria}).
    Los detalles del proceso de diseño y entrenamiento de las arquitecturas de redes neuronales planteadas se exploran con detalle en la 
    Sección(\ref{sec:design}) para el diseño de la red neuronal y en la Sección(\ref{sec:training}) para el entrenamiento. 

    \item \textit{Diseñar el subsistema de control y actuación para la conducción autónoma de un vehículo con características 
    similares a las de un vehículo doméstico real.}

    \textbf{Conclusión:} Se ha diseñado el Subsistema de Control y Actuación de acuerdo con la arquitectura planteada en la 
    Sección(\ref{sec:arquitectura}) en la Sección(\ref{sec:control}). Se ha considerado las características y requerimientos 
    planteados para dicho subsistema en 
    las tareas de control de tiempo real implementado en un microcontrolador, interfaz con los actuadores y sensores 
    sobre la plataforma de comunicación de ROS. Se han expuesto los detalles de implementación tanto a nivel de hardware 
    como de software procurando que el diseño de este subsistema sea modular.

    \item \textit{Diseñar el subsistema de inferencia y control autónomo basado en el uso de redes neuronales convolucionales.}
    
    \textbf{Conclusión:} Por su parte, se ha diseñado a detalle el Inferencia y Control autónomo en base a todos los fundamentos 
    teóricos de redes neuronales exploradas en la Sección(\ref{sec:sec:esqinferencia}). Los detalles de la implementación se pueden 
    encontrar en la Sección(\ref{sec:inferencia})
    donde se abunda en la implementación de los módulos que componen este subsistema para las tareas de predicción de dirección
    con la red neuronal convolucional, detección de obstáculos con un sensor de proximidad y el algoritmo del piloto automático 
    implementados como nodos de ROS para garantizar la modularidad del sistema.

    \item \textit{Analizar los resultados del entrenamiento e implementación del subsistema de inferencia y control autónomo.}
    
    \textbf{Conclusión:} El análisis de los resultados del entrenamiento de la red neuronal convolucional se ha explorado en 
    la Sección(\ref{sec:analisistrain}) pudiendo hacer la comparación de resultados en errores de entrenamiento, validación 
    y prueba de dos arquitecturas:
    una con una red neuronal tradicional o densamente conectada, y otra con una red neuronal convolucional. En base a los 
    resultados y puntajes en el entrenamiento obtenidos se puede concluir que una red neuronal convolucional es capaz de cumplir la 
    tarea de generar comandos de control para la tarea de conducción autónoma. 

    Por otro lado, la implementación de la red y los resultados en pruebas de campo se han explorado en la 
    Sección(\ref{sec:analisistest}) en la cual 
    se pudo observar la efectividad de la red neuronal convolucional para generar representaciones internas relevantes y útiles. 
    En base a este análisis se puede concluir que, efectivamente, una la red convolucional entrenada es capaz de generar 
    representaciones internas relevantes en muestras nunca antes vistas, en otras palabras, que la capacidad de generalización 
    es aceptable.

    \item \textit{Realizar pruebas de rendimiento y análisis comparativos en el sistema implementado.}
    
    \textbf{Conclusión:} Con el fin de realizar un análisis comparativo entre distintas implementaciones de redes neuronales 
    para la tarea especificada por este proyecto, se ha diseñado y entrenado un par de arquitecturas de red neuronal sobre las 
    cuales se realiza un análisis comparativo de rendimiento en base a indicadores y puntajes estándar en el campo de la estadística
    en la Sección(\ref{sec:scores}). 

    Por su parte, también se ha analizado las predicciones generadas por la red convolucional, para distintos casos, explorando 
    la naturaleza de las representaciones generadas por la misma en la Sección(\ref{sec:representaciones}). Por tanto, se puede 
    concluir que la implementación 
    de la red neuronal convolucional planteada se ha realizado exitosamente.

\end{itemize}

\section{Recomendaciones}

Luego de haber analizado y generado conclusiones relativas a los resultados obtenidos en base a los objetivos planteados en 
la etapa inicial del presente proyecto, se plantea una serie de recomendaciones:

\begin{itemize}
    \item Debido a la capacidad de generalización que puede lograrse con una red neurona convolucional se recomienda 
    generar distintos conjuntos de entrenamiento en condiciones climáticas diversas. Esto incrementará la complejidad 
    de la red convolucional y logrará que las predicciones de la red sean más robustas en cambios de iluminación 
    causados por distintos aspectos climáticos.
    \item Por su parte, gracias a la naturaleza modular del presente proyecto, se recomienda diseñar, entrenar y 
    validar arquitecturas de redes neuronales distintas o con variaciones a una red neuronal convolucional 
    tradicional. Se recomienda, por ejemplo, implementar una red convolucional recurrente que sea capaz de tomar 
    en cuenta el estado anterior de la misma. Implementar otros algoritmos de predicción y visión artificial también 
    ayudarían al desarrollo de sistemas más robustos.
    \item Se recomienda también extender la implementación de la red a una tarea de clasificación categórica para 
    comparar el rendimiento entre el entrenamiento y rendimiento obtenidos en el problema planteado como una regresión 
    y una clasificación.
    \item Debido ala modularidad del sistema planteando en el presente proyecto, se recomienda implementar el mismo 
    sistema fin a fin para otros modelos de vehículos, como pueden ser un robot de tracción diferencial u otro modelo.
    \item Es también importante poder extender el sistema presentado en este proyecto con tareas de control y detección 
    avanzadas, incluyendo en el flujo de trabajo módulos de planificación de trayectorias, detección de objetos y 
    programación de misiones.
    \item Dada la gran potencialidad en el área de sistemas de conducción autónoma se recomienda que se pueda crear 
    una línea de investigación dedicada al diseño e implementación de sistemas robóticos inteligentes en la cual 
    se agrupen esfuerzos para el desarrollo de cada uno de los subsistemas que componen un vehículo autónomo. 
    \item Por último, la forma en la que ha sido implementado el presente proyecto permite modificarlo y extenderlo 
    de distintas maneras y a distintos niveles. Se recomienda considerar este proyecto como una plataforma de 
    desarrollo sobre la cual se puedan implementar diversos algoritmos y aplicaciones útiles para el desarrollo de 
    nuestra sociedad.

\end{itemize}

% % \appendix
\chapter{Código Fuente del Proyecto}\label{apx:source}
\section{Control de Actuadores en Tiempo Real}
\begin{lstlisting}[title={main.cpp},language=c++]
#include <mbed.h>
#include <Servo.h>
#include <Motor.h>
#include <ros.h>

// ros msgs
#include <geometry_msgs/Twist.h> // sub to Twist message for control
#include <sensor_msgs/Range.h>   // pub to Range message for sensor

DigitalOut led(D13);

Servo steering(D9);            // steering servo
Motor motor(D12, D10, D11); // motor driver

// ROS part
// software utils
Timer t;
Ticker pub_ticker;
// ros part
ros::NodeHandle nh;

// messages
geometry_msgs::Twist control_cmd; // control command is of type twist

// callback prototypes
void ctrlCommandCb(const geometry_msgs::Twist &command);

// subscribers
// ros subscriber for control command
ros::Subscriber<geometry_msgs::Twist> ctrlSub("cmd_vel", ctrlCommandCb);

// prototypes 
// callbacks
void ctrlCommandCb(const geometry_msgs::Twist &command);

int main() {
    
    motor.period(0.0005);     // frecuencia 2 Khz
    steering = 0;
    // init sensors
    led = 1;
    // ros initialization
    nh.initNode();
    nh.subscribe(ctrlSub);
    //
    led = 0;

    while(1) {
        nh.spinOnce();
        wait_ms(1);
    }
}

// control command callback
void ctrlCommandCb(const geometry_msgs::Twist &command)
{
    led = !led;
    float linear = command.linear.x;
    float angular = command.angular.z / 2 + 0.5; // (command.angular.z * 0.35) + 0.32;
    steering = angular;
    // testigo = abs(linear * 8);
    motor.speed(linear);
}
\end{lstlisting}

\section{Script de entrenamiento}
\begin{lstlisting}[title={train\_model.py},language=Python]
from keras.preprocessing.image import load_img
from keras.preprocessing.image import img_to_array

from keras.models import model_from_json
    
import numpy as np
import pandas as pd
import bcolz
import threading
    
from time import time
import os
import sys
import glob
import shutil

from sklearn.model_selection import train_test_split
    
from keras.models import Sequential
from keras.layers import Dense, Dropout, Flatten
from keras.layers import Conv2D, MaxPooling2D
from keras import backend as K
from keras.callbacks import TensorBoard, ModelCheckpoint
    
from keras.utils import plot_model
import models
from utils import *
    
class RobocarTrainer(object):
        
    # data for training
    input_shape=(224,224,3)
    im_shape = (224, 224)
    
    # train parameters
    batch_size = 32
    n_epochs = 100
    
    def __init__(self, model_name, model_path, dataset_path, log_path='trainlogs'):
        self.model_name = model_name
        self.model_path = model_path
        self.log_path = log_path
        self.dataset_path = dataset_path
            
        # creating callbacks
        self.tfBoardCB = TensorBoard('{}/{}_{}'.format(self.log_path, model_name, time()), write_graph=True)
    
        filepath= model_path + model_name + '_best.h5'
    
        self.checkpointCB = ModelCheckpoint(filepath, monitor='val_loss', verbose=1, save_best_only=True, mode='min')
    
        
    def LoadDataset(self):
    
        print('loading dataset...')
        self.dataset = pd.read_csv(self.dataset_path + 'target.csv')
    
        self.dataset['imgpath'] = self.dataset.id.apply(file_path_from_db_id, args=("%d.bmp", self.dataset_path))
    
        self.train, self.test = train_test_split(self.dataset, test_size=0.2)
        self.valid, self.test = train_test_split(self.test, test_size=0.7)
        
        self.train_steps = int(self.train.shape[0] / self.batch_size)
        self.valid_steps = int(self.valid.shape[0] / self.batch_size)
        self.test_steps = int(self.test.shape[0] / self.batch_size)
        print('dataset loaded!')
    
    def Train(self):
    
        print('loading model...')
        self.model = models.vanilla(self.input_shape)
        self.model.summary()
        model_json = self.model.to_json()
        with open(self.model_path + self.model_name + '.json', "w") as json_file:
                json_file.write(model_json)
                
        plot_model(self.model, to_file=self.model_name + '.png', show_shapes=True)
        print('dataset size: ', self.train.shape[0],
                    'train_steps: ', self.train_steps, 
                    'valid steps: ', self.valid_steps, 
                    'test_steps: ', self.test_steps)
            
        print('hiperparameters:')
        print('batch size:{}'.format(self.batch_size))
            
    
        print('training...')
        self.model.fit_generator(
                                generator_from_df(self.train, self.batch_size, self.im_shape, 'angular'),
                                steps_per_epoch=self.train_steps, 
                                epochs=self.n_epochs,
                                validation_data=generator_from_df(self.valid, self.batch_size, self.im_shape, 'angular'),
                                validation_steps=self.valid_steps,
                                callbacks=[self.tfBoardCB, self.checkpointCB],
                                verbose=2
                                )
        print('finished training')
        self.score = self.model.evaluate_generator(
                                generator_from_df(self.test, self.batch_size, self.im_shape, 'angular'), 
                                steps=self.test_steps
                                )
        print('loss: ', self.score)
    
    def SaveModel(self):
        # serialize model to JSON
        self.model_json = self.model.to_json()
        with open(self.model_name + '.json', "w") as json_file:
                json_file.write(self.model_json)
        # serialize weights to HDF5
        self.model.save_weights(self.model_name + '.h5')
        print("Saved model to disk")
    
    
    
    
if __name__ == '__main__':
        
    import argparse
    parser = argparse.ArgumentParser(description='Entrenamiento de una red neuronal convolucional')
    parser.add_argument("model_name", help="name of the model to train")
    parser.add_argument("model_path", help="path where the model files will be saved")
    parser.add_argument("dataset_path", help="path where the the dataset is saved")
        
    args = parser.parse_args()
        
    trainer = RobocarTrainer(args.model_name, args.model_path, args.dataset_path)
    trainer.LoadDataset()
    trainer.Train()
    trainer.SaveModel()

\end{lstlisting}

\section{Sincronización de Mensajes}

\begin{lstlisting}[title={msg\_sync.py},language=Python]
    from keras.preprocessing.image import load_img
    from keras.preprocessing.image import img_to_array
    
    from keras.models import model_from_json
        
    import numpy as np
    import pandas as pd
    import bcolz
    import threading
        
    from time import time
    import os
    import sys
    import glob
    import shutil
    
    from sklearn.model_selection import train_test_split
        
    from keras.models import Sequential
    from keras.layers import Dense, Dropout, Flatten
    from keras.layers import Conv2D, MaxPooling2D
    from keras import backend as K
    from keras.callbacks import TensorBoard, ModelCheckpoint
        
    from keras.utils import plot_model
    import models
    from utils import *
        
    class RobocarTrainer(object):
            
        # data for training
        input_shape=(224,224,3)
        im_shape = (224, 224)
        
        # train parameters
        batch_size = 32
        n_epochs = 100
        
        def __init__(self, model_name, model_path, dataset_path, log_path='trainlogs'):
            self.model_name = model_name
            self.model_path = model_path
            self.log_path = log_path
            self.dataset_path = dataset_path
                
            # creating callbacks
            self.tfBoardCB = TensorBoard('{}/{}_{}'.format(self.log_path, model_name, time()), write_graph=True)
        
            filepath= model_path + model_name + '_best.h5'
        
            self.checkpointCB = ModelCheckpoint(filepath, monitor='val_loss', verbose=1, save_best_only=True, mode='min')
        
            
        def LoadDataset(self):
        
            print('loading dataset...')
            self.dataset = pd.read_csv(self.dataset_path + 'target.csv')
        
            self.dataset['imgpath'] = self.dataset.id.apply(file_path_from_db_id, args=("%d.bmp", self.dataset_path))
        
            self.train, self.test = train_test_split(self.dataset, test_size=0.2)
            self.valid, self.test = train_test_split(self.test, test_size=0.7)
            
            self.train_steps = int(self.train.shape[0] / self.batch_size)
            self.valid_steps = int(self.valid.shape[0] / self.batch_size)
            self.test_steps = int(self.test.shape[0] / self.batch_size)
            print('dataset loaded!')
        
        def Train(self):
        
            print('loading model...')
            self.model = models.vanilla(self.input_shape)
            self.model.summary()
            model_json = self.model.to_json()
            with open(self.model_path + self.model_name + '.json', "w") as json_file:
                    json_file.write(model_json)
                    
            plot_model(self.model, to_file=self.model_name + '.png', show_shapes=True)
            print('dataset size: ', self.train.shape[0],
                        'train_steps: ', self.train_steps, 
                        'valid steps: ', self.valid_steps, 
                        'test_steps: ', self.test_steps)
                
            print('hiperparameters:')
            print('batch size:{}'.format(self.batch_size))
                
        
            print('training...')
            self.model.fit_generator(
                                    generator_from_df(self.train, self.batch_size, self.im_shape, 'angular'),
                                    steps_per_epoch=self.train_steps, 
                                    epochs=self.n_epochs,
                                    validation_data=generator_from_df(self.valid, self.batch_size, self.im_shape, 'angular'),
                                    validation_steps=self.valid_steps,
                                    callbacks=[self.tfBoardCB, self.checkpointCB],
                                    verbose=2
                                    )
            print('finished training')
            self.score = self.model.evaluate_generator(
                                    generator_from_df(self.test, self.batch_size, self.im_shape, 'angular'), 
                                    steps=self.test_steps
                                    )
            print('loss: ', self.score)
        
        def SaveModel(self):
            # serialize model to JSON
            self.model_json = self.model.to_json()
            with open(self.model_name + '.json', "w") as json_file:
                    json_file.write(self.model_json)
            # serialize weights to HDF5
            self.model.save_weights(self.model_name + '.h5')
            print("Saved model to disk")
        
        
        
        
    if __name__ == '__main__':
            
        import argparse
        parser = argparse.ArgumentParser(description='Entrenamiento de una red neuronal convolucional')
        parser.add_argument("model_name", help="name of the model to train")
        parser.add_argument("model_path", help="path where the model files will be saved")
        parser.add_argument("dataset_path", help="path where the the dataset is saved")
            
        args = parser.parse_args()
            
        trainer = RobocarTrainer(args.model_name, args.model_path, args.dataset_path)
        trainer.LoadDataset()
        trainer.Train()
        trainer.SaveModel()
    
    \end{lstlisting}

\section{Aumentación de datos}

\begin{lstlisting}[title={augmentation.py},language=Python]
    from keras.preprocessing.image import load_img, ImageDataGenerator
    from keras.preprocessing.image import img_to_array
    
    from pandas.plotting import bootstrap_plot
    import numpy as np
    import pandas as pd
    import bcolz
    import threading
    
    import cv2
    
    from utils import *
    
    import os
    import sys
    import glob
    import shutil
    import matplotlib.pyplot as plt
    get_ipython().magic(u'matplotlib inline')
    
    dataset1 = pd.read_csv('../../datasets/dataset/target.csv')
    dataset1['imgpath'] = dataset1.id.apply(file_path_from_db_id)
    
    datagen = ImageDataGenerator(
            rotation_range=20,
            height_shift_range=0.2,
            shear_range=0.15,
            zoom_range=0.15,
            fill_mode='nearest')
    
    test_gen = batch_generator(dataset1, 1, (224,224), 'angular', process=False, shuffle=False)
    img, tg = test_gen.next()
    img = img[0]
    img = (img)*255
    plt.subplot(131)
    plt.imshow(img)
    
    for i, row in dataset1.sample(1).iterrows():
        img3 = img_to_array(load_img(row['imgpath']))
    
    plt.subplot(132)
    plt.imshow(img3*255)
    
    img2, tg2 = horizontal_flip(img, tg)
    plt.subplot(133)
    plt.imshow(img2)
    
    datagen = ImageDataGenerator(
            rotation_range=20,
            height_shift_range=0.2,
            shear_range=0.15,
            zoom_range=0.15,
            fill_mode='nearest')
    
    test_gen = batch_generator(dataset1, 1, (224,224), 'angular', process=False, shuffle=False)
    
    offset = dataset1.shape[0] + 1
    id_offset = dataset1['id'].max() + 1
    i = 0
    newData = []
    for idx, row in dataset1.iterrows():
        # generador auxiliar
        
        ### extraemos la imagen en una variable auxiliar
        img = img_to_array(load_img(row['imgpath']))
        target = row['angular']
        ### aplicamos la transformacion de acuerdo a ciertas condiciones
        ## si es cero, o muy cercano a 0 no aplicamos la transformacion
        if np.abs(target) > 0.09:
            ## transformacion horizontal
            hImg, hTarget = horizontal_flip(img, target) 
            # guardar imagen y entry en el dataframe
            filename = "dataset/" + str(id_offset + i) + ".bmp";
            hImg = cv2.cvtColor(hImg,cv2.COLOR_BGR2RGB)
            cv2.imwrite(filename, hImg)
            newData.append([(id_offset + i), 
                            row['linear'], 
                            hTarget,
                            filename])
            # incrementa indice para siguiente entrada
            i += 1
            
            ## transformacion aleatoria
            choice = np.random.choice([0,1])
            if choice == 1:
                rImg = datagen.flow(img.reshape((1,) + img.shape),y=None, batch_size=1).next()
                rImg = rImg[0]
                # guardar imagen y entry en el dataframe
                filename = "dataset/" + str(id_offset + i) + ".bmp";
                rImg = cv2.cvtColor(rImg,cv2.COLOR_BGR2RGB)
                cv2.imwrite(filename, rImg)
                newData.append([(id_offset + i), 
                                row['linear'], 
                                target,
                                filename])
                # incrementa indice para siguiente entrada
                i += 1
 
        ### guardamos la imagen en el directorio y el nuevo target en el d
    columns = ['id', 'linear', 'angular', 'imgpath']
    newDataset = pd.DataFrame(newData, columns=columns)
    
    
    newDataset = pd.DataFrame(newData, columns=columns)
    
    dataset = dataset1.append(newDataset, ignore_index=True)
    dataset.to_csv('augmented.csv')
    \end{lstlisting}

\section{Generación de datos}
\begin{lstlisting}[title={utils.py},language=Python]
    from keras.preprocessing.image import load_img
    from keras.preprocessing.image import img_to_array
    
    from keras.models import model_from_json
    
    import numpy as np
    import pandas as pd
    import bcolz
    import threading
    
    import cv2
    
    import os
    import sys
    import glob
    import shutil
    
    from sklearn.model_selection import train_test_split
    
    import models
    
    def generator_from_df(df, batch_size, target_size, target_column='target', features=None, process=True):
        print('generating minibatch!')
        nbatches, n_skipped_per_epoch = divmod(df.shape[0], batch_size)
        #print nbatches
        count = 1
        epoch = 0
        # New epoch.
        while 1:
            df = df.sample(frac=1) # shuffle in every epoch
            epoch += 1
            i, j = 0, batch_size
            # Mini-batches within epoch.
            mini_batches_completed = 0
            for _ in range(nbatches):
                sub = df.iloc[i:j]
                try:
                    if process == True:
                        X = np.array([(2 * (img_to_array(load_img(f, target_size=target_size)) / 255.0 - 0.5)) for f in sub.imgpath])
                    else:
                        X = np.array([((img_to_array(load_img(f, target_size=target_size)))) for f in sub.imgpath])
                        
                    Y = sub[target_column].values
                    # Simple model, one input, one output.
                    mini_batches_completed += 1
                    print ".",
                    
                    yield X, Y
    
                except IOError as err:
                    count -= 1
    
                i = j
                j += batch_size
                count += 1
    
\end{lstlisting}

\section{Nodo de Inferencia}

\begin{lstlisting}[title={neural\_node.py},language=Python]
#!/usr/bin/env python
from __future__ import print_function
import roslib
import rospkg
roslib.load_manifest('robocar')
import sys
import csv
import numpy as np
import rospy
import cv2
import message_filters
from std_msgs.msg import String, Float32
from sensor_msgs.msg import Image, Range, CompressedImage, Joy
from geometry_msgs.msg import Twist, TwistStamped
from cv_bridge import CvBridge, CvBridgeError

from keras.preprocessing.image import load_img
from keras.preprocessing.image import img_to_array

from keras.models import model_from_json

import pandas as pd
import threading

import tensorflow as tf
import os
import sys
import glob
import shutil

from keras.models import Sequential
from keras.layers import Dense, Dropout, Flatten
from keras.layers import Conv2D, MaxPooling2D
from keras import backend as K
from keras.callbacks import TensorBoard, ModelCheckpoint

from keras.utils import plot_model
import models
from models import custom_loss
import imutils

class AutoPilot:
    idx = 0
    
    dim = (224, 224)

    linear = 0
    angular_joy = 0

    def __init__(self, folder):
        rospack = rospkg.RosPack()
        pack_path = rospack.get_path('robocar')
        model_path = pack_path + '/scripts/simple2'
        
        # get ros params
        self.img_topic = rospy.get_param('img_topic', default='/camera/image/compressed')
        self.output_topic = rospy.get_param('output_topic', default='/neural_output')
        self.model_name = rospy.get_param('model', default=model_path)

        ## cargar la red neuronal en la memoria 
        # load json and create model
        json_file = open(self.model_name + '.json', 'r')
        loaded_model_json = json_file.read()
        json_file.close()
        self.model = model_from_json(loaded_model_json)

        # load weights into new model
        self.model.load_weights(self.model_name + "_best.h5")
        print("Loaded model from disk") 
        self.model.compile(loss='mse', optimizer='adam', metrics=['accuracy'])
        self.model.summary()
        self.graph = tf.get_default_graph()

        print('creando subs y pubs...')
        # image subscriber for the predictor
        self.image_sub = rospy.Subscriber(self.img_topic, CompressedImage, self.imCallback, queue_size=1)
        
        # float32 publisher for output 
        self.output_pub = rospy.Publisher(self.output_topic, Float32, queue_size=1)
    
    #this callback executes when the two subscribers sync
    def imCallback(self, img):
        """ este calback lee la imagen de la camara, la preprocesa y obtiene 
        una prediccion para el comando de control del robot"""

        # lee la imagen y la preprocesa
        np_image = cv2.imdecode(np.fromstring(img.data, np.uint8),cv2.IMREAD_COLOR)
        np_image = cv2.resize(np_image, self.dim, interpolation = cv2.INTER_AREA)
        #print (np_image.shape)
        np_image = (2 * (np_image / 255.0 - 0.5))
        x = np_image.reshape((1,) + np_image.shape)  # this is a Numpy array with shape (1, h,w, c)
        #print (x.shape)
        # obtiene la prediccion de la red neuronal
        angular = 0.0
        with self.graph.as_default():
            angular = self.model.predict(x, batch_size=1, verbose=0)

        angular = np.asscalar(angular.flatten())
        #print ("prediccion: ", angular)
        # crea el mensaje para el control del carro y publica 
        # msg = Twist()
        
        # msg.angular.z = angular
        # self.twist_pub.publish(msg)

        output_msg = Float32()
        output_msg.data = angular
        self.output_pub.publish(output_msg)

        
def main(args):
    rospy.init_node('neural_node', anonymous=True)
    stamper = AutoPilot(None)

    try:
        rospy.spin()
    except KeyboardInterrupt:
        print("shutting down")
    cv2.destroyAllWindows()

if __name__ == '__main__':
    main(sys.argv)

\end{lstlisting}

\section{Nodo de detección de obstáculos}
\begin{lstlisting}[title={obstacle\_node.py},language=Python]
    #!/usr/bin/env python
    from __future__ import print_function
    import roslib
    import rospkg
    import rospy
    roslib.load_manifest('robocar')
    
    from sensor_msgs.msg import Range
    from std_msgs.msg import Float32
    import sys
    import csv
    
    class ObstacleDetector:
        idx = 0
        # model_name = '/home/pepe/catkin_ws/src/robocar/scripts/simple2'
        dim = (224, 224)
    
        linear = 0
        angular_joy = 0
        max_acc = 0.3
    
        kp = 4.0
    
        def __init__(self, folder):
            # get ros params
            rospy.loginfo("Obstacle node init")
            self.range_topic = rospy.get_param('range_topic', default='/laser')
            self.output_topic = rospy.get_param('output_topic', default='/obstacle_output')
            self.stop_distance = rospy.get_param('stop_distance', default=0.15)
            # input subscriber for the predictor
            self.range_sub = rospy.Subscriber(self.range_topic, Range, self.RangeCallback, queue_size=1)
            
            # float32 publisher for output 
            self.output_pub = rospy.Publisher(self.output_topic, Float32, queue_size=1)
        
        #this callback executes when the two subscribers sync
        def RangeCallback(self, msg):
            """ este calback recibe el rango del sensor y calcula la salida 
            correspondiente
            """
            error = msg.range - self.stop_distance
            error = error * 2 if error < 0 else error 
            
            acceleration = self.kp * error
            acceleration = self.max_acc if acceleration > self.max_acc else acceleration
            acceleration = -self.max_acc * 2 if acceleration < -self.max_acc else acceleration
            acceleration = 0 if (acceleration < (self.stop_distance + 0.03)) and (acceleration > (self.stop_distance - 0.01)) else acceleration
    
            out_msg = Float32()
            out_msg.data = acceleration
            self.output_pub.publish(out_msg)
        
    def main(args):
        rospy.init_node('neural_node', anonymous=True)
        stamper = ObstacleDetector(None)
        rospy.spin()
       
    if __name__ == '__main__':
        main(sys.argv)
    
\end{lstlisting}

\section{Piloto Automático}

\begin{lstlisting}[title={pilot\_node.cpp},language=c++]
    #include <ros/ros.h>
    #include <boost/thread.hpp>
    #include <std_msgs/Float32.h>
    
    #include <std_msgs/String.h>
    #include <geometry_msgs/Twist.h>
    #include <geometry_msgs/TwistStamped.h>
    #include <sensor_msgs/Joy.h>
    
    #include <cmath>
    
    class Pilot
    {
    
      private:
        // data
        bool manual_;
        // topics from param server
        std::string neural_topic_;
        std::string obstacle_topic_;
        std::string joy_twist_topic_;
    
        std::string output_topic_;
    
        int freq_hz_;
        // messages
        geometry_msgs::Twist twist_msg_;          // simple twist message from joystick
        geometry_msgs::Twist manual_twist_msg_;          // simple twist message from joystick
        // geometry_msgs::TwistStamped stamped; // stamped twist for sync
    
        // subs and PUbs
        ros::Subscriber neural_sub_;
        ros::Subscriber obstacle_sub_;
        ros::Subscriber joy_twist_sub_;
    
        ros::Publisher twist_cmd_pub_;
    
        // subs callback
        // for neural steering
        void NeuralCallback(const std_msgs::Float32::ConstPtr &steering);
    
        // for obsTacle nOde
        void ObstacleCallback(const std_msgs::Float32::ConstPtr &acceleration);
    
        // for joy tELeop
        void JoyTwistCallback(const geometry_msgs::Twist::ConstPtr &joy_twist);
    
        // for find object 2d
        void ObjectCallback(const std_msgs::Float32::ConstPtr &steering);
    
      public:
        Pilot();
    
        ros::Timer timer;
    
        ros::NodeHandle nh_;
        void TimerCallback(const ros::TimerEvent &event);
        float GetInterval();
    };
    // constructor
    Pilot::Pilot() : manual_(false)
    {
        ROS_INFO("Iniciando nodos");
    
        nh_.param<int>("freq_hz", freq_hz_, 20);
        if (freq_hz_ <= 0)
        {
            ROS_WARN("Invalid frequency value, default: 20Hz");
            freq_hz_ = 20;
        }

        // messages are received in QUEues
        nh_.param<std::string>("neural_topic", neural_topic_, "/neural_output");
        nh_.param<std::string>("obstacle_topic", obstacle_topic_, "/obstacle_output");
        nh_.param<std::string>("joy_twist_topic", joy_twist_topic_, "/joy_cmd_vel");
    
        nh_.param<std::string>("output_topic", output_topic_, "/cmd_vel");
        
        // subscribe to nodes
        ROS_INFO("Creando subs y pubs");
        neural_sub_ = nh_.subscribe<std_msgs::Float32>(neural_topic_, 1, &Pilot::NeuralCallback, this);
        obstacle_sub_ = nh_.subscribe<std_msgs::Float32>(obstacle_topic_, 1, &Pilot::ObstacleCallback, this);
        joy_twist_sub_ = nh_.subscribe<geometry_msgs::Twist>(joy_twist_topic_, 1, &Pilot::JoyTwistCallback, this);

        // output topic por piLOt node
        twist_cmd_pub_ = nh_.advertise<geometry_msgs::Twist>(output_topic_, 1);
    
    }
    
    float Pilot::GetInterval()
    {
        return 1.0 / freq_hz_;
    }
    
    ///////////////////////////////////////
    // CALLBACKS //////////////////////////
    ///////////////////////////////////////
    
    void Pilot::NeuralCallback(const std_msgs::Float32::ConstPtr &steering)
    {
        // extract the command
        // put in the message
        // double angular = steering;
        twist_msg_.angular.z = steering->data;
    }
    
    void Pilot::ObstacleCallback(const std_msgs::Float32::ConstPtr &acceleration)
    {
        // double LinEAr = acceleration;
        twist_msg_.linear.x = acceleration->data;
    }
    
    void Pilot::JoyTwistCallback(const geometry_msgs::Twist::ConstPtr &joy_twist)
    {
        manual_ = true;
        manual_twist_msg_.linear.x = joy_twist->linear.x;
        manual_twist_msg_.angular.z = joy_twist->angular.z;
    }
    
    void Pilot::TimerCallback(const ros::TimerEvent &event)
    {
        if(abs(manual_twist_msg_.linear.x) > 0.01 || abs(manual_twist_msg_.angular.z) > 0.01)
        {
            twist_cmd_pub_.publish(manual_twist_msg_);
            manual_ = false;
        }
        else
        {
            twist_cmd_pub_.publish(twist_msg_);
            manual_ = false;
        }
    }
    
    
    int main(int argc, char **argv)
    {
        ros::init(argc, argv, "pilot_node");
        Pilot teleop_joy;
    
        teleop_joy.timer = teleop_joy.nh_.createTimer(ros::Duration(teleop_joy.GetInterval()),
                                                        &Pilot::TimerCallback,
                                                        &teleop_joy);
        ros::spin();
    }
\end{lstlisting}

\section{Control teleoperado con Joystick}

\begin{lstlisting}[title={joy\_teleop.cpp},language=c++]
    #include <ros/ros.h>
    #include <boost/thread.hpp>
    #include <geometry_msgs/Twist.h>
    #include <geometry_msgs/TwistStamped.h>
    #include <sensor_msgs/Joy.h>
    
    class TeleopRobocar
    {
      public:
        TeleopRobocar();
    
        ros::Timer timer;
    
        ros::NodeHandle nh_;
        void timerCallback(const ros::TimerEvent &event);
        float GetInterval();
    
      private:
        void joyCallback(const sensor_msgs::Joy::ConstPtr &joy);
    
        int linear_; // axis id
        int brake_;
        int angular_;
    
        double l_scale_;
        double a_scale_;
        double l_offset_;
    
        std::string output_topic;
    
        int freq_hz_;
    
        ros::Publisher vel_pub_;
        ros::Publisher velStamped_pub_;
        ros::Subscriber joy_sub_;
    
        // for callbacks
        geometry_msgs::Twist twist;          // simple twist message from joystick
        geometry_msgs::TwistStamped stamped; // stamped twist for sync
    };
    
    TeleopRobocar::TeleopRobocar()
    {
        nh_.param<int>("freq_hz", freq_hz_, 20);
        if (freq_hz_ <= 0)
        {
            ROS_WARN("Invalid frequency value, default: 20Hz");
            freq_hz_ = 20;
        }
    
        nh_.param<int>("axis_linear", linear_, 5);
        nh_.param<int>("axis_brake", brake_, 2);
        nh_.param<double>("scale_linear", l_scale_, -0.5);
        nh_.param<double>("offset_linear", l_offset_, 0.5);
        nh_.param<int>("axis_angular", angular_, 0);
        nh_.param<double>("scale_angular", a_scale_, -1.32);    // -0.34
        nh_.param<std::string>("output_topic", output_topic, "/joy_cmd_vel");
    
        vel_pub_ = nh_.advertise<geometry_msgs::Twist>(output_topic, 1);
        velStamped_pub_ = nh_.advertise<geometry_msgs::TwistStamped>("/stamped_cmd_vel", 1);
    
        joy_sub_ = nh_.subscribe<sensor_msgs::Joy>("joy", 1, &TeleopRobocar::joyCallback, this);
    }
    
    float TeleopRobocar::GetInterval()
    {
        return 1.0 / freq_hz_;
    }
    
    void TeleopRobocar::joyCallback(const sensor_msgs::Joy::ConstPtr &joy)
    {
    
        twist.angular.z = joy->axes[angular_] * a_scale_;
    
        double acceleration = l_scale_ * joy->axes[linear_] + l_offset_;
        ROS_DEBUG("%f", acceleration);
    
        double reverse = l_scale_ * joy->axes[brake_] + l_offset_;
        ROS_DEBUG("%f", reverse);
    
        twist.linear.x = acceleration - reverse;
    
        ROS_DEBUG("%f", twist.linear.x);
    
        // vel_pub_.publish(twist);
    }
    
    void TeleopRobocar::timerCallback(const ros::TimerEvent &event)
    {
        stamped.twist = twist;
        stamped.header.stamp = ros::Time::now();
    
        vel_pub_.publish(twist);
        velStamped_pub_.publish(stamped);
    }
    
    int main(int argc, char **argv)
    {
        ros::init(argc, argv, "joy_teleop");
        TeleopRobocar teleop_joy;
    
        teleop_joy.timer = teleop_joy.nh_.createTimer(ros::Duration(teleop_joy.GetInterval()),
                                                        &TeleopRobocar::timerCallback,
                                                        &teleop_joy);
        ros::spin();
    }
\end{lstlisting}
\section{Nodo del sensor de proximidad}

\begin{lstlisting}[title={laser\_node.cpp},language=c++]
    #include "VL53L0X.h"
    #include <ros/ros.h>
    #include <sensor_msgs/Range.h> // para el sensor
    // libreria para el sensor laser
    class LaserNode
    {
      public:
        LaserNode();
    
        ros::Timer timer;
    
        ros::NodeHandle nh_;
        void timerCallback(const ros::TimerEvent &event);
        float GetInterval();
      private:
        // sensor
        VL53L0X sensor_;
    
        int freq_hz_;
        double l_scale_;
        double a_scale_;
        double l_offset_;
    
        ros::Publisher laser_pub_;
        // msgs
        sensor_msgs::Range laser_msg_;
    };
    
    LaserNode::LaserNode()
    {
        nh_.param<int>("freq_hz", freq_hz_, 20);
        if (freq_hz_ <= 0)
        {
            ROS_WARN("Invalid frequency value, default: 20Hz");
            freq_hz_ = 20;
        }
    
        // init publisher
        laser_pub_ = nh_.advertise<sensor_msgs::Range>("laser", 1);
        
        // laser
        sensor_.initialize();
        sensor_.setTimeout(200);
        
        // init some fixed data 
        laser_msg_.radiation_type = 1;
        laser_msg_.header.frame_id = "laser";
        laser_msg_.field_of_view = 0.1;
        laser_msg_.min_range = 0.0;
        laser_msg_.max_range = 1.20;
    }
    
    float LaserNode::GetInterval()
    {
        return 1.0 / freq_hz_;
    }
    
    void LaserNode::timerCallback(const ros::TimerEvent &event)
    {
        uint16_t distance = sensor_.readRangeSingleMillimeters();
    
        if (!sensor_.timeoutOccurred())
        {
            // create message and publish
            laser_msg_.range = distance / 1000.0;
            laser_msg_.header.stamp = ros::Time::now();
            laser_pub_.publish(laser_msg_);
        }
        else
        {
            ROS_WARN("Laser sensor timeout");
        }
    }
    
    int main(int argc, char **argv)
    {
        ros::init(argc, argv, "laser_node");
        LaserNode laser_node;
        
        laser_node.timer = laser_node.nh_.createTimer(ros::Duration(laser_node.GetInterval()), 
                                                    &LaserNode::timerCallback, 
                                                    &laser_node);
        ros::spin();
    }
\end{lstlisting}


% \lhead[\thepage]{REFERENCIAS}
\rhead[REFERENCIAS]{\thepage}

\cleardoublepage
\addcontentsline{toc}{chapter}{Bibliografía}
\bibliographystyle{IEEEtran}
\bibliography{referencias}
\cleardoublepage
% \chapter{Bibliografía}
\addcontentsline{toc}{chapter}{Bibliografía}
\bibliographystyle{IEEEtran}
\bibliography{referencias}


\end{document}
